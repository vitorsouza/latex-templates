\chapter{Dicionário de Projeto}
\label{sec-dicionario}
\vspace{-1cm}

Esta seção apresenta as definições detalhadas das classes, descrevendo seus atributos e associações e servindo como um glossário do projeto. As definições são organizadas por subsistema, cada classe sendo apresentada em uma tabela separada. A coluna ``Obr.?'' indica com um ``x'' se o atributo é obrigatório (deve possuir um valor para se criar um objeto da classe).

Vale destacar que eventuais operações que estas classes vierem a ter não são listadas e descritas nesta fase do projeto. Além disso, na Seção~\ref{sec-modelo-estrutural}, algumas classes podem ser incluídas nos diagramas de outros subsistemas para ilustrar a relação entre eles. No dicionário de projeto, no entanto, classes são descritas apenas em seus subsistemas de origem.

\vitor{Criar uma subseção para cada subsistema. Dentro de cada subseção, criar uma tabela similar às tabelas~\ref{tbl-dicionario-subsistema-primeiro-classe-01} e~\ref{tbl-dicionario-subsistema-primeiro-classe-02}, abaixo, descrevendo as classes daquele subsistema. As linhas da tabela descrevem atributos e associações, explicando o que representam no domínio do problema. Os tipos devem ser genéricos (i.e., não específicos de uma linguagem de programação) e podem ilustrar também tipos específicos de domínio (ex.: CEP e CPF podem ser tipos de atributos). Quando a extremidade de uma associação possuir um nome, usar este nome na tabela (ex.: ``objetos'' na Tabela~\ref{tbl-dicionario-subsistema-primeiro-classe-01}), do contrário usar um nome genérico dependendo da cardinalidade da extremidade. Não é necessário referenciar as tabelas em texto, pois a legenda de cada uma já indica qual classe está sendo descrita.}



% Definir uma subseção para cada subsistema.	
\section{Subsistema 01}
\label{sec-dicionario-subsistema-primeiro}


% Tabela de dicionário de projeto referente a uma classe.
\begin{longtable}{|p{3.5cm}|c|c|p{8cm}|}
	\caption{Detalhamento da classe \emph{Classe 01}.}
	\label{tbl-dicionario-subsistema-primeiro-classe-01} \\\hline 
	
	% Cabeçalho e repetição do mesmo em cada nova página. Manter como está.
	\rowcolor{lightgray}
	\textbf{Propriedade} & \textbf{Tipo} & \textbf{Obr.?} & \textbf{Descrição} \\\hline
	\endfirsthead
	\hline
	\rowcolor{lightgray}
	\textbf{Propriedade} & \textbf{Tipo} & \textbf{Obr.?} & \textbf{Descrição} \\\hline
	\endhead
	
	% Especificar os atributos e associações da classe abaixo, substituindo os exemplos.
	atributo da classe 01 	& Texto 	& x & Descrição do atributo da classe 01. \\\hline
	classe 02 				& Classe 02 &	& Descrição da associação com a Classe 02. \\\hline 
	objetos 				& Classe 03 &	& Associação com Classe 03 possui nome ``objetos''. \\\hline 
\end{longtable}


% Tabela de dicionário de projeto referente a uma classe.
\begin{longtable}{|p{3.5cm}|c|c|p{8cm}|}
	\caption{Detalhamento da classe \emph{Classe 02}.}
	\label{tbl-dicionario-subsistema-primeiro-classe-02} \\\hline 
	
	% Cabeçalho e repetição do mesmo em cada nova página. Manter como está.
	\rowcolor{lightgray}
	\textbf{Propriedade} & \textbf{Tipo} & \textbf{Obr.?} & \textbf{Descrição} \\\hline
	\endfirsthead
	\hline
	\rowcolor{lightgray}
	\textbf{Propriedade} & \textbf{Tipo} & \textbf{Obr.?} & \textbf{Descrição} \\\hline
	\endhead
	
	% Especificar os atributos e associações da classe abaixo, substituindo os exemplos.
	um atributo				& Inteiro	& x & Descrição deste atributo. \\\hline
	outro atributo			& Real		& 	& Exemplos \\\hline
	acento não tem problema	& Data		& 	& de \\\hline
	nem espaço em branco	& Booleano	& x	& tipos de dados. \\\hline
	classes 01 				& Classe 01 &	& Descrição da associação com a Classe 01. \\\hline 
\end{longtable}



% Definir uma subseção para cada subsistema.	
\section{Subsistema 02}
\label{sec-dicionario-subsistema-segundo}

Etc...