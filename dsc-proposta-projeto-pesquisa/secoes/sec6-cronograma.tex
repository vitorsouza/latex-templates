% ==============================================================================
% Plano de Estudo no Exterior - Nome do Aluno
% Seção 6 - Cronograma
% ==============================================================================
\section{Cronograma}
\label{sec-crono}

\begin{center}
\footnotesize 
\begin{tabular}{ | c | c | c | c | c | c | c | c | c | c | c | c | c | c | c | c | c |}\hline
	\rowcolor{gray!30} & \multicolumn{16}{|c|}{\textbf{Trimestre}}\\\hhline{~|----------------}
	\rowcolor{gray!30}\textbf{Atividade} &  \multicolumn{4}{|c|}{\textbf{<<Ano1>>}} & \multicolumn{4}{|c|}{\textbf{<<Ano2>>}} & \multicolumn{4}{|c|}{\textbf{<<Ano3>>}} & \multicolumn{4}{|c|}{\textbf{<<Ano4>>}}\\\hhline{~|----------------}
	\rowcolor{gray!30} & \textbf{01} & \textbf{02} & \textbf{03} & \textbf{04} & \textbf{05} & \textbf{06} & \textbf{07} & \textbf{08} & \textbf{09} & \textbf{10} & \textbf{11} & \textbf{12} & \textbf{13} & \textbf{14} & \textbf{15} & \textbf{16}\\\hline
	\cellcolor{gray!30}\textbf{1} & & & & & & & & & & & & & & & & \\\hline
	\cellcolor{gray!30}\textbf{2} & & & & & & & & & & & & & & & & \\\hline
	\cellcolor{gray!30}\textbf{3} & & & & & & & & & & & & & & & & \\\hline
	\cellcolor{gray!30}\textbf{4} & & & & & & & & & & & & & & & & \\\hline
	\cellcolor{gray!30}\textbf{5} & & & & & & & & & & & & & & & & \\\hline
	\cellcolor{gray!30}\textbf{6} & & & & & & & & & & & & & & & & \\\hline
\end{tabular}
\end{center}

\instrucoes{\begin{itemize}
		\item O cronograma deve ser organizado em trimestres e deve contemplar os quatro anos de duração do curso,\footnote{Segundo o Regimento Interno do PPGI, o prazo máximo para concluir o doutorado é de 60 meses. No entanto, orienta-se que o cronograma do projeto de doutorado seja elaborado considerando-se 48 meses, que é o tempo considerado ideal para a conclusão do doutorado. Ao longo do trabalho, caso haja necessidade de estender o tempo de desenvolvimento, o aluno poderá utilizar o tempo ainda disponível até o prazo máximo permitido.} conforme tabela acima;
		\item Substitua <<Ano1>>, <<Ano2>>, <<Ano3>> e <<Ano4>> pelos anos correspondentes;
		\item Os números das atividades na primeira coluna devem corresponder às atividades descritas na Seção~\ref{sec-metodo} (Metodologia). Adicione/remova linhas se necessário, marque com um X os meses em que as atividades serão realizadas.
\end{itemize}}

