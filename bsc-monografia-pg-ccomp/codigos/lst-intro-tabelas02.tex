% Exemplo de tabela 03:
\begin{table}
\caption{Exemplo que mostra equações em duas colunas (adaptada de~\cite{souza-mylopoulos:spe13}).}
\label{tbl-intro-exemplo03}
\centering
\vspace{1mm}
\fbox{\begin{minipage}{.98\linewidth}
\begin{minipage}{0.51\linewidth}
\vspace{-4mm}
\begin{eqnarray}
\Delta \left( I_{AR1} / NoSM \right) \left[ 0, maxSM \right] > 0\\
\Delta \left( I_{AR2} / NoSM \right) \left[ 0, maxSM \right] > 0\\
\Delta \left( I_{AR3} / LoA \right) < 0\\
\end{eqnarray}
\vspace{-6mm}
\end{minipage}
\hspace{2mm}
\vline 
\begin{minipage}{0.41\linewidth}
\vspace{-4mm}
\begin{eqnarray}
\Delta \left( I_{AR11} / VP2 \right) < 0\\
\Delta \left( I_{AR12} / VP2 \right) > 0\\
\Delta \left( I_{AR6} / VP3 \right) > 0\\
\end{eqnarray}
\vspace{-6mm}
\end{minipage}
\end{minipage}}
\end{table}