% =====================================================================
% Modelo para Relatório Parcial de Iniciação Científica (em Português)
% Prof. Vítor E. Silva Souza - NEMO/UFES :: DI/UFES :: PPGI/UFES
%
% Baseado no modelo fornecido pela PRPPG/UFES:
% https://prppg.ufes.br/programa-institucional-de-ic-piic
%
% O modelo já preenche grande área e área CNPq para a Computação.
% Procurar por textos envolvidos em \hl{}, ler as instruções e 
% substituir pelo texto do relatório.
% =====================================================================
\documentclass[10pt, a4paper]{article}

% Seleção de códigos de fonte.
\usepackage[T1]{fontenc}

% Codificação do documento em Unicode.
\usepackage[utf8]{inputenc}

% Especifica o uso de português do Brasil.
\usepackage[portuges,brazilian]{babel}

% Controle das cores.
\usepackage[usenames,dvipsnames]{xcolor}

% Inclusão de gráficos.
\usepackage{graphicx}

% Posicionamento de elementos.
\usepackage{float}

% Melhor controle de layout em tabelas.
\usepackage{tabularx}

% Permite usar tabelas que ocupam mais de uma página
\usepackage{longtable}

% \rm is deprecated and should not be used in a LaTeX2e document
% http://tex.stackexchange.com/questions/151897/always-textrm-never-rm-a-counterexample
\renewcommand{\rm}{\textrm}

% Inclusão de símbolos não padrão.
\usepackage{amssymb}
\usepackage{eurosym}

% Para utilizar \eqref para referenciar equações.
\usepackage{amsmath}

% Permite mostrar figuras muito largas em modo paisagem com \begin{sidewaysfigure} ao invés de \begin{figure}.
\usepackage{rotating}

% Permite customizar listas enumeradas/com marcadores.
\usepackage{enumitem}

% Permite inserir hiperlinks com \url{}.
\usepackage{bigfoot}
\usepackage[hidelinks]{hyperref}

% Permite usar o comando \hl{} para evidenciar texto com fundo amarelo. Útil para chamar atenção a itens a fazer.
\usepackage{soulutf8}

% Permite inserir espaço em branco condicional (incluído no texto final só se necessário) em macros.
\usepackage{xspace}

% Permite inserir comentários para controle de revisão de documentos.
\usepackage[colorinlistoftodos, textwidth=20mm, textsize=footnotesize]{todonotes}
\newcommand{\aluno}[1]{\todo[author=\textbf{Aluno},color=green!30,caption={},inline]{#1}}
\newcommand{\vitor}[1]{\todo[author=\textbf{Vítor},color=red!30,caption={},inline]{#1}}

% Permite incluir listagens de código com o comando \lstinputlisting{}.
\usepackage{listings}
\usepackage{caption}
\DeclareCaptionFont{white}{\color{white}}
\DeclareCaptionFormat{listing}{\colorbox{gray}{\parbox{\textwidth}{#1#2#3}}}
\captionsetup[lstlisting]{format=listing,labelfont=white,textfont=white}
\renewcommand{\lstlistingname}{Listagem}
\definecolor{mygray}{rgb}{0.5,0.5,0.5}
\lstset{
	basicstyle=\scriptsize,
	breaklines=true,
	numbers=left,
	numbersep=5pt,
	numberstyle=\tiny\color{mygray}, 
	rulecolor=\color{black},
	showstringspaces=false,
	tabsize=2,
	inputencoding=utf8,
	extendedchars=true,
	literate=%
	{é}{{\'{e}}}1
	{è}{{\`{e}}}1
	{ê}{{\^{e}}}1
	{ë}{{\¨{e}}}1
	{É}{{\'{E}}}1
	{Ê}{{\^{E}}}1
	{û}{{\^{u}}}1
	{ù}{{\`{u}}}1
	{â}{{\^{a}}}1
	{à}{{\`{a}}}1
	{á}{{\'{a}}}1
	{ã}{{\~{a}}}1
	{Á}{{\'{A}}}1
	{Â}{{\^{A}}}1
	{Ã}{{\~{A}}}1
	{ç}{{\c{c}}}1
	{Ç}{{\c{C}}}1
	{õ}{{\~{o}}}1
	{ó}{{\'{o}}}1
	{ô}{{\^{o}}}1
	{Õ}{{\~{O}}}1
	{Ó}{{\'{O}}}1
	{Ô}{{\^{O}}}1
	{î}{{\^{i}}}1
	{Î}{{\^{I}}}1
	{í}{{\'{i}}}1
	{Í}{{\~{Í}}}1
}

% Configurações de bibliografia.
\usepackage{natbib}
\let\oldbibliography\thebibliography
\renewcommand{\thebibliography}[1]{\oldbibliography{#1}\setlength{\itemsep}{0pt}}

% Redefinição das fontes usadas no documento.
\usepackage{helvet}
\renewcommand{\familydefault}{\sfdefault}

% Define margens e espaçamentos personalizados para o documento.
\usepackage{anysize}
\marginsize{30mm}{30mm}{15mm}{15mm}
\renewcommand{\baselinestretch}{1.5}

% Início do documento.
\begin{document}

% Capa.
\begin{center}
	\begin{figure}[h!]
	\centering
	\includegraphics[scale=0.5]{figuras/ufes}
	\end{figure}

	{\bf \Large UNIVERSIDADE FEDERAL DO ESPÍRITO SANTO}\\
	Pró-Reitoria de Pesquisa e Pós-Graduação\\
	Departamento de Pesquisa\\
	{\bf \Large DESCRIÇÃO PARCIAL DAS METAS ALCANÇADAS}\\
	\ \\
	{\bf \Large IDENTIFICAÇÃO DO PROJETO DE PESQUISA}\\
	Grande Área CNPq: Ciências Exatas e da Terra \\
	Área CNPq: Ciência da Computação
	
	\vspace{1cm}
	
	{\bf \Large \hl{Título do Projeto}}\\
	{\bf  - - - - - - - - - - -}
\end{center}
%
Nome do grupo de pesquisa: \hl{Grupo} \\
Linha de pesquisa: \hl{Linha} \\
Pesquisador responsável (orientador): \hl{Orientador}

\vspace{1.5cm}

\begin{center}
	{\bf \Large \hl{Nome do sub-projeto do aluno}}\\
	{- - - - - - - - - - - - }
\end{center}
%
Nome do aluno: \hl{Aluno} \\
Curso e período: \hl{Curso}, \hl{N}\textordmasculine\ período

\clearpage




%%% Início de seção. %%%
\section*{Introdução e justificativa:}
\label{sec-intro}

\hl{Em, no máximo, 15 linhas resumir o problema abordado durante a vigência do projeto descrevendo sucintamente o racional do estudo.}




%%% Início de seção. %%%
\section*{Materiais e métodos utilizados:}

\hl{Em, no máximo, 15 linhas descrever sucintamente os métodos, as técnicas e as abordagens utilizadas no decorrer do primeiro semestre de vinculação ao PIBIC/PIVIC - UFES. Exemplo de referência:} \cite{guarino-et-al:hobook09} (in-line) ou \citep{guarino-et-al:hobook09}.




%%% Início de seção. %%%
\section*{Resultados e discussão:}

\hl{Em, no máximo, 15 linhas descrever os principais resultados obtidos no decorrer da pesquisa executada pelo aluno no primeiro semestre de vinculação ao PIBIC/PIVIC - UFES.}




%%% Início de seção. %%%
\section*{Metas futuras:}

\hl{Em, no máximo, 10 linhas descrever sucintamente as metas a serem alcançadas nos próximos seis (6)  meses de vínculo do aluno ao Programa:}
\begin{enumerate}
	\item ...
	
	\item ....
	
	\item ...
	
	\item ....
\end{enumerate}


% Bibliografia.
\bibliographystyle{hapalike2-NOand}
\bibliography{bibliografia}

\hl{No máximo 10 referências, nos moldes ABNT.}


% Fim do documento.
\end{document}