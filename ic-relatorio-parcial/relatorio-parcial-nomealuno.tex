% ==============================================================
% Modelo para Relatório Parcial de Iniciação Científica da UFES
% Prof. Vítor E. Silva Souza - DI/CT/UFES :: PPGI/CT/UFES
%
% Baseado no modelo fornecido pela PRPPG/UFES:
% https://prppg.ufes.br/programa-institucional-de-ic-piic
%
% Procure por (*) para instruções de preenchimento.
%
% This work may be distributed and/or modified under the 
% conditions of the LaTeX Project Public License, either version 
% 1.3 of this license or (at your option) any later version. The 
% latest version of this license is in 
% http://www.latex-project.org/lppl.txt.
% ==============================================================
\documentclass[12pt, a4paper]{article}

% Importa pacotes a serem utilizados no documento.
\usepackage[pdftex]{graphicx,color}
\usepackage[dvipsnames]{xcolor,colortbl}
\usepackage[hidelinks]{hyperref}
\usepackage{anysize}
\usepackage{graphicx}
\usepackage[utf8]{inputenc}
\usepackage[T1]{fontenc}
\usepackage[brazil]{babel}
\usepackage{fancyhdr}
\usepackage{ifthen}
\usepackage{array}
\usepackage{natbib}
\usepackage{amsmath}
\usepackage{enumerate}
\usepackage{adjustbox}
\usepackage{titlesec}
\usepackage{mathptmx}

% Reduz o espaço entre os itens da bibliografia
\let\oldbibliography\thebibliography
\renewcommand{\thebibliography}[1]{\oldbibliography{#1}
\setlength{\itemsep}{0pt}}

% Configura a fonte do documento.
\usepackage{helvet}
\renewcommand{\familydefault}{\sfdefault}

% Configura espaçamento 1,5 entre linhas.
\usepackage{setspace}
\onehalfspacing

% Configura espaçamento entre os parágrafos.
\usepackage{parskip}
\setlength{\parindent}{0pt}
\setlength{\parskip}{6pt} 
\linespread{1.25}

% Configura tamanho dos títulos.
\titleformat{\section}
  {\normalfont\fontsize{14}{16}\bfseries}{\thesection}{1em}{}[{\titlerule[0.8pt]}]

% Configura o tamanho e alinhamento das legendas.
\usepackage{caption}
\usepackage{subcaption}
\captionsetup{font={footnotesize}, labelsep=endash}
\captionsetup{singlelinecheck=false,justification=raggedright}
\captionsetup[subfigure]{justification=centering, singlelinecheck=true}
\captionsetup[figure]{skip=2pt}
\captionsetup[grafico]{skip=2pt}
\captionsetup[quadro]{skip=2pt}
\captionsetup[table]{skip=2pt}


% Cria um tipo coluna com alinhamento à esquerda.
\usepackage{tabularx}
\newcolumntype{L}[1]{>{\raggedright\arraybackslash}p{#1}}

% Define opção de espaçamento interno em tabelas.

% Define os elementos gráfico e quadro.
\usepackage{newfloat}
\DeclareFloatingEnvironment[name={Gráfico}]{grafico}
\DeclareFloatingEnvironment[name={Quadro}]{quadro}

% Configura listas enumeradas.
\usepackage{enumitem}
\setlist[enumerate]{nosep}

% Permite inclusão de notas, úteis para revisão.
\setlength{\marginparwidth}{2cm}
\usepackage[colorinlistoftodos, textwidth=20mm, textsize=footnotesize]{todonotes}
\newcommand{\aluno}[1]{\todo[author=\textbf{Aluno},color=green!30,caption={},inline]{#1}}
\newcommand{\professor}[1]{\todo[author=\textbf{Professor},color=red!30,caption={},inline]{#1}}

% Permite usar o comando \hl{} para evidenciar texto.
\usepackage{soul}

% Permite colocar espaços no final das macros quando necessário.
\usepackage{xspace}

% Definição de símbolos para o cronograma.
\usepackage{amssymb}
\usepackage{pifont}
\usepackage{tikz}
\newcommand{\totalmente}{\ding{51}\xspace}
\newcommand{\parcialmente}{%
\begin{tikzpicture}[baseline=-0.5ex]
    \draw (0,0) circle (0.8ex);
    \fill (0,0) -- (90:0.8ex) arc (90:270:0.8ex) -- cycle;
\end{tikzpicture}}
\newcommand{\naorealizada}{%
\begin{tikzpicture}[baseline=0ex, line width=0.25ex]
    \draw (0,0) -- (1.2ex,1.2ex);
    \draw (0,1.2ex) -- (1.2ex,0);
\end{tikzpicture}}


% Definição de macros.
\newcommand{\mnras}{Mon. Not. R. Astron. Soc.}
\newcommand{\aap}{Astronomy $\&$ Astrophysics}
\newcommand{\apjs}{ApJS}
\newcommand{\apj}{Astrophys. J.}
\newcommand{\apjl}{Astrophys. J. Letters}
\newcommand{\aj}{Astron. J.}
\newcommand{\pasa}{PASA}
\newcommand{\nat}{Nature}
\newcommand{\java}{Java\texttrademark\xspace}

% Configuração das margens.
\usepackage[a4paper, left=29mm, top=30mm, right=19mm, bottom=20mm, includehead, headsep=25mm]{geometry}
\addtolength{\voffset}{-25mm}
\addtolength{\textheight}{25mm}

% Definição do cabeçalho.
% (*) Substituir a área do conhecimento e remover o \color{Red}.
\usepackage{afterpage}
\pagestyle{fancy}
\fancyhf{}
\fancyhead[R]{\fontsize{10}{12}\selectfont\color{Gray} Universidade Federal do Espírito Santo\\ Programa Institucional de Iniciação Científica\\ Relatório Parcial de Pesquisa\\ {\color{Red} Área do Conhecimento (CNPq)}}
\renewcommand{\headrulewidth}{0pt}

% Definição do rodapé.
\fancyfoot[R]{\thepage}
\renewcommand{\footrulewidth}{0pt}

% Início do documento.
\begin{document}

% (*) Preencher {\bf } com dados do projeto e subprojeto.
\bgroup
\def\arraystretch{1.1}
\begin{tabularx}{\textwidth}{|>{\columncolor{gray!25}}L{61mm}|L{92mm}|}
\hline
{\bf Edital:} & {\bf Edital PIIC 20\_\_ /20\_\_} \\
\hline
{\bf Área do Conhecimento (CNPq):} & {\bf } \\
\hline
{\bf Subárea do Conhecimento (CNPq):} & {\bf } \\
\hline
{\bf Título do Projeto:} & {\bf } \\
\hline
{\bf Título do Subprojeto:} & {\bf } \\
\hline
{\bf Orientador(a):} & {\bf } \\
\hline
{\bf Estudante:} & {\bf } \\
\hline
\end{tabularx}
\egroup
\vspace{2mm}

% (*) Marcar X no lugar de \ do meio em (\ \ \ ).
% (*) Preencher { } em \multicolumn{2}{|l|}{ } se for o caso.
\bgroup
\def\arraystretch{1.1}
\begin{tabularx}{\textwidth}{|>{\columncolor{gray!25}}L{61mm}|L{44mm}|L{44mm}|}
\hline
{\bf Estudante apresenta alguma necessidade de acessibilidade?} & {\vspace{1mm} Sim (\ \ \ )} & {\vspace{1mm} Não (\ \ \ )} \\
\hline
{Se sim, informar o recurso ou suporte de acessibilidade que necessita.} & \multicolumn{2}{|l|}{ } \\
\hline
\end{tabularx}
\egroup
\vspace{.5cm}

% (*) Ler as orientações gerais e excluí-las na versão final.
{\color{Red} 
{\bf Orientações Gerais (remover na versão final)}
\begin{itemize}
	\item Este relatório deve ser elaborado pelo(a) estudante sob supervisão do(a) orientador(a).
	\item O relatório deve ser enviado pelo(a) orientador(a) via Sistema Acadêmico de Pesquisa (SAPPG), conforme as datas do edital.
	\item O atraso no envio pode resultar na suspensão da bolsa e em restrições em edições futuras.
	\item O relatório deve conter as seções: {\bf Introdução; Materiais e Métodos; Resultados e Discussão; Atividades Realizadas e Metas Futuras; Referências}.
	\item Formatação mínima obrigatória (já configurada neste modelo \LaTeX):
	\begin{itemize}
		\item Tamanho A4; margens: 3 cm (superior e esquerda) e 2 cm (inferior e direita)
		\item Fonte \textbf{Arial}, corpo \textbf{12} no texto e \textbf{14} nos títulos
		\item Espaçamento \textbf{1,5} entre linhas
		\item Alinhamento justificado
		\item Sem recuo na primeira linha dos parágrafos
		\item Cabeçalho em Arial 10
		\item Inserir, na \textbf{quarta linha do cabeçalho}, a \textbf{área do conhecimento conforme classificação do CNPq}
	\end{itemize}
	\item O documento deve ser exportado em \textbf{PDF}
\end{itemize}
}

% Início das seções.
% (*) Ler as instruções e exemplos e escrever seu relatório.
\section{Introdução}
\label{sec-intro}

A linguagem utilizada ao longo do trabalho deve ser, sempre que possível, técnica e impessoal, uma vez que se trata de um trabalho acadêmico. Deve-se evitar o uso de gírias e de termos de linguagem coloquial, bem como o uso das primeiras pessoas do singular e do plural (fiz ou fizemos...; obtive ou obtivemos...), e priorizar a estrutura passiva (fez-se...; foram obtidos...). O conteúdo de cada seção deve estar de acordo com as recomendações descritas neste modelo. Na Introdução, o autor deve apresentar, de forma concisa, uma contextualização do tema de sua pesquisa, mostrando sua relevância, justificando o tema escolhido, e descrevendo claramente a sua pergunta ou seu problema de pesquisa. Deve-se também ressaltar a ligação do Subprojeto de Iniciação Científica com o Projeto de Pesquisa ao qual está vinculado. Esta seção deve conter, \textbf{no máximo, 30 linhas}.




\section{Materiais e Métodos}
\label{sec-metodo}

Nesta seção devem ser descritos, de maneira sucinta, \textbf{em até 30 linhas}, o detalhamento da metodologia utilizada no decorrer da pesquisa executada pelo discente no primeiro semestre de vinculação ao PIIC/UFES, bem como os procedimentos de trabalho adotados, os materiais que foram utilizados, o tratamento da informação realizado e o procedimento estatístico aplicado, se for o caso.




\section{Resultados e Discussão}
\label{sec-resultados}

Esta seção deve explicitar, \textbf{em até 30 linhas}, os principais resultados obtidos no decorrer da pesquisa executada pelo discente no primeiro semestre de vinculação ao PIIC/UFES. Em todo o relatório, ilustrações (figuras, gráficos, fotos, fluxogramas, quadros e etc.) e tabelas podem também ser incluídas e devem ser utilizadas a fim de de organizar os dados, realizar comparações, sintetizar informações e dar suporte às análises. \textbf{Deve-se destacar que tais ilustrações ou tabelas não são computadas como linhas do texto}. As ilustrações e tabelas devem ser inseridas imediatamente após o trecho ao qual fazem referência. Estes elementos devem estar centralizados na página, com identificação na parte superior, em fonte Arial 10. Na parte inferior, deve constar a indicação da fonte consultada (elemento obrigatório, mesmo que seja ``produção do próprio autor''), legenda, notas e/ou outras informações necessárias à compreensão (se houver). O texto da parte inferior também deve ser formatado em fonte Arial 10, alinhado à borda esquerda da ilustração e limitado pela borda direita da mesma, como mostra a Figura~\ref{fig-exemplo} e o Gráfico~\ref{graf-exemplo}.

\begin{figure}[h!]
	\centering
	\caption{(a) Fotografia do modelo de edificação utilizado no experimento e (b) representação esquemática do comportamento do escoamento sobre a edificação}
	\label{fig-exemplo}
	\begin{subfigure}[b]{0.45\textwidth}
		\centering
		\includegraphics[width=\textwidth]{fig-exemploA}
		\caption{}
		\label{fig-exemploA}
	\end{subfigure}
	\hfill
	\begin{subfigure}[b]{0.45\textwidth}
		\centering
		\includegraphics[width=\textwidth]{fig-exemploB}
		\caption{}
		\label{fig-exemploB}
	\end{subfigure}
	\vspace{5pt}
	\caption*{Fonte: \cite{toledo-pereira:clacs2004}.}
	\caption*{\textit{Descrição da Figura~\ref{fig-exemplo}:} fotografia de um modelo reduzido de edificação e representação esquemática do comportamento do escoamento sobre a edificação.}
\end{figure}

No caso de figuras e tabelas, sugere-se a inserção de um \textbf{texto alternativo}, que é uma breve descrição textual que permite a pessoas com deficiência visual, que utilizam softwares leitores de tela, compreendam o conteúdo visual do relatório. 

{\bf Orientações para a Redação do texto alternativo:}
\begin{enumerate}
	\item \textbf{Seja Objetivo:} Comece pela informação mais importante. Evite expressões como "Imagem de..." ou "Foto de...", pois o leitor de tela já identifica o elemento como gráfico.
	\item \textbf{Hierarquia de Dados:} Em gráficos, descreva primeiro o tipo (ex: "Gráfico de barras"), o título e os valores de maior destaque ou tendência geral.
	\item \textbf{Contexto Científico:} Descreva apenas o que é relevante para a pesquisa. Se uma cor no gráfico indica uma variável específica, mencione-a; se for apenas decorativa, não é necessário descrevê-la.
	\item \textbf{Tabelas:} Certifique-se de que a descrição resuma a principal conclusão que os dados da tabela apresentam.
\end{enumerate}

\begin{grafico}[h!]
	\centering
	\caption{Consumo final de energia por fonte no Brasil em 2011}
	\label{graf-exemplo}
	\includegraphics[width=.75\textwidth]{graf-exemplo}	
	\vspace{5pt}
	\caption*{Fonte: \cite{epe:ben2011}.}
	\caption*{Nota: $^1$ Inclui biodiesel. $^2$ Inclui apenas gasolina A (automotiva). $^3$ Inclui gás de refinaria, coque de carvão mineral e carvão vegetal, dentre outros.}
	\caption*{\textit{Descrição do Gráfico~\ref{graf-exemplo}:} gráfico de setores (estilo pizza) com efeito 3D, intitulado ``Consumo final de energia por fonte no Brasil em 2011''. O gráfico utiliza diversas cores para distinguir doze categorias de fontes energéticas. A maior fonte de consumo é o óleo diesel, com 19,1\%.}
\end{grafico}

As tabelas e os quadros, apesar de possuírem certa semelhança entre si, diferenciam-se não apenas no formato exigido, mas também pelo conteúdo que exibem:

\begin{enumerate}[label=\alph*)]
	\item um quadro apresenta informações ou resultados qualitativos, ou seja, em forma de texto, mesmo que este empregue números;
	\item uma tabela apresenta informações ou resultados quantitativos, ou seja, números tratados estatisticamente. 
\end{enumerate}

Quanto ao formato e à apresentação de tabelas e quadros, como se verifica na Tabela~\ref{tbl-exemplo} e no Quadro~\ref{qdr-exemplo}, devem-se observar as seguintes regras:

\begin{enumerate}[label=\alph*)]
	\item a moldura das tabelas não deve ser fechada com traços verticais à esquerda e à direita;
	\item deve-se evitar o uso de traços verticais para separar as colunas e de traços horizontais para separar as linhas de uma tabela;
	\item o quadro é um elemento fechado, portanto, deve conter traços horizontais e verticais para separar suas linhas e colunas, além de traços horizontais e verticais para delimitar sua moldura.
\end{enumerate}

\begin{table}[h]
	\fontsize{10}{12}
	\centering
	\caption{Exemplo de formatação de uma tabela para a apresentação de resultados}
	\label{tbl-exemplo}
	\begin{tabular}{c c c}
		\textbf{Grupo de idade (em meses)} & \textbf{Número de indivíduos no grupo} & \textbf{Indivíduos viáveis [\%]} \\
		\hline
		0--10		& 20	& 9,0  \\
		10--15		& 20	& 10,0 \\
		15--20		& 25	& 4,0  \\
		Acima de 20	& 15	& 3,4  \\
		\hline
	\end{tabular}
	\vspace{5pt}
	\caption*{Fonte: Produção do(a) próprio(a) autor(a).}
	\caption*{\textit{Descrição da Tabela~\ref{tbl-exemplo}:} tabela com bordas horizontais simples no topo e na base, com dados ilustrativos para demonstrar as normas de apresentação de dados.}
\end{table}

\begin{quadro}[h]
	\fontsize{10}{12}
	\centering
	\caption{Dimensionamento dos elementos de um conversor \textit{boost}}
	\label{qdr-exemplo}
	\begin{tabular}{|p{60mm}|p{60mm}|}
		\hline
		\rowcolor{gray!25}
		\centering \textbf{Elemento ou Grandeza} &  \centering \textbf{Valor ou Modelo}
		\tabularnewline
		\hline
		Tensão de entrada 		& $48 V$
		\\\hline
		Tensão de saída 		& $200 V$
		\\\hline
		Potência de saída 		& $200 W$
		\\\hline
		Frequência de comutação	& $50 kHz$
		\\\hline
		Indutor de entrada 		& $880 \mu H$
	\\\hline
		Capacitor de saída 		& $22 \mu F$
	\\\hline
		Diodo					& $FES8HT$
		\\\hline
		Interruptor				& $IRFP360$	
	\\\hline
	\end{tabular}
	\vspace{5pt}
	\caption*{Fonte: \cite{menegaz:thesis2005}.}
	\caption*{\textit{Descrição do Quadro~\ref{qdr-exemplo}:} ...}
\end{quadro}

Outros elementos textuais que podem fazer parte do Relatório Parcial de Pesquisa são as equações e fórmulas. Para facilitar a leitura, recomenda-se que as equações sejam destacadas do texto e numeradas com algarismos arábicos entre parênteses, alinhados à margem direita da página, como mostra a Equação~\eqref{eqn-exemplo}. Assim como no caso de figuras, tabelas e quadros, a citação, ou a chamada, de todas as equações ou fórmulas no texto é obrigatória, e sua localização deve acontecer o mais próximo possível do trecho onde são mencionadas pela primeira vez.

\begin{equation}
	\label{eqn-exemplo}
	v_{r}(t) = R \times i(t)
\end{equation}
{\small\textit{Descrição da Equação~\eqref{eqn-exemplo}:} equação matemática centralizada, identificada pelo número 1 entre parênteses à direita. A fórmula descreve uma relação ilustrativa para demonstrar as normas de apresentação dos dados.}




\section{Atividades Realizadas e Metas Futuras}
\label{sec-atividades}

Nesta seção deve-se resgatar as atividades previstas do Subprojeto de Pesquisa e explicitar aquelas que foram totalmente realizadas (\totalmente), parcialmente realizadas (\parcialmente) ou não foram realizadas (\naorealizada), conforme o Quadro~\ref{qdr-cronograma}:

\begin{quadro}[h]
	\centering
	\caption{Cronograma de atividades previstas e realizadas do Subprojeto (set./20\_\_ a ago./20\_\_)}
	\label{qdr-cronograma}
	\footnotesize
	\renewcommand{\arraystretch}{2}
	\setlength{\tabcolsep}{3pt}
	\begin{tabular}{| p{5.5cm} *{12}{| c} |}
		\hline
		\rowcolor{lightgray}
		\textbf{Atividade} & \textbf{set.} & \textbf{out.} & \textbf{nov.} & \textbf{dez.} & \textbf{jan.} & \textbf{fev.} & \textbf{mar.} & \textbf{abr.} & \textbf{mai.} & \textbf{jun.} & \textbf{jul.} & \textbf{ago.} \\
		\hline
		a) Xxxx xxx xxxxxx &  & \totalmente  &   &    &  &   &   &   &   &   &   &  \\
		\hline
		b) Yyyy yyy yyy yy yyy &  &   &   & \parcialmente  &   &   &    &    &   &   &   &  \\
		\hline
		c) Zzzzz z zzzz &   &   &   &   &   &   &   &   &    &   &  & \\
		\hline
		d) Wwww ww www wwww www & \naorealizada  &   &   &   &   &   &   &   &    &   &  & \\
		\hline
	\end{tabular}
	\vspace{5pt}
	\caption*{Fonte: Produção do próprio autor.\\Nota: atividade totalmente realizada (\totalmente), parcialmente realizada (\parcialmente) e não realizada (\naorealizada).}
	\caption*{\textit{Descrição do Quadro~\ref{qdr-cronograma}:} Cronograma de atividades previstas e realizadas no período de vigência do subprojeto. As atividades devem ser registradas conforme execução (atividade totalmente realizada, atividade parcialmente realizada e atividade não realizada.}
\end{quadro}

\textbf{A execução parcial ou a não realização de uma determinada atividade deve ser justificada.} Além disso, as metas a serem alcançadas, nos 6 (seis) meses seguintes devem também ser sucintamente descritas.

Toda a seção deve ter, no máximo, 20 linhas, sem contar com o quadro que contém o cronograma de atividades.

\textbf{Sobre a seção Referências (apenas informativo, apagar este trecho no texto final): }
a seção Referências (a seguir, gerada pelo BibTeX) deve descrever as fontes consultadas durante esta primeira etapa de desenvolvimento do Subprojeto de Iniciação Científica, seguindo a norma técnica pertinente, a saber a NBR 6023:2025. Deve conter apenas as obras citadas no texto, ou seja, ``não liste se não citar'' e ``não cite se não listar'' (o que o BibTeX deve fazer por você). A lista das referências bibliográficas deve estar em ordem alfabética. A \href{https://biblioteca.ufes.br/normalizacao}{Biblioteca Central da Ufes (BC-UFES)} orienta a aplicação da nova ABNT NBR 6023:2025.


%\bibliographystyle{apalike2}
\bibliographystyle{hapalike2-NOand}

\renewcommand{\bibsection}{\section*{Referências}}
\bibliography{biblio}


\end{document}



