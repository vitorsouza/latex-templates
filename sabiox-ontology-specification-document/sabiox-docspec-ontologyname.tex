\documentclass[table,usenames,dvipsnames]{article}

%
% LaTeX Packages
%

% Determines the paper size and the size of the contents.
\usepackage[a4paper, total={170mm, 257mm}]{geometry}

% To insert blank spaces in macros.
\usepackage{xspace}

% To insert colored comments so authors can collaborate on the content.
\usepackage[colorinlistoftodos, textwidth=20mm, textsize=footnotesize]{todonotes}
\newcommand{\camila}[1]{\todo[author=\textbf{Camila},color=green!30,caption={},inline]{#1}}
\newcommand{\vitor}[1]{\todo[author=\textbf{Vítor},color=red!30,caption={},inline]{#1}}

% To use the \hl{} command to highlight pieces of text.
\usepackage{soulutf8}

% To change the margins in paragraphs.
\usepackage{changepage}

% To define a box with colored background.
\usepackage{tcolorbox}

% To insert hyperlinks and use \nameref{}. Hides the default red boxes for links.
\usepackage[hidelinks]{hyperref}

% To include tables that span across pages.
\usepackage{longtable}

% To have cells span multiple rows in tables.
\usepackage{multirow}

% To include listings with the command \lstinputlisting{}.
\usepackage{listings}
\usepackage{caption}
\definecolor{mygray}{rgb}{0.5,0.5,0.5}
\lstset{
	xrightmargin=0.05\linewidth,
	basicstyle=\scriptsize\linespread{0.8},
	breaklines=true,
	breakatwhitespace=false,
	showstringspaces=false,
	keepspaces=true,
	showspaces=false,
	showtabs=false, 
	numbers=left,
	numbersep=5pt,
	numberstyle=\tiny\color{mygray}, 
	rulecolor=\color{black},	
	tabsize=6,
	inputencoding=utf8,
	extendedchars=true,
	literate=%
	{é}{{\'{e}}}1
	{è}{{\`{e}}}1
	{ê}{{\^{e}}}1
	{ë}{{\¨{e}}}1
	{É}{{\'{E}}}1
	{Ê}{{\^{E}}}1
	{û}{{\^{u}}}1
	{ù}{{\`{u}}}1
	{â}{{\^{a}}}1
	{à}{{\`{a}}}1
	{á}{{\'{a}}}1
	{ã}{{\~{a}}}1
	{Á}{{\'{A}}}1
	{Â}{{\^{A}}}1
	{Ã}{{\~{A}}}1
	{ç}{{\c{c}}}1
	{Ç}{{\c{C}}}1
	{õ}{{\~{o}}}1
	{ó}{{\'{o}}}1
	{ô}{{\^{o}}}1
	{Õ}{{\~{O}}}1
	{Ó}{{\'{O}}}1
	{Ô}{{\^{O}}}1
	{î}{{\^{i}}}1
	{Î}{{\^{I}}}1
	{í}{{\'{i}}}1
	{Í}{{\~{Í}}}1
}



%
% Macros.
%

% The name of the method.
\newcommand{\sabiox}{SABiOx\xspace}
\newcommand{\sabioxfull}{Extended Systematic Approach for Building Ontologies\xspace}

% Document meta-data.
\newcommand{\ontologyacronym}{YA-O\xspace}
\newcommand{\ontologyname}{Yet Another Ontology\xspace}
\newcommand{\authorname}{\sabiox User\xspace}
\newcommand{\documentversion}{0.1\xspace}


% Title Page
\title{\ontologyacronym: \ontologyname
	\\{\large Ontology Specification Document}
	\\{\normalsize Version: \documentversion}}
\author{\authorname}



% Document contents.
\begin{document}
\maketitle


\section{Introduction}

This document presents the specification of \ontologyname (\ontologyacronym) as result of the Requirements phase of \sabiox: the \sabioxfull~\cite{aguiar-souza:report24}.

This document is organized as follows:
Section~\ref{sec-purpose} describes the purpose and intended uses of the ontology;
Section~\ref{sec-domain} sets the ontlogy domain;
Section~\ref{sec-dimension} establishes the scope of the ontology on such domain, in both horizontal and vertical dimensions; finally, 
sections~\ref{sec-freqs} and~\ref{sec-nfreqs} list the ontology's functional and non-functional requirements, respectively.



\section{Purpose of the Ontology:}
\label{sec-purpose}

The purpose of the ontology is ...



\section{Ontology Domain:}
\label{sec-domain}

Describe the domain of the ontology here.



\section{Ontology Dimension:}
\label{sec-dimension}

In the horizontal dimension, the domain is defined as ...; for the vertical dimension, the domain is limited in representing ...



\section{Functional Requirements:}
\label{sec-freqs}

% Define counter and ID for functional requirements.
\newcounter{frcount}
\renewcommand*\thefrcount{FR-\arabic{frcount}}
\newcommand*\FR{\refstepcounter{frcount}\thefrcount}
\setcounter{frcount}{0}

% Define a macro for a functional requirement.
\newcommand{\freq}[2]{\FR\label{#1} & #2 \\\hline}


\subsection*{Subdomain: Some Subdomain}

\begin{center}
	\begin{small}
		\begin{longtable}{ p{15mm} p{145mm} }
			\hline
			\textbf{ID} & \textbf{Description} \\\hline
			
			\freq{fr-first-example}{
				Lorem ipsum dolor sit amet, consectetur adipiscing elit. Nunc purus sem, rutrum eget enim eu, vestibulum maximus nisl. Phasellus tristique purus a magna aliquet dapibus.
			}
			
			\freq{fr-second-example}{
				Maecenas nunc diam, accumsan id commodo a, maximus ultrices metus. Sed molestie imperdiet massa, quis mollis felis facilisis ac. In vel turpis et diam sodales commodo eget sed enim.
			}
		\end{longtable}
	\end{small}
\end{center}


\subsection*{Subdomain: Some Other Subdomain}

\begin{center}
\begin{small}
	\begin{longtable}{ p{15mm} p{145mm} }
		\hline
		\textbf{ID} & \textbf{Description} \\\hline
		
		\freq{fr-third-example}{
			Proin sagittis arcu erat, at aliquam tellus egestas ac. Mauris pulvinar molestie dictum. 
		}
		
		\freq{fr-fourth-example}{
			Phasellus scelerisque orci id sapien molestie ullamcorper. Sed quis urna quis orci ultrices congue. 
		}
		
		\freq{fr-fifth-example}{
			Ut eu eros ut erat congue efficitur ac eget sem. Mauris porttitor ligula vulputate dui pulvinar condimentum. 
		}
	\end{longtable}
\end{small}
\end{center}



\section{Non-Functional Requirements:}
\label{sec-nfreqs}

% Define counter and ID for non-functional requirements.
\newcounter{nfrcount}
\renewcommand*\thenfrcount{NFR-\arabic{nfrcount}}
\newcommand*\NFR{\refstepcounter{nfrcount}\thenfrcount}
\setcounter{nfrcount}{0}

% Define a macro for a functional requirement.
\newcommand{\nfreq}[2]{\NFR\label{#1} & #2 \\\hline}

			
\begin{center}
	\begin{small}
		\begin{longtable}{ p{15mm} p{145mm} }
			\hline
			\textbf{ID} & \textbf{Description} \\\hline
			
			\nfreq{nfr-first-example}{
				Mauris tempor orci at semper rhoncus.
			}
			
			\nfreq{nfr-second-example}{
				Mauris auctor, nulla ut malesuada tristique, purus tellus tempor lectus, aliquam aliquet enim tellus sit amet enim.
			}
		\end{longtable}
	\end{small}
\end{center}



% Bibliography
\bibliographystyle{alpha}
\bibliography{bibliography}

\end{document}          
