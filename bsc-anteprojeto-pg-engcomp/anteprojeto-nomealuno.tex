% =========================================================
% Modelo para Anteprojeto de Graduação - Engenharia de Computação (em Português)
% Prof. Vítor E. Silva Souza - NEMO / PPGI / DI / UFES
%
% Baseado em abtex2-modelo-trabalho-academico.tex, v-1.9.2 laurocesar
% Copyright 2012-2014 by abnTeX2 group at http://abntex2.googlecode.com/ 
%
% This work may be distributed and/or modified under the conditions of the LaTeX 
% Project Public License, either version 1.3 of this license or (at your option) 
% any later version. The latest version of this license is in
% http://www.latex-project.org/lppl.txt.
%
% IMPORTANTE:
% Instruções encontram-se espalhadas pelo documento. Para facilitar sua leitura,
% tais instruções são precedidas por (*) -- utilize a função localizar do seu
% editor para passar por todas elas.
% =========================================================

% Usa o estilo abntex2, configurando detalhes de formatação e hifenização.
\documentclass[
article,           % Seções ao invés de capítulos.
12pt,              % Tamanho da fonte.
oneside,         % Para impressão em apenas um lado (não inclui páginas em branco).
a4paper,        % Tamanho do papel.
english,			% Idioma adicional para hifenização.
french,				% Idioma adicional para hifenização.
spanish,			% Idioma adicional para hifenização.
brazil				% O último idioma é o principal do documento.
]{abntex2}




%%% Importação de pacotes. %%%

% Conserta o erro "No room for a new \count"
% Em alguns sistemas operacionais, a linha \reserveinserts{28} deve ser comentada.
\usepackage{etex}
%\reserveinserts{28}

% Usa a fonte Latin Modern.
\usepackage{lmodern}

% Seleção de códigos de fonte.
\usepackage[T1]{fontenc}

% Codificação do documento em Unicode.
\usepackage[utf8]{inputenc}

% Indenta o primeiro parágrafo de cada seção.
\usepackage{indentfirst}

% Controle das cores.
\usepackage[usenames,dvipsnames]{xcolor}

% Inclusão de gráficos.
\usepackage{graphicx}

% Posicionamento de elementos.
\usepackage{float}

% Melhor controle de layout em tabelas.
\usepackage{tabularx}
\usepackage{multirow}
\usepackage{hhline}

% Para melhorias de justificação.
\usepackage{microtype}

% Citações padrão ABNT.
\usepackage[brazilian,hyperpageref]{backref}
\usepackage[alf]{abntex2cite}	
\renewcommand{\backrefpagesname}{Citado na(s) página(s):~}		% Usado sem a opção hyperpageref de backref.
\renewcommand{\backref}{}										% Texto padrão antes do número das páginas.
\renewcommand*{\backrefalt}[4]{									% Define os textos da citação.
	\ifcase #1
	Nenhuma citação no texto.
	\or
	Citado na página #2.
	\else
	Citado #1 vezes nas páginas #2.
	\fi}

% \rm is deprecated and should not be used in a LaTeX2e document
% http://tex.stackexchange.com/questions/151897/always-textrm-never-rm-a-counterexample
\renewcommand{\rm}{\textrm}

% Inclusão de símbolos não padrão.
\usepackage{amssymb}
\usepackage{eurosym}

% Para utilizar \eqref para referenciar equações.
\usepackage{amsmath}

% Permite mostrar figuras muito largas em modo paisagem com \begin{sidewaysfigure} ao invés de \begin{figure}.
\usepackage{rotating}

% Permite customizar listas enumeradas/com marcadores.
\usepackage{enumitem}

% Permite inserir hiperlinks com \url{}.
\usepackage{bigfoot}
\usepackage{hyperref}

% Permite usar o comando \hl{} para evidenciar texto com fundo amarelo. Útil para chamar atenção a itens a fazer.
\usepackage{soulutf8}

% Permite inserir espaço em branco condicional (incluído no texto final só se necessário) em macros.
\usepackage{xspace}

% Permite inserir comentários para controle de revisão de documentos.
\usepackage[colorinlistoftodos, textwidth=20mm, textsize=footnotesize]{todonotes}
\newcommand{\aluno}[1]{\todo[author=\textbf{Aluno},color=green!30,caption={},inline]{#1}}
\newcommand{\professor}[1]{\todo[author=\textbf{Professor},color=red!30,caption={},inline]{#1}}

% Permite incluir listagens de código com o comando \lstinputlisting{}.
\usepackage{listings}
\usepackage{caption}
\DeclareCaptionFont{white}{\color{white}}
\DeclareCaptionFormat{listing}{\colorbox{gray}{\parbox{\textwidth}{#1#2#3}}}
\captionsetup[lstlisting]{format=listing,labelfont=white,textfont=white}
\renewcommand{\lstlistingname}{Listagem}
\definecolor{mygray}{rgb}{0.5,0.5,0.5}
\lstset{
	basicstyle=\scriptsize,
	breaklines=true,
	numbers=left,
	numbersep=5pt,
	numberstyle=\tiny\color{mygray}, 
	rulecolor=\color{black},
	showstringspaces=false,
	tabsize=2,
	inputencoding=utf8,
	extendedchars=true,
	literate=%
	{é}{{\'{e}}}1
	{è}{{\`{e}}}1
	{ê}{{\^{e}}}1
	{ë}{{\¨{e}}}1
	{É}{{\'{E}}}1
	{Ê}{{\^{E}}}1
	{û}{{\^{u}}}1
	{ù}{{\`{u}}}1
	{â}{{\^{a}}}1
	{à}{{\`{a}}}1
	{á}{{\'{a}}}1
	{ã}{{\~{a}}}1
	{Á}{{\'{A}}}1
	{Â}{{\^{A}}}1
	{Ã}{{\~{A}}}1
	{ç}{{\c{c}}}1
	{Ç}{{\c{C}}}1
	{õ}{{\~{o}}}1
	{ó}{{\'{o}}}1
	{ô}{{\^{o}}}1
	{Õ}{{\~{O}}}1
	{Ó}{{\'{O}}}1
	{Ô}{{\^{O}}}1
	{î}{{\^{i}}}1
	{Î}{{\^{I}}}1
	{í}{{\'{i}}}1
	{Í}{{\~{Í}}}1
}




%%% Definição de variáveis. %%%

\renewcommand{\imprimircapa}{%
	\begin{capa}%
		\center

		{\ABNTEXchapterfont\large
			Centro Tecnológico
			\par
			Colegiado do Curso de Engenharia de Computação
			\par
			Coordenação de Projeto de Graduação
			\par
			Disciplina Projeto de Graduação I
			\vfill
			\textbf{ANTEPROJETO}
			\par
		
		}
		\vfill
		\begin{center}
			\ABNTEXchapterfont\bfseries\LARGE\imprimirtitulo
		\end{center}
		
		\hspace{.45\textwidth}
		\begin{minipage}{.5\textwidth}
			\flushright
			
			% (*) Substituir os textos abaixo com as informações apropriadas.
			\assinatura{\textbf{Aluno: Nome do Aluno}}
			\assinatura{\textbf{Orientador: Nome do Professor}}
		\end{minipage}
		
		\vfill
		\large\imprimirlocal
		\linebreak
		\large\imprimirdata
		\vspace*{1cm}
	\end{capa}
}

\newcommand{\versao}{2.0}
\newcommand{\subtitulo}{Anteprojeto}

% (*) Substituir os textos abaixo com as informações apropriadas.
\titulo{Título}
\autor{Nome do Aluno}
\local{Vitória, ES}
\data{9999}

\instituicao{
  Universidade Federal do Espírito Santo -- UFES
  \par
  Centro Tecnológico
  \par
  Departamento de Informática}
\tipotrabalho{Monografia (PG)}







% Macros específicas do trabalho.
% (*) Inclua aqui termos que são utilizados muitas vezes e que demandam formatação especial.
% Os exemplos abaixo incluem i* (substituindo o asterisco por uma estrela) e Java com TM em superscript.
% Use sempre \xspace para que o LaTeX inclua espaço em branco após a macro somente quando necessário.
\newcommand{\istar}{\textit{i}$^\star$\xspace}
\newcommand{\java}{Java\texttrademark\xspace}
\newcommand{\latex}{\LaTeX\xspace}




%%% Configurações finais de aparência. %%%

% Altera o aspecto da cor azul.
\definecolor{blue}{RGB}{41,5,195}

% Informações do PDF.
\makeatletter
\hypersetup{
	pdftitle={\@title}, 
	pdfauthor={\@author},
	pdfsubject={\imprimirpreambulo},
	pdfcreator={LaTeX with abnTeX2},
	pdfkeywords={abnt}{latex}{abntex}{abntex2}{trabalho acadêmico}, 
	colorlinks=true,				% Colore os links (ao invés de usar caixas).
	linkcolor=blue,					% Cor dos links.
	citecolor=blue,					% Cor dos links na bibliografia.
	filecolor=magenta,				% Cor dos links de arquivo.
	urlcolor=blue,					% Cor das URLs.
	bookmarksdepth=4
}
\makeatother

% Espaçamentos entre linhas e parágrafos.
\setlength{\parindent}{1.3cm}
\setlength{\parskip}{0.2cm}



%%% Páginas iniciais do documento: capa, folha de rosto, ficha, resumo, tabelas, etc. %%%

% Compila o índice. <--- Desnecessário em Plano de Estudo.
%\makeindex

% Inicia o documento.
\begin{document}
	
% Retira espaço extra obsoleto entre as frases.
\frenchspacing


% Brasão da instituição.
\begin{figure}[h]
	\centering
	\includegraphics[scale=0.055]{figuras/brasao}
\end{figure} 

% Capa do trabalho.
\imprimircapa

% Lista de silgas.
% (*) Indicar as siglas utilizadas no trabalho como no exemplo abaixo.
\begin{siglas}
	\item [UML] Linguagem de Modelagem Unificada, do inglês \textit{Unified Modeling Language}
\end{siglas}

% Índice de capítulos.
% \tableofcontents*


%%% Início da parte de conteúdo do documento. %%%

% Marca o início dos elementos textuais.
\clearpage
\textual

% Inclusão dos capítulos.
% (*) Para facilitar a organização, os capítulos foram divididos em arquivo separados e colocados dentro da.
% pasta capitulos/. Caso o aluno prefira trabalhar com um só arquivo, basta substituir os comandos \include 
% pelos conteúdos dos arquivos que estão sendo incluídos, excluindo a pasta capitulos/ em seguida.
% ==============================================================================
% Anteprojeto de PG - Nome do Aluno
% Capítulo 1 - Introdução
% ==============================================================================

\section{Introdução}
\label{chap-intro}

Parágrafo introdutório do capítulo. Exemplos de referência: \citeonline{guarino-et-al:hobook09} (in-line) ou \cite{guarino-et-al:hobook09}.



%%% Início de seção. %%%
\subsection{Motivação e Justificativa}
\label{sec-motivacao}

Motivação e justificativa.



%%% Início de seção. %%%
\subsection{Objetivos}
\label{sec-objetivos}

Descreva o objetivo geral do trabalho e em seguida apresente uma lista de subobjetivos. Objetivos devem ser escritos como algo a ser alcançado e não como tarefas (algo a ser feito).



%%% Início de seção. %%%
\subsection{Descrição}
\label{sec-descricao}

Descrição.



%%% Início de seção. %%%
\subsection{Metodologia}
\label{sec-metodologia}

As seguinte atividades são realizadas no contexto deste trabalho:

\begin{itemize}
	\item ...
\end{itemize}



%%% Início de seção. %%%
\subsection{Resultados Esperados}
\label{sec-resultados-esperados}

Resultados esperados.



%%% Início de seção. %%%
\subsection{Cronograma}
\label{sec-cronograma}

As tabelas~\ref{tab:cronograma-1} e~\ref{tab:cronograma-2} apresentam o cronograma deste trabalho, referindo-se às atividades abaixo listadas por número.

\hl{No caso de PGs iniciados no meio do ano, troque a primeira linha de uma tabela com a outra para começar em julho.}

\begin{enumerate}
	\item ...
\end{enumerate}

\begin{table}[htb]
	\centering
	\caption{Cronograma de Atividades do primeiro semestre.}
	\label{tab:cronograma-1}
	\resizebox{\columnwidth}{!}{
		\begin{tabular}{c|c|c|c|c|c|c}
			Atividade & Janeiro/99 & Fevereiro/99  & Março/99  & Abril/99 & Maio/99 & Junho/99\\ \hline
			1&     X      &  	  X   	       &  	X	  	   & 		X	&     X   &      X    \\ \hline
			2&            &  	   	     	   &  	  X  	   & 		X	&         &           \\ \hline
			3&            &  		           &  	  X 	   & 		X   &   X     &     X      \\ \hline
			4&            &  			       &  			   & 	        &         &     X      \\ \hline
			5&            &  			       &  			   & 	        &    X    &   X       \\ \hline
			6&            &  			       &  			   & 	        &         &           \\ \hline
			7&            &  			       &  			   & 	        &         &           \\ \hline
		\end{tabular}
	}
\end{table}

\begin{table}[htb]
	\centering
	\caption{Cronograma de Atividades do segundo semestre.}
	\label{tab:cronograma-2}
	\resizebox{\columnwidth}{!}{
		\begin{tabular}{c|c|c|c|c|c|c}
			Atividade & Julho/99 & Agosto/99  & Setembro/99  & Outubro/99 & Novembro/99 & Dezembro/99\\ \hline
			1&            &  	      	       &  	 	  	   & 		 	&         &           \\ \hline
			2&            &  	   	     	   &  	     	   & 		 	&         &           \\ \hline
			3&            &  		           &  	    	   & 		    &         &            \\ \hline
			4&    X       &  		X	       &  			   & 	        &         &            \\ \hline
			5&    X       &  		X          &  	X   	   & 	 X      &         &           \\ \hline
			6&            &  			       &  			   & 	 X      &    X    &           \\ \hline
			7&            &  			       &  			   & 	 X      &    X    &    X      \\ \hline
		\end{tabular}
	}
\end{table}



%%% Início de seção. %%%
\subsection{Recursos Necessários}
\label{sec-recursos-necessarios}

Ao longo do trabalho, foram (ou serão) utilizados:

\begin{itemize}
	\item ...
\end{itemize}


% ==============================================================================
% Anteprojeto de PG - Nome do Aluno
% Capítulo 2 - Fundamentação Teórica
% ==============================================================================

\section{Fundamentação Teórica}
\label{sec-fundamentacao}

Parágrafo introdutório do capítulo.



%%% Início de seção. %%%
\subsection{Engenharia de Software}
\label{sec-fundamentacao-engsoft}

Exemplo de seção de fundamentação teórica. Substituir pelos assuntos tratados no seu PG.
% ==============================================================================
% Anteprojeto de PG - Nome do Aluno
% Capítulo 3 - Proposta de Sumário
% ==============================================================================

\section{Proposta de Sumário}
\label{chap-sumario}

A monografia entregue como produto final deste trabalho será dividida da seguinte forma:

\begin{itemize}
	\item \textbf{Capítulo I: Introdução}
	
	Introdução sobre o conteúdo do projeto, seus objetivos e metodologia.
	
	\item \textbf{Capítulo II: Referencial Teórico}
	
	Resumo da fundamentação teórica do trabalho.
	
	\item \textbf{Capítulo ...}
	
	E assim por diante...	
\end{itemize}

% Finaliza a parte no bookmark do PDF para que se inicie o bookmark na raiz e adiciona espaço de parte no sumário.
\phantompart

% Marca o fim dos elementos textuais.
\postextual

% Referências bibliográficas
\clearpage
\bibliography{bibliografia}

% Fim do documento.
\end{document}
