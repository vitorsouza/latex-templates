% ==============================================================================
% Anteprojeto de PG - Nome do Aluno
% Capítulo 1 - Introdução
% ==============================================================================

\section{Introdução}
\label{chap-intro}

Parágrafo introdutório do capítulo. Exemplos de referência: \citeonline{guarino-et-al:hobook09} (in-line) ou \cite{guarino-et-al:hobook09}.



%%% Início de seção. %%%
\subsection{Motivação e Justificativa}
\label{sec-motivacao}

Motivação e justificativa.



%%% Início de seção. %%%
\subsection{Objetivos}
\label{sec-objetivos}

Descreva o objetivo geral do trabalho e em seguida apresente uma lista de subobjetivos. Objetivos devem ser escritos como algo a ser alcançado e não como tarefas (algo a ser feito).



%%% Início de seção. %%%
\subsection{Descrição}
\label{sec-descricao}

Descrição.



%%% Início de seção. %%%
\subsection{Metodologia}
\label{sec-metodologia}

As seguinte atividades são realizadas no contexto deste trabalho:

\begin{itemize}
	\item ...
\end{itemize}



%%% Início de seção. %%%
\subsection{Resultados Esperados}
\label{sec-resultados-esperados}

Resultados esperados.



%%% Início de seção. %%%
\subsection{Cronograma}
\label{sec-cronograma}

As tabelas~\ref{tab:cronograma-1} e~\ref{tab:cronograma-2} apresentam o cronograma deste trabalho, referindo-se às atividades abaixo listadas por número.

\hl{No caso de PGs iniciados no meio do ano, troque a primeira linha de uma tabela com a outra para começar em julho.}

\begin{enumerate}
	\item ...
\end{enumerate}

\begin{table}[htb]
	\centering
	\caption{Cronograma de Atividades do primeiro semestre.}
	\label{tab:cronograma-1}
	\resizebox{\columnwidth}{!}{
		\begin{tabular}{c|c|c|c|c|c|c}
			Atividade & Janeiro/99 & Fevereiro/99  & Março/99  & Abril/99 & Maio/99 & Junho/99\\ \hline
			1&     X      &  	  X   	       &  	X	  	   & 		X	&     X   &      X    \\ \hline
			2&            &  	   	     	   &  	  X  	   & 		X	&         &           \\ \hline
			3&            &  		           &  	  X 	   & 		X   &   X     &     X      \\ \hline
			4&            &  			       &  			   & 	        &         &     X      \\ \hline
			5&            &  			       &  			   & 	        &    X    &   X       \\ \hline
			6&            &  			       &  			   & 	        &         &           \\ \hline
			7&            &  			       &  			   & 	        &         &           \\ \hline
		\end{tabular}
	}
\end{table}

\begin{table}[htb]
	\centering
	\caption{Cronograma de Atividades do segundo semestre.}
	\label{tab:cronograma-2}
	\resizebox{\columnwidth}{!}{
		\begin{tabular}{c|c|c|c|c|c|c}
			Atividade & Julho/99 & Agosto/99  & Setembro/99  & Outubro/99 & Novembro/99 & Dezembro/99\\ \hline
			1&            &  	      	       &  	 	  	   & 		 	&         &           \\ \hline
			2&            &  	   	     	   &  	     	   & 		 	&         &           \\ \hline
			3&            &  		           &  	    	   & 		    &         &            \\ \hline
			4&    X       &  		X	       &  			   & 	        &         &            \\ \hline
			5&    X       &  		X          &  	X   	   & 	 X      &         &           \\ \hline
			6&            &  			       &  			   & 	 X      &    X    &           \\ \hline
			7&            &  			       &  			   & 	 X      &    X    &    X      \\ \hline
		\end{tabular}
	}
\end{table}



%%% Início de seção. %%%
\subsection{Recursos Necessários}
\label{sec-recursos-necessarios}

Ao longo do trabalho, foram (ou serão) utilizados:

\begin{itemize}
	\item ...
\end{itemize}

