% Usa o estilo abntex2, configurando detalhes de formatação e hifenização.
\documentclass[
	12pt,				
	oneside,		
	a4paper,			
	english,			% Idioma adicional para hifenização.
	french,				% Idioma adicional para hifenização.
	spanish,			% Idioma adicional para hifenização.
	brazil				% O último idioma é o principal do documento.
	]{abntex2}



%%% Importação de pacotes. %%%

% Pacotes não documentados:
\usepackage{etex}
%\reserveinserts{28}
\usepackage{colortbl}
\usepackage{framed}
\usepackage{multirow}

\usepackage{longtable}
\usepackage{pdflscape}

% Posicionamento de elementos.
\usepackage{float}

% Usa a fonte Latin Modern.
\usepackage{lmodern}

% Seleção de códigos de fonte.
\usepackage[T1]{fontenc}

% Codificação do documento em Unicode.
\usepackage[utf8]{inputenc}

% Usado pela ficha catalográfica.
\usepackage{lastpage}

% Indenta o primeiro parágrafo de cada seção.
\usepackage{indentfirst}

% Controle das cores.
\usepackage[usenames,dvipsnames]{xcolor}

% Inclusão de gráficos.
\usepackage{graphicx}

% Inclusão de páginas em PDF diretamente no documento (para uso nos apêndices).
\usepackage{pdfpages}

% Para melhorias de justificação.
\usepackage{microtype}

% Citações padrão ABNT.
\usepackage[brazilian,hyperpageref]{backref}
\usepackage[alf]{abntex2cite}	
\renewcommand{\backrefpagesname}{Citado na(s) página(s):~}		% Usado sem a opção hyperpageref de backref.
\renewcommand{\backref}{}										% Texto padrão antes do número das páginas.
\renewcommand*{\backrefalt}[4]{									% Define os textos da citação.
	\ifcase #1
		Nenhuma citação no texto.
	\or
		Citado na página #2.
	\else
		Citado #1 vezes nas páginas #2.
	\fi}

% Pacotes não incluídos no template abntex2. 
% Podem ser comentados caso não queira utilizá-los.

% Inclusão de símbolos não padrão.
\usepackage{amssymb}
\usepackage{eurosym}

% Para utilizar \eqref para referenciar equações.
\usepackage{amsmath}

% Permite mostrar figuras muito largas em modo paisagem com \begin{sidewaysfigure} ao invés de \begin{figure}.
\usepackage{rotating}

% Permite customizar listas enumeradas/com marcadores.
\usepackage{enumitem}

% Permite inserir hiperlinks com \url{}.
\usepackage{bigfoot}
\usepackage{hyperref}

% Color control.
\usepackage[usenames,dvipsnames]{xcolor}

% Permite usar o comando \hl{} para evidenciar texto com fundo amarelo. Útil para chamar atenção a itens a fazer.
% O comando \phl é definido para que o professor evidencie texto com uma cor diferente para adicionar notas.
\usepackage{soulutf8}
\newcommand{\phl}[2][Peach]{{\sethlcolor{#1} \hl{#2}}}


% Permite usar o comando \hl{} para evidenciar texto com fundo amarelo. Útil para chamar atenção a itens a fazer.
\usepackage{soul}


% Permite inserir espaço em branco condicional (incluído no texto final só se necessário) em macros.
\usepackage{xspace}

% Permite usar tabelas que ocupam mais de uma pág
\usepackage{longtable}

% Permite incluir listagens de código com o comando \lstinputlisting{}.
\usepackage{listings}
\usepackage{caption}
\DeclareCaptionFont{white}{\color{white}}
\DeclareCaptionFormat{listing}{\colorbox{gray}{\parbox{\textwidth}{#1#2#3}}}
\captionsetup[lstlisting]{format=listing,labelfont=white,textfont=white}
\renewcommand{\lstlistingname}{Listagem}
\definecolor{mygray}{rgb}{0.5,0.5,0.5}
\lstset{
	basicstyle=\scriptsize,
	breaklines=true,
	numbers=left,
	numbersep=5pt,
	numberstyle=\tiny\color{mygray}, 
	rulecolor=\color{black},
	showstringspaces=false,
	tabsize=2,
    inputencoding=utf8,
    extendedchars=true,
    literate=%
    {é}{{\'{e}}}1
    {è}{{\`{e}}}1
    {ê}{{\^{e}}}1
    {ë}{{\¨{e}}}1
    {É}{{\'{E}}}1
    {Ê}{{\^{E}}}1
    {û}{{\^{u}}}1
    {ù}{{\`{u}}}1
    {â}{{\^{a}}}1
    {à}{{\`{a}}}1
    {á}{{\'{a}}}1
    {ã}{{\~{a}}}1
    {Á}{{\'{A}}}1
    {Â}{{\^{A}}}1
    {Ã}{{\~{A}}}1
    {ç}{{\c{c}}}1
    {Ç}{{\c{C}}}1
    {õ}{{\~{o}}}1
    {ó}{{\'{o}}}1
    {ô}{{\^{o}}}1
    {Õ}{{\~{O}}}1
    {Ó}{{\'{O}}}1
    {Ô}{{\^{O}}}1
    {î}{{\^{i}}}1
    {Î}{{\^{I}}}1
    {í}{{\'{i}}}1
    {Í}{{\~{Í}}}1
}

% Colorinlistoftodos package: to insert colored comments so authors can collaborate on the content.
\usepackage[colorinlistoftodos, textwidth=20mm, textsize=footnotesize]{todonotes}
\newcommand{\luiz}[1]{\todo[author=\textbf{Luiz Felipe},color=green!30,caption={},inline]{#1}}
\newcommand{\vitor}[1]{\todo[author=\textbf{Vítor},color=red!30,caption={},inline]{#1}}




%%% Definição de variáveis. %%%

\renewcommand{\imprimircapa}{%
	\begin{capa}%
		\center
		
		{\ABNTEXchapterfont\large\subtitulo{}}
		\vfill
		\begin{center}
			\ABNTEXchapterfont\bfseries\LARGE\imprimirtitulo
		\end{center}
		
		\vfill
		Registro de Alterações:
		\begin{table}[h]
			\centering
			\vspace{0.5cm}
			\begin{tabular}{|c|c|c|p{4.5cm}|}  \hline  \rowcolor[rgb]{0.8,0.8,0.8}
 				
 				Versão & Responsável & Data  & Alterações \\	\hline  
 				                            
				1.0 & Luiz Felipe F. Mai    & 13/03/2017  & Versão inicial da documentação \\ \hline 
				1.1 & Vítor E. Silva Souza & 17/03/2017  & \textit{Revisado até Seção~\ref{sec-requisitos}.} \\ \hline 
				1.2 & Luiz Felipe F. Mai & 22/03/2017  & \textit{Correções realizadas na versão 1.1} \\ \hline 
				1.3 & Vítor E. Silva Souza & 17/04/2017 & \textit{Revisado até Seção~\ref{sec-caso-de-uso}.} \\ \hline
				1.4 & Luiz Felipe F. Mai & 25/04/2017 & \textit{Correções realizadas na versão 1.3} \\ \hline
				1.5 & Vítor E. Silva Souza & 15/05/2017 & \textit{Revisado até o final.} \\ \hline
				1.6 & Luiz Felipe F. Mai & 30/05/2017 & \textit{Correções realizadas na versão 1.4} \\ \hline
				1.7 & Vítor E. Silva Souza & 14/06/2017 & \textit{Nova revisão completa.} \\ \hline
				1.8 & Luiz Felipe F. Mai & 27/06/2017 & \textit{Redivisão dos subsistemas e correções realizadas na versão 1.7} \\ \hline
				2.0 & Luiz Felipe F. Mai & 16/12/2017 & \textit{Correções após apresentação} \\ \hline

			 
			\end{tabular}
		\end{table}
		
		\vfill
		\large\imprimirlocal
		\linebreak
		\large\imprimirdata
		\vspace*{1cm}
	\end{capa}
}

\newcommand{\versao}{2.0}
\newcommand{\subtitulo}{Documento de Especificação de Requisitos}

\titulo{Khoeus - Plataforma de Aprendizado}
\autor{Luiz Felipe Ferreira Mai}
\local{Vitória, ES}
\data{2017}

\instituicao{
  Universidade Federal do Espírito Santo -- UFES
  \par
  Centro Tecnológico
  \par
  Departamento de Informática}
\tipotrabalho{Monografia (PG)}







% Macros específicas do trabalho.
% (*) Inclua aqui termos que são utilizados muitas vezes e que demandam formatação especial.
% Os exemplos abaixo incluem i* (substituindo o asterisco por uma estrela) e Java com TM em superscript.
% Use sempre \xspace para que o LaTeX inclua espaço em branco após a macro somente quando necessário.
\newcommand{\istar}{\textit{i}$^\star$\xspace}
\newcommand{\java}{Java\texttrademark\xspace}
\newcommand{\latex}{\LaTeX\xspace}




%%% Configurações finais de aparência. %%%

% Altera o aspecto da cor azul.
\definecolor{blue}{RGB}{41,5,195}

% Informações do PDF.
\makeatletter
\hypersetup{
	pdftitle={\@title}, 
	pdfauthor={\@author},
	pdfsubject={\imprimirpreambulo},
	pdfcreator={LaTeX with abnTeX2},
	pdfkeywords={abnt}{latex}{abntex}{abntex2}{trabalho acadêmico}, 
	colorlinks=true,				% Colore os links (ao invés de usar caixas).
	linkcolor=blue,					% Cor dos links.
	citecolor=blue,					% Cor dos links na bibliografia.
	filecolor=magenta,				% Cor dos links de arquivo.
	urlcolor=blue,					% Cor das URLs.
	bookmarksdepth=4
}
\makeatother

% Espaçamentos entre linhas e parágrafos.
\setlength{\parindent}{1.3cm}
\setlength{\parskip}{0.2cm}



%%% Páginas iniciais do documento: capa, folha de rosto, ficha, resumo, tabelas, etc. %%%

% Compila o índice.
\makeindex















% Inicia o documento.
\begin{document}

% Retira espaço extra obsoleto entre as frases.
\frenchspacing


\begin{figure}[h]
  \centering
  \includegraphics[scale=0.055]{figuras/brasao.jpg}
  \label{ppts3}
  \end{figure} 


% Capa do trabalho.




\imprimircapa





%%% Início da parte de conteúdo do documento. %%%
% Marca o início dos elementos textuais.
\textual
% Inclusão dos capítulos.
% (*) Para facilitar a organização, os capítulos foram divididos em arquivo separados e colocados dentro da.
% pasta capitulos/. Caso o aluno prefira trabalhar com um só arquivo, basta substituir os comandos \include 
% pelos conteúdos dos arquivos que estão sendo incluídos, excluindo a pasta capitulos/ em seguida.

\chapter{Introdução}
\label{sec-intro}


Este documento apresenta os requisitos de usuário e a análise dos requisitos do sistema Khoeus. A atividade de análise de requisitos foi conduzida aplicando-se técnicas de modelagem de casos de uso, modelagem de classes e levando em consideração as boas práticas de programação relacionadas ao desenvolvimento de aplicações web. Os modelos apresentados foram elaborados usando a UML.

Na Seção~\ref{sec-proposito} deste documento descreve-se de forma geral o sistema, apresentando superficialmente suas principais características. Já na Seção~\ref{sec-minimundo} estão apresentadas as funcionalidades do Khoeus, bem como suas dependências. A Seção~\ref{sec-requisitos} lista os requisitos de usuário do sistema (funcionais, não funcionais e suas regras de negócio). Além disso, a Seção~\ref{sec-subsistemas} exibe os subsistemas considerados para este projeto enquanto a Seção~\ref{sec-caso-de-uso} apresenta o modelo de casos de uso, incluindo descrições de atores, os diagramas de casos de uso e suas respectivas descrições. Na Seção~\ref{sec-modelo-estrutural} estão apresentados os modelos conceituais estruturais do sistema na forma de diagramas de classes. Por fim, a Seção~\ref{sec-dicionario} apresenta o  dicionário do projeto, contendo as definições das classes identificadas.


\chapter{Descrição do Propósito do Sistema}
\label{sec-proposito}

A possibilidade de um ambiente em que professores possam disponibilizar materiais de aula e requisitar tarefas dos alunos é bastante atraente para ambas as partes e, por isso, a utilização de plataformas de aprendizado tem se tornado cada vez maior em ambientes de ensino como escolas, faculdades ou cursos externos. No entanto, numa tentativa de atender a todos, as plataformas mais utilizadas atualmente acabam se tornando confusas e fornecendo ferramentas que, muitas vezes, não são utilizadas.

Não obstante a vasta gama de funcionalidades fornecidas por essas plataformas, nenhuma delas lida com códigos de programação de forma eficiente: em sua maioria, códigos são gerenciados como arquivos ou textos comuns, impedindo que seja possível explorar ao máximo o aprendizado do aluno na área de programação.

Dessa forma, o Khoeus vem com o intuito de facilitar o processo de aprendizado de alunos e, ao mesmo, tentar corrigir as falhas e os excessos das plataformas já existentes de forma a ser o mais simples e prático possível, agregando a possibilidade de trabalhar com códigos de programação.


\chapter{Descrição do Minimundo}
\label{sec-minimundo}

Em um ambiente de aprendizagem como uma escola ou universidade, espera-se que os alunos tenham acesso fácil e direto às informações relativas às aulas, como materiais de aula, notas das atividades realizadas e lista de presença. Para isso, espera-se centralizar essas funcionalidade no formato de turmas, de forma que cada uma delas possua seus alunos, professores, atividades, dentre outros elementos envolvidos no sistema.

\section{Organização de turmas}
\label{sec-minimundo:organizacao-turmas}

A uma turma, estarão vinculados uma série de usuários que poderão exercer papel (\textit{role}) de professor ou de aluno. Para participar de uma turma, um usuário deverá visitá-la e, em seu primeiro acesso, inserir a chave de acesso (que foi definida no momento em que a turma foi criada). Caso a chave esteja correta, o usuário é associado àquela turma e pode acessar seu conteúdo. Além disso, cada turma contém um \textit{board}, que corresponde a todos os itens que serão exibidos aos alunos, como tarefas, arquivos, URLs, dentre outros (descritos na seção \ref{sec-minimundo:itens-board}). Para melhor organizá-lo, um \textit{board} é subdividido em seções definidas pelo professor e todos os itens deverão estar associados a uma seção. Um \textit{board} contém obrigatoriamente:

\begin{itemize}
	\item Uma \textbf{lista de notícias} que serão criadas exclusivamente pelos professores da turma. As notícias não possuem réplica.
	\item Um \textbf{fórum de discussão} com tópicos que podem ser tanto criados quanto respondidos por professores ou alunos da turma.
	\item Uma \textbf{lista de \textit{LiveQuestions}} com perguntas que os alunos podem realizar durante as aulas
	\item A \textbf{lista de presença} dos alunos da turma
	\item O \textbf{livro de notas} contendo a avaliação de cada aluno em cada uma das atividades realizadas.
\end{itemize}

Além disso, o \textit{board} também poderá conter:

\begin{itemize}
	\item \textbf{Arquivos:} permite realizar o download de um arquivo que foi enviado pelo professor;
	\item \textbf{Links:} permite acessar páginas externas;
	\item \textbf{Questionários:} permite responder a perguntas objetivas sem que existam opções corretas, simulando uma enquete.
	\item \textbf{Provas:} permite responder a perguntas objetivas e discursivas que serão avaliadas.
	\item \textbf{Tarefas:}  permite enviar arquivos, textos ou códigos de programação de acordo com a tarefa passada pelo professor.
	\item \textbf{Atividades Externas:}  permite avaliar uma atividade que tenha sido realizada fora da plataforma, como uma atividade feita em sala de aula, por exemplo.
\end{itemize}


\section{Controle de usuários}
\label{sec-minimundo:controle-usuarios}

Todas as funcionalidades do sistema dependem do usuário estar logado em sua conta. Para isso, o Khoeus conta com uma página de login, uma de cadastro e uma de recuperação de senha, permitindo que os usuários possam acessar o sistema. Na página de cadastro, o usuário deverá informar seus dados pessoais (como nome, endereço e telefone), seu e-mail (que será utilizado para realizar o login) e uma senha. Feito isso, um e-mail de confirmação será enviado para o e-mail cadastrado a fim de que o usuário confirme seu cadastro. Na página de login, basta inserir o email e senha para que seja verificada sua autenticidade. Caso estejam corretos, o usuário é redirecionado para a lista de turmas. Caso contrário, deverá inserir novamente seus dados. Na página de recuperação de senha, basta que o usuário digite o e-mail cadastrado em sua conta para que receba um link em sua caixa de entrada permitindo a mudança de senha. 

Um usuário do sistema pode ter permissão de \textbf{Administrador} ou de \textbf{Membro}. Enquanto os membros podem apenas alterar o próprio perfil, ingressar em uma turma ou acessar uma turma já ingressada anteriormente, administradores têm acesso completo ao sistema e podem alterar as configurações gerais do sistema, das turmas e editar o perfil de todos os usuários. Para o sistema, espera-se que seja possível configurar o fuso horário utilizado pelo sistema, definir as informações de SMTP para disparo de e-mails, colocar o sistema em modo de manutenção (acessível apenas para administradores). Para as turmas, espera-se poder definir seu título, sua senha de acesso, definir se a turma possui livro de notas e lista de presença além de ser possível definir as categorias de nota que serão utilizadas para a avaliação dos alunos (mais detalhes na Seção~{\ref{sec-minimundo:avaliacao}}). Já para os alunos, espera-se ser possível alterar seu perfil (nome, senha, endereço, etc.). Além disso, os administradores têm acesso a um registro com logs de todo o sistema, incluindo todas as ações realizadas pelos usuários, como cadastro, login, criação de itens de board e até mesmo submissão de novas tarefas.

Além das permissões vinculadas à conta, os usuários também possuem um papel (\textit{role}) em cada turma da qual fazem parte, podendo este ser de \textbf{Professor} ou \textbf{Aluno}. Assim, um usuário com a permissão de membro pode assumir a posição de professor ou de aluno dentro de uma turma. Em uma turma, um aluno poderá consultar os itens do \textit{board}, como enviar tarefas e baixar arquivos, consultar as próprias presenças na lista de chamada, consultar suas notas no livro de notas, verificar o calendário de aulas planejadas, criar novas \textit{Live Questions} e enviar mensagens para alunos e professores. Paralelamente a isso, um professor pode inserir novos itens no \textit{board}, lançar presenças e notas dos alunos cadastrados na turma, inserir uma nova aula no calendário, verificar as \textit{Live Questions} vigentes, enviar mensagem para alunos e professores, realizar download em lote dos envios de uma tarefa, prova ou questionário, visualizar o log de atividades dos alunos e definir as configurações da turma, já descritas anteriormente.

Vale ressaltar que um usuário pode ter sua \textit{role} alterada de acordo com a vontade de um administrador. Dessa forma, é possível que um Membro torne-se Administrador (ou vice-versa) e que um Aluno vire Professor (ou vice-versa).


\section{Itens do Board}
\label{sec-minimundo:itens-board}

Itens do \textit{board} estão sempre associados a uma seção com exceção do fórum de discussões, das notícias, das \textit{Live Questions}, da lista de presença e do livro de notas, que são fixados no topo do \textit{board}. Cada item do \textit{board} possui uma ação diferente e possui campos diferentes no momento da inclusão.

\subsection{Arquivo}
\label{sec-minimundo:itens-board:arquivo}
Ao acessar um arquivo, o aluno realiza o download do mesmo. Para criar um novo arquivo, o professor deverá informar seu nome, a descrição e deverá fazer o upload a partir dos arquivos de seu computador.

\subsection{Link}
\label{sec-minimundo:itens-board:link}
Ao acessar um link, o aluno é redirecionado para a URL correspondente. Para inserir um novo link no board, o professor deverá informar o título do link, sua descrição e a sua respectiva URL.

\subsection{Notícias}
\label{sec-minimundo:itens-board:noticias}
No topo do board, o aluno pode acessar a lista de notícias relacionada àquela turma e onde estão todas as notícias já criadas pelos professores da turma. Para inserir uma nova noticia, o professor deverá informar apenas o título e o conteúdo desejado.

\subsection{Fórum de Discussão}
\label{sec-minimundo:itens-board:forum}
No fórum de discussão, tanto alunos quanto professores podem criar novos tópicos e, da mesma forma, responder tópicos já criados anteriormente. Para criar um novo tópico, espera-se que o usuário insira um título e seu conteúdo. No caso da réplica de um tópico, basta seu conteúdo.

\subsection{\textit{Live Questions}}
\label{sec-minimundo:itens-board:live-questions}
Na lista de \textit{Live Questions}, alunos poderão criar novas dúvidas e professores poderão visualizá-las por até 24h após sua criação. A ideia é que as dúvidas sejam criadas e sanadas no momento da aula. Na sua criação, o aluno deve inserir sua dúvida e informar se deseja ser tratado como Anônimo ou não. 

\subsection{Questionário}
\label{sec-minimundo:itens-board:questionario}
Os questionários simulam, basicamente, uma enquete ou uma pesquisa de opinião: no momento da criação, o professor define uma série de perguntas (todas de múltipla escolha) e as respostas possíveis para cada uma delas, sendo possível marcar apenas uma das alternativas. Nesse item, não existe resposta correta e, por isso, os questionários devem ser utilizados para recolher a opinião dos alunos sobre algum assunto determinado. Ao acessar um questionário, os alunos visualizarão as perguntas que foram criadas pelo professor e deverão respondê-las, não sendo possível responder a um questionário mais de uma vez.  Questionários possuem uma data para iniciar e um tempo-limite para que possam ser respondidos. Após isso, é possível coletar os resultados do questionário. 

\subsection{Prova}
\label{sec-minimundo:itens-board:prova}
Uma prova, como o próprio nome já diz, é elegível para ser uma forma de avaliação do aluno. Dessa forma, no momento da criação, o professor deve associá-la a uma categoria de nota e informar um título e uma descrição. Provas possuem uma data para iniciar e uma data de término que também deverão ser informadas. Além disso, o professor deve inserir quais perguntas haverão na prova, o tipo da questão (discursiva ou objetiva) e qual a nota máxima para cada uma das questões (totalizando 100 pontos). No caso de perguntas objetivas, deve-se informar qual das alternativas é a correta. Após encerrado o prazo da prova, os professores poderão visualizar as respostas dos alunos para cada uma das questões e atribuir a elas uma nota variando de 0 à nota máxima para aquela questão, sendo possível também informar um feedback para cada questão.

\subsection{Tarefa}
\label{sec-minimundo:itens-board:tarefa}
Tarefas representam a principal forma de avaliar um aluno, uma vez que permitem o envio de textos, arquivos ou até mesmo códigos de programas. No momento da criação da tarefa, o professor deverá informar o tipo de tarefa, a que categoria de nota esta pertence, título, descrição, data de início e data limite de envio. Uma vez finalizada, o professor poderá avaliar a tarefa de cada aluno concedendo-o uma nota de 0 a 100 e sendo possível ainda dar um feedback em forma de texto ou comentar linhas do código enviado pelo aluno.

A forma de submissão da tarefa varia de acordo com o seu tipo: caso seja do tipo texto, o aluno deverá preencher um campo de texto com o que foi pedido na descrição da tarefa. Caso seja do tipo arquivo, o aluno deverá fazer o upload do arquivo que foi requisitado. No caso da tarefa ser do tipo código, ele deverá inserí-lo no campo de texto, podendo realizar a compilação em tempo real a fim de verificar a corretude de seu código. 

\subsection{Atividade Externa}
\label{sec-minimundo:itens-board:ativida-externa}
Atividades externas representam exercícios ou provas que foram feitas em sala e que, por isso, não são avaliados diretamente no Khoeus. Dessa forma, alunos não precisarão submeter quaisquer informações neste tipo de item (uma vez que esta foi feita presencialmente), mas o professor poderá lançar as notas de cada um dos alunos e incluir um pequeno feedback a respeito da atividade. 

Atividades externas também devem estar vinculadas a uma categoria de nota.

\section{Avaliação}
\label{sec-minimundo:avaliacao}
Dentro das configurações de uma turma, é possível definir a forma com que a média final é calculada e quais as condições de aprovação dos alunos. Dessa forma, professores podem criar categorias de nota e associar a cada uma delas um peso no cálculo da média final. É possível criar, por exemplo, as categorias de nota ``Prova'' e ``Trabalho'' e definir a média final como 60\% da nota de prova somado a 40\% da nota de trabalho.

Para que a média seja calculada corretamente, o professor deverá definir no momento da criação de uma \textbf{Tarefa}, \textbf{Prova} ou \textbf{Atividade Externa} a que categoria de nota ela pertence.



\chapter{Requisitos de Usuário}
\label{sec-requisitos}

\noindent Tomando por base o contexto do sistema acima e considerando como principais \textit{stakeholders} os professores e alunos do Introcomp e de outros cursos relacionados ao ensino de programação, foram identificados os seguintes requisitos de usuário e regras de negócio:

\newcounter{rfcount}
\renewcommand*\therfcount{RF-\arabic{rfcount}}
\newcommand*\RF{\refstepcounter{rfcount}\therfcount}
\setcounter{rfcount}{0}

\begin{longtable}{|c|p{10cm}|c|p{2cm}|}
	
	\caption{Requisitos Funcionais} \\ \hline \rowcolor[rgb]{0.8,0.8,0.8}
	
	ID &  Descrição  &  Prioridade   &  	Depende   \\ \hline	
	
	
	\RF\label{rf-configuracoes} &  O sistema deve permitir que administradores possam definir as configurações de sistema (fuso horário, definições SMTP e modo de manutenção). Essas configurações devem ser armazenadas de forma genérica, de forma que seja guardada o nome da configuração e seu respectivo valor.  & Alta & \\ \hline
	
	\RF\label{rf-criar-turma} & Administradores devem poder criar novas turmas informando seu título, sua senha de acesso e se possui livro de notas e lista de presença. & Alta &  \\ \hline 

	\RF\label{rf-gerenciamento-turma}  &  O sistema deve permitir que professores possam definir as configurações de turmas (título, senha de acesso, categorias de nota e existência de livro de notas e lista de presença). Deve-se também ser definida a forma com que a média final será calculada (baseada nas categorias de nota, informando o título e o peso de cada uma delas) e qual a condição para a aprovação dos alunos inscritos. & Alta & \ref{rf-criar-turma} \\ \hline 
	
	\RF\label{rf-inscricao} & O sistema deve permitir que alunos se inscrevam em uma turma por meio de uma senha associada a ela. & Alta & \ref{rf-gerenciamento-turma} \\
	\hline	
	
	\RF\label{rf-gerenciamento-usuario-turma}  & O sistema deve permitir que administradores possam gerenciar os usuários dentro das turmas, sendo possível alterar sua \textit{role} de aluno para professor ou vice-versa. & Alta & \ref{rf-inscricao} \\
	\hline 
	
	\RF\label{rf-adicionar-secao}  & Professores devem ser capazes de criar novas seções no board de uma turma. & Alta & \ref{rf-gerenciamento-usuario-turma} \\
	\hline 
	
	\RF\label{rf-adicionar-arquivo}  & Professores devem ser capazes de adicionar novos arquivos ao \textit{board} de sua turma e configurar corretamente seu título, descrição e realizar o \textit{upload} do arquivo em questão. & Alta & \ref{rf-adicionar-secao} \\
	\hline 
	 
	\RF\label{rf-adicionar-link}  & Professores devem ser capazes de adicionar novos links ao \textit{board} de sua turma e configurar corretamente seu título, descrição e informar a URL do link em questão. & Alta & \ref{rf-adicionar-secao} \\
	\hline  
	 
	\RF\label{rf-adicionar-questionario}  & Professores devem ser capazes de adicionar novos questionários ao \textit{board} de sua turma e configurar corretamente seu título, descrição, data de início e data de término. Além disso, o professor deve informar todas as perguntas e suas respectivas respostas. & Alta & \ref{rf-adicionar-secao} \\
	\hline  
	
	\RF\label{rf-adicionar-prova}  & Professores devem ser capazes de adicionar novas provas ao \textit{board} de sua turma e configurar corretamente seu título, descrição, data de início e data de término. Além disso, o professor deve informar todas as questões da prova, seu tipo (discursiva ou objetivo) e sua pontuação (a soma da pontuação de todas as questões deve totalizar 100 pontos). No caso de perguntas objetivas, o professor deve informar todas as alternativas e qual delas é a correta. & Alta & \ref{rf-adicionar-secao} \\
	\hline  
	
	\RF\label{rf-adicionar-tarefa}  & Professores devem ser capazes de adicionar novas tarefas ao \textit{board} de sua turma e configurar corretamente seu título, descrição, data de início, data de término e tipo (texto, arquivo ou código). Caso seja do tipo Arquivo, o professor também deverá informar o número máximo de arquivos que podem ser enviados. & Alta & \ref{rf-adicionar-secao} \\
	\hline  
	
	\RF\label{rf-adicionar-atividadeexterna}  & Professores devem ser capazes de adicionar novas atividades externas ao \textit{board} de sua turma e configurar corretamente seu título e descrição. & Alta & \ref{rf-adicionar-secao} \\
	\hline  
	
	\RF\label{rf-adicionar-noticia}  & Professores devem ser capazes de publicar notícias no \textit{board} de uma turma. & 	Alta &  \ref{rf-adicionar-secao} \\
	\hline 
	
	\RF\label{rf-gerenciamento-forum}  & Professores e alunos devem ser capazes de visualizar, publicar e responder tópicos no \textit{board} de uma turma. & 	Alta &  \ref{rf-adicionar-secao} \\
	\hline
	
	\RF\label{rf-visualizar-livequestions}  & Professores devem ser capazes de visualizar \textit{live questions} no \textit{board} de uma turma. & 	Alta &  \ref{rf-adicionar-livequestion} \\ \hline
	
	\RF\label{rf-deletar-livequestions}  & As \textit{Live Questions} deverão ser deletadas automaticamente decorrido um dia de sua criação. & 	Alta &  \ref{rf-adicionar-livequestion} \\
	\hline
	
	\RF\label{rf-gerenciamento-calendario}  & Professores devem ser capazes de gerenciar as aulas e os eventos de uma turma por meio do calendário, inserindo, removendo e alterando-os. & 	Alta &  \ref{rf-gerenciamento-usuario-turma} \\ \hline

	\RF\label{rf-acessar-arquivo}  &  Alunos devem ser capazes de fazer download dos arquivos de sua turma. & Alta & \ref{rf-adicionar-arquivo} \\
	\hline

	\RF\label{rf-acessar-link}  &  Alunos devem ser capazes de acessar os links de sua turma. & Alta & \ref{rf-adicionar-link} \\
	\hline

	\RF\label{rf-acessar-questionario}  &  Alunos devem ser capazes de  responder aos questionários de sua turma. & Alta & \ref{rf-adicionar-questionario} \\
	\hline

	\RF\label{rf-acessar-prova}  &  Alunos devem ser capazes de resolver as provas de sua turma. & Alta & \ref{rf-adicionar-prova} \\
	\hline

	\RF\label{rf-acessar-tarefa}  &  Alunos devem ser capazes de submeter as tarefas de sua turma. No caso de tarefas do tipo Código, os alunos devem ter acesso a um compilador acoplado ao sistema para que possam testar seu código antes de submetê-lo. & Alta & \ref{rf-adicionar-tarefa} \\
	\hline

	\RF\label{rf-acessar-calendario}  &  Alunos devem ser capazes de consultar as aulas e eventos no calendário de sua turma. & Alta & \ref{rf-gerenciamento-calendario} \\
	\hline
	
	\RF\label{rf-resultado-questionario}  &  Após encerrado um questionário, professores devem poder consultar seu resultado. & Alta & \ref{rf-acessar-questionario} \\
	\hline
	
	\RF\label{rf-visualizar-noticia}  & Alunos devem ser capazes de visualizar as notícias no \textit{board} de sua turma. & 	Alta &  \ref{rf-adicionar-noticia} \\
	\hline 
	
	\RF\label{rf-adicionar-livequestion}  & Alunos devem ser capazes de criar novas \textit{Live Questions} no \textit{board} de sua turma. & 	Alta &  \ref{rf-adicionar-secao} \\
	\hline 
	  
	\RF\label{rf-download-lote}  & O sistema deve permitir que professores realizem download em lote das submissões de uma prova ou tarefa. & 	Alta &  \ref{rf-acessar-prova} e \ref{rf-acessar-tarefa} \\
	\hline
		
	\RF\label{rf-avaliar-tarefa} & Professores devem ser capazes de avaliar provas, tarefas e atividades externas, atribuindo-lhes uma nota e, opcionalmente, um feedback. Essas avaliações poderão ser feitas individualmente (aluno por aluno) ou em lote (via planilha externa).  & Alta & \ref{rf-acessar-tarefa} \\ \hline 
	
	\RF\label{rf-consultar-nota-prova} & Deve ser possível que alunos consultem sua nota em uma prova (assim como a pontuação em cada questão) a qualquer momento após sua data limite.  & Alta & \ref{rf-avaliar-tarefa} \\ \hline 
	
	\RF\label{rf-consultar-nota-tarefa} & Deve ser possível que alunos consultem sua nota e seu respectivo feedback em uma tarefa a qualquer momento após sua data limite.  & Alta & \ref{rf-avaliar-tarefa} \\ \hline 
	
	\RF\label{rf-consultar-nota-atividadeexterna} & Deve ser possível que alunos consultem sua nota e seu respectivo feedback em uma atividade externa a qualquer momento após sua data limite. & Alta & \ref{rf-avaliar-tarefa} \\ \hline 
	
	\RF\label{rf-consultar-notas} & Deve ser possível que alunos consultem suas notas por meio do livro de notas, sendo possível ver todas as suas notas ao longo do curso e sua média parcial até o momento. & Alta & \ref{rf-avaliar-tarefa} \\ \hline 
	
	\RF\label{rf-lancar-presenca}  & Professores devem ser capazes de lançar as presenças de seus alunos em cada aula no sistema. & 	Alta &  \ref{rf-gerenciamento-calendario} \\
	\hline 
	
	\RF\label{rf-consultar-presenca}  & Deve ser possível que alunos consultem suas presenças em aula a qualquer momento. & 	Alta &  \ref{rf-lancar-presenca} \\
	\hline 

	\RF\label{rf-enviar-mensagem}  & O sistema deve permitir que membros de uma turma enviem mensagens entre si. & 	Média &  \ref{rf-gerenciamento-usuario-turma} \\ \hline 
				
	\RF\label{rf-gerar-log-prof}  & O sistema deve permitir que professores de uma turma gerem logs com os acessos, visualizações e submissões de seus alunos. & Baixa &  \ref{rf-gerenciamento-usuario-turma} \\ \hline 
					
	\RF\label{rf-gerar-log-admin}  & O sistema deve permitir que administradores gerem relatórios com os acessos, visualizações e interações dos professores e alunos. & 	Baixa &  \ref{rf-gerenciamento-usuario-turma} \\ \hline 
\end{longtable}

\newcounter{rnfcount}
\renewcommand*\thernfcount{RNF-\arabic{rnfcount}}
\newcommand*\RNF{\refstepcounter{rnfcount}\thernfcount}
\setcounter{rnfcount}{0}


\begin{longtable}{|c|p{7.5cm}|p{2.5cm}|c|c|}
	
	\caption{Requisitos Não Funcionais}\\ \hline \rowcolor[rgb]{0.8,0.8,0.8}
	
	ID &	Descrição  & Categoria  & Escopo  & Prioridade   \\ \hline		
	
	\RNF\label{rnf-aplicacao-web}  & A plataforma deve estar disponível como uma aplicação Web, acessível a partir dos principais navegadores disponíveis no mercado.  & Portabilidade  & Sistema &	Alta   \\ \hline 	
	
	\RNF\label{rnf-seguranca}  & O sistema deve controlar o acesso às suas funcionalidades por meio de autenticação e autorização, utilizando o papel de cada usuário para definir a que funcionalidades ele terá acesso.  & Segurança  & Sistema &	Alta   \\ \hline 
	
	\RNF\label{rnf-criptografia}  & Todas as informações sensíveis (como senhas e tokens de acesso) devem ser criptografados.  & Confidencia\-bilidade  & Sistema &	Alta   \\ \hline 
	
	\RNF\label{rnf-apredizado-facil}  & A plataforma dever ser de aprendizado fácil, não sendo necessário nenhum treinamento especial para seu uso.  & Facilidade de Aprendizado  & Sistema &	Alta   \\ \hline 
	
	\RNF\label{rnf-consistencia}  & O sistema deve garantir a consistência de dados por meio da obrigatoriedade no momento da inserção de certos dados.  & Confiabilidade & Sistema & Alta \\ \hline  
		
	\RNF\label{rnf-facil-operacao}  & A plataforma deve ser de fácil operação, não sendo necessário uso contínuo para uma boa operação do sistema.  & Usabilidade  &	Sistema & Média  \\\hline 
	
	\RNF\label{rnf-facilitar-manutencao}  & O desenvolvimento do sistema deve facilitar manutenções futuras, utilizando padrões de design de software.  & Redigibilidade & Sistema &	Média  \\ 	\hline
	
\end{longtable}


\newcounter{rncount}
\renewcommand*\therncount{RN-\arabic{rncount}}
\newcommand*\RN{\refstepcounter{rncount}\therncount}
\setcounter{rncount}{0}

\begin{longtable}{|c|p{10cm}|c|p{2cm}|}
	
	\caption{Regras de Negócio} \\ \hline \rowcolor[rgb]{0.8,0.8,0.8}
	
	ID & 	Descrição  &  Prioridade    & Depende  \\ \hline	
	
	\RN\label{rn-roles}  & Um usuário não pode estar associado às \textit{roles} de aluno e professor simultaneamente em uma mesma turma. & Alta & \ref{rf-gerenciamento-usuario-turma}  \\ \hline 
	
	\RN\label{rn-item-secao}  & Um item de \textit{board} precisa estar, necessariamente, associado a uma seção do \textit{board}. & Alta &  \ref{rf-adicionar-arquivo}, \ref{rf-adicionar-link}, \ref{rf-adicionar-questionario}, \ref{rf-adicionar-prova}, \ref{rf-adicionar-tarefa} e \ref{rf-adicionar-atividadeexterna} \\ \hline
	
	\RN\label{rn-submeter} & Alunos não poderão submeter questionários, tarefas e provas após a data limite. & Alta & \ref{rf-acessar-questionario}, \ref{rf-acessar-tarefa} e \ref{rf-acessar-prova} \\ \hline	

	\RN\label{rn-avaliar} & Professores não poderão avaliar tarefas e provas antes da data limite. & Alta & \ref{rf-avaliar-tarefa} \\ \hline	
	
	\RN\label{rn-limite-arquivos} & Não é possível enviar um número de arquivos superior ao limite estipulado em uma tarefa. & Alta & \ref{rf-acessar-tarefa} \\ \hline
	
	\RN\label{rn-nota} & A nota de uma prova, tarefa ou atividade externa não poderá ser superior a 100. & Alta & \ref{rf-avaliar-tarefa} \\ \hline
	
	\RN\label{rn-peso} & A somatória do peso das categorias de notas deve ser igual a 100. & Alta & \ref{rf-gerenciamento-turma} \\ \hline

	\RN\label{rn-mensagem} & Usuários não podem trocar mensagens com alguém que não esteja em uma mesma turma que ele. & Alta & \ref{rf-enviar-mensagem} \\ \hline

\end{longtable}



\chapter{ Identificação de Subsistemas}
\label{sec-subsistemas}

Tal projeto foi dividido em 4 subsistemas a fim de facilitar o desenvolvimento de funcionalidades que são, de certa forma, isoladas, originando assim os subsistemas Auth, Log, Classroom e Board.
A Figura~\ref{figura-subsistema} mostra os subsistemas identificados no contexto do presente projeto e suas interdependências, enquanto a Tabela~\ref{tabela-subsistema} apresenta breve descrição de cada um deles.


\begin{figure}[h]
	\centering
	\includegraphics[width=\textwidth]{figuras/subsistemas.png}
	\caption{Diagrama de Pacotes e os Subsistemas Identificados.}
	\label{figura-subsistema}
\end{figure} 

\begin{table}[h]
	\centering	
	\vspace{0.5cm}
	\caption{ Subsistemas}
	\label{tabela-subsistema}
	\begin{tabular}{|p{3cm}|p{12cm}|}  \hline \rowcolor[rgb]{0.8,0.8,0.8}
		
		Subsistema & Descrição \\\hline 
		
		Classroom & Subsistema contendo as funcionalidades relacionadas diretamente às turmas.  \\\hline
		
		Board & Envolve todas as funcionalidades relativas ao gerenciamento dos itens do board. \\\hline
		
		Board Interactions & Corresponde a todas as interações que os usuários podem realizar com os itens do board, como submeter tarefas ou realizar provas. \\\hline
		
		Feedback & Permite a avaliação de atividades e a visualização das notas do aluno em uma turma. \\\hline
		
		Discussions & Subsistema contendo as funcionalidades relativas ao fórum de discussão, às notícias e às \textit{Live Questions}. \\\hline
		
		Logs & Este sistema consiste na visualização de logs gerados a partir das atividades dos usuários do sistema. \\\hline  
		
	\end{tabular}	
\end{table}


\chapter{Modelo de Casos de Uso}
\label{sec-caso-de-uso}

\newcounter{uccount}                                      \renewcommand*\theuccount{UC-\arabic{uccount}}
\newcommand*\UC{\refstepcounter{uccount}\theuccount}      \setcounter{uccount}{0}

O modelo de casos de uso corresponde a uma tentativa de descrever a relação das funcionalidades do sistema com cada um de seus atores. Os atores identificados no contexto deste projeto estão descritos na Tabela~\ref{tabela-atores}.

\begin{table}[H]
	\centering \vspace{0.5cm} \caption{ Atores}
	\begin{tabular}{|p{3cm}|p{12cm}|} \hline \rowcolor[rgb]{0.8,0.8,0.8}
		Ator & Descrição \\\hline                              
		Visitante & Qualquer pessoa que acesse o sistema e não tenha realizado o \textit{login}. \\\hline                              
		Administrador & Usuário que realizou o \textit{login} e pode gerenciar todo o sistema. \\\hline                              
		Membro & Usuário que realizou o \textit{login} no sistema mas que não possui permissões para configurá-lo. \\\hline	                              
		Aluno & Corresponde a um Membro vinculado a alguma turma com permissões de aluno, como enviar tarefas e consultas notas. \\\hline	                              
		Professor & Corresponde a um Membro vinculado a alguma turma com permissões de professor, como criar itens do \textit{board} e configurar as informações da turma. \\\hline			 
	\end{tabular}
	\label{tabela-atores}	
\end{table}

A seguir, são apresentados os diagramas de casos de uso e descrições associadas, organizados por subsistema.
	
\section{Subsistema Classroom}

A Figura~\ref{figura-caso-de-uso-classroom} apresenta o diagrama de casos de uso do subsistema Classroom.

\begin{figure}[h!]
	\centering
	\includegraphics[width=\textwidth]{figuras/casos-de-uso-classroom.png}
	\caption{Diagrama de Casos de Uso do subsistema Classroom.}
	\label{figura-caso-de-uso-classroom}
\end{figure}

Com exceção do \ref{uc-definir-configuracoes}, todos os outros casos de uso abaixo mencionados geram um \textit{Log} indicando o usuário que realizou a ação, qual ação foi tomada e o momento em que isso aconteceu. Além disso, cada \textit{Log} fica associado a uma determinada categoria. Por exemplo, o \ref{uc-enviar-mensagens} pertenceria à categoria "Envio de mensagens".


A seguir, são apresentadas as descrições de cada um dos casos de uso identificados. Os casos de uso cadastrais de baixa complexidade, envolvendo inclusão, alteração, consulta e exclusão, são descritos na Tabela~\ref{tabela-classroom-cadastrais}.


\begin{table}[H]
	\centering  \vspace{0.5cm} 	\footnotesize 
	\caption{Casos de Uso Cadastrais}
	\begin{tabular}{|c|c|c|p{6cm}|p{1.5cm}|p{2cm}|} \hline  \rowcolor[rgb]{0.8,0.8,0.8}
		
		Id & Nome  &  Ações  &  Observações & Requisitos   & Classes  \\ 	\hline \hline	
		
		{}  &  {}  &  I   & Informar: título e senha de acesso. Opcionalmente, também é possível escolher professores da turma caso estes já estejam cadastrados no sistema. &   {}   & {}    \\\cline{3-4}
		{}  &  {}  &  A   &  {}   &   {}   &  {}  \\ \cline{3-4}
		{}  &  {}  &  C  &   {}   &   {}  &   {}    \\\cline{3-4}
		\multirow{-4}{*}{\UC\label{uc-gerenciar-turmas}}   &  \multirow{-4}{*}{\parbox{2cm}{Gerenciar turmas}}   &    E    &    A exclusão de uma turma deve acarretar na exclusão de todos os itens de \textit{board} e suas interações (como submissões de tarefa e respostas de provas).   &  \multirow{-4}{1.5cm}{\ref{rf-criar-turma}}  & \multirow{-4}{2cm}{Classroom}  \\ \hline 
		
		{}  &  {}  &  I   &  Informar: título, descrição, data e tipo (Evento ou Aula). &   {}   & {}    \\\cline{3-4}
		{}  &  {}  &  A   &  {}   &   {}   &  {}  \\ \cline{3-4}
		{}  &  {}  &  C  &   {}   &   {}  &   {}    \\\cline{3-4}
		\multirow{-5}{*}{\UC\label{uc-gerenciar-calendario}}   &  \multirow{-5}{*}{\parbox{2cm}{Gerenciar eventos no calendário}}   &    E    &   {}   &  \multirow{-5}{1.5cm}{\ref{rf-gerenciamento-calendario}}  & \multirow{-5}{2cm}{Classroom, Event, Class}  \\ \hline 
		
		
	\end{tabular}
	\label{tabela-classroom-cadastrais}
\end{table}

Os casos de uso de consulta mais abrangente que as consultas a um único objeto, mas ainda de baixa complexidade, tais como consultas que combinam informações de vários objetos envolvendo filtros, estão descritos na Tabela~\ref{tabela-khoeus-consulta}.

\begin{table}[H]
	\centering  \vspace{0.5cm} 	\footnotesize 
	\caption{Casos de Uso de Consulta}
	\begin{tabular}{|c|p{2.3cm}|p{6.8cm}|c|p{1.8cm}|} \hline  \rowcolor[rgb]{0.8,0.8,0.8}
		
		Id & Nome   &  Observações & Requisitos   & Classes  \\ 	\hline	
		
		
		\UC\label{uc-consultar-presencas} & Consultar suas presenças & Os alunos de uma turma terão acesso ao seu quadro de presenças contendo a lista de aulas do curso em que estão inscritos e se estiveram presentes (ou não) nas aulas que já ocorreram.  &    \ref{rf-consultar-presenca}        & User, Classroom, Class e Presence\\ \hline
		
		
		\UC\label{uc-consultar-calendario} & Consultar calendário & Os alunos de uma turma terão acesso ao calendário contendo as aulas e eventos registrados para aquela turma. Para cada um, exibe-se o nome, a descrição (caso tenha) e a data em que acontecerá. &   \ref{rf-acessar-calendario}        & User, Classroom, Class e Event	\\ \hline
		
	\end{tabular}
	\label{tabela-khoeus-consulta}
\end{table}

\clearpage
\begin{flushright}    \textbf{Descrição de Caso de Uso}   \end{flushright}         
\noindent \textbf{Projeto:} \imprimirtitulo  \\
\textbf{Identificador do Caso de Uso:} \UC\label{uc-definir-configuracoes} \\
\textbf{Caso de Uso:} Definir configurações do sistema \\
\noindent \textbf{Descrição Sucinta:} Este caso de uso permite que o usuário altere as principais configurações do sistema, acessíveis apenas para administradores.\\

\begin{table}[H]
	\centering \vspace{0.5cm} \footnotesize
	\caption{Fluxos de Eventos Normais}
	\begin{tabular}{|p{2.3cm}|p{2.5cm}|p{10cm}|} \hline  \rowcolor[rgb]{0.8,0.8,0.8}
		
		Nome do Fluxo & Precondição & Descrição  \\ \hline		
		
		Definir configurações & O usuário  deverá estar logado e& 1. O administrador poderá alterar todas as configurações do sistema listadas no \ref{rf-configuracoes}  \\
		{} & ser um administrador & 2. Ao finalizar, o sistema salva todas essas alterações de forma genérica, armazenando o nome da configuração bem como seu respectivo valor.\\ \hline
		
		
	\end{tabular}
\end{table}

\noindent  \textbf{Requisitos Relacionados:} \ref{rf-configuracoes}       \\ \textbf{Classes Relacionadas:} User, Configuration.

\newpage
\clearpage
\begin{flushright}    \textbf{Descrição de Caso de Uso}   \end{flushright}         
\noindent \textbf{Projeto:} \imprimirtitulo  \\
\textbf{Identificador do Caso de Uso:} \UC\label{uc-ingressar-turma} \\
\textbf{Caso de Uso:} Ingressar em uma turma \\
\noindent \textbf{Descrição Sucinta:} Este caso de uso permite que um usuário se vincule a uma turma por meio de uma chave de inscrição.\\

\begin{table}[H]
	\centering \vspace{0.5cm} \footnotesize
	\caption{Fluxos de Eventos Normais}
	\begin{tabular}{|p{2.3cm}|p{2.5cm}|p{10cm}|} \hline  \rowcolor[rgb]{0.8,0.8,0.8}
		
		Nome do Fluxo & Precondição & Descrição  \\ \hline		
		
		Ingressar em uma turma & O usuário  deverá estar logado & 1. O usuário deverá escolher a turma que deseja ingressar e informar a senha de acesso fornecida por um  dos professores.  \\
		{} & {} & 2. O sistema vincula aquele usuário àquela turma em uma \textit{role} de aluno, garantindo que este possa acessá-la a qualquer momento.\\ \hline	
	\end{tabular}
\end{table}

\begin{table}[H]
	\centering \vspace{0.5cm} \footnotesize
	\caption{Fluxos de Eventos Variantes}
	\begin{tabular}{|p{2.3cm}|p{1.8cm}|p{10.7cm}|} \hline  \rowcolor[rgb]{0.8,0.8,0.8}
		
		Nome do Fluxo & Variante & Descrição  \\ \hline		
		
		Chave incorreta & O usuário digitou a chave de inscrição incorreta & 1.  O sistema informa que a chave de inscrição está errada e solicita que o usuário tente novamente.  \\ \hline 
		
	\end{tabular}
\end{table}


\noindent  \textbf{Requisitos Relacionados:} \ref{rf-inscricao}       \\ \textbf{Classes Relacionadas:} User, Classroom, Subscription.

\newpage
\clearpage
\begin{flushright}    \textbf{Descrição de Caso de Uso}   \end{flushright}         
\noindent \textbf{Projeto:} \imprimirtitulo  \\
\textbf{Identificador do Caso de Uso:} \UC\label{uc-lancar-presenca} \\
\textbf{Caso de Uso:} Lançar presença \\
\noindent \textbf{Descrição Sucinta:} Este caso de uso permite que o professor lance a presença dos alunos de sua turma para cada uma das aulas cadastradas no calendário.\\

\begin{table}[H]
	\centering \vspace{0.5cm} \footnotesize
	\caption{Fluxos de Eventos Normais}
	\begin{tabular}{|p{2.3cm}|p{2.5cm}|p{10cm}|} \hline  \rowcolor[rgb]{0.8,0.8,0.8}
		
		Nome do Fluxo & Precondição & Descrição  \\ \hline		
		
		Lançar presença & O usuário  deverá estar logado e& 1. O usuário deverá informar em uma tabela aluno x aula se o aluno esteve presente (ou não) na aula em questão.  \\
		{} & ser um professor da turma correspondente & 2. O sistema salva uma presença para cada um dos alunos que compareceram à(s) aula(s).\\ \hline
		
		
	\end{tabular}
\end{table}


\noindent  \textbf{Requisitos Relacionados:} \ref{rf-lancar-presenca}       \\ \textbf{Classes Relacionadas:} User, Classroom, Class, Presence.

\newpage
\clearpage
\begin{flushright}    \textbf{Descrição de Caso de Uso}   \end{flushright}         
\noindent \textbf{Projeto:} \imprimirtitulo  \\
\textbf{Identificador do Caso de Uso:} \UC\label{uc-definir-roles} \\
\textbf{Caso de Uso:} Definir \textit{roles} de usuários nas turmas \\
\noindent \textbf{Descrição Sucinta:} Este caso de uso permite que um administrador possa conceder a \textit{role} de professor ou de usuário a um determinado aluno.\\

\begin{table}[H]
	\centering \vspace{0.5cm} \footnotesize
	\caption{Fluxos de Eventos Normais}
	\begin{tabular}{|p{2.3cm}|p{2.5cm}|p{10cm}|} \hline  \rowcolor[rgb]{0.8,0.8,0.8}
		
		Nome do Fluxo & Precondição & Descrição  \\ \hline		
		
		Definir \textit{roles} & O usuário  deverá estar logado e ser um & 1. Em uma lista de todos os usuários inscritos naquela turma, o administrador poderá escolher qual será a \textit{role} associada àquele usuário naquela turma.  \\
		{} &  administrador do sistema  & 2.  O sistema salva a nova \textit{role} do(s) usuário(s), garantindo novas permissões para aqueles que foram alterados.\\ \hline
		
		
	\end{tabular}
\end{table}


\noindent  \textbf{Requisitos Relacionados:} \ref{rf-gerenciamento-usuario-turma}, \ref{rn-roles}       \\ \textbf{Classes Relacionadas:} User, Classroom, Subscription.

\newpage
\clearpage
\begin{flushright}    \textbf{Descrição de Caso de Uso}   \end{flushright}         
\noindent \textbf{Projeto:} \imprimirtitulo  \\
\textbf{Identificador do Caso de Uso:} \UC\label{uc-configurar-turma} \\
\textbf{Caso de Uso:} Configurar turma \\
\noindent \textbf{Descrição Sucinta:} Este caso de uso permite que o professor altere as configurações de uma turma.\\

\begin{table}[H]
	\centering \vspace{0.5cm} \footnotesize
	\caption{Fluxos de Eventos Normais}
	\begin{tabular}{|p{2.3cm}|p{2.5cm}|p{10cm}|} \hline  \rowcolor[rgb]{0.8,0.8,0.8}
		
		Nome do Fluxo & Precondição & Descrição  \\ \hline		
		
		Configurar turma & O usuário  deverá estar logado e& 1. O usuário poderá alterar todas as configurações de turma listadas no \ref{rf-gerenciamento-turma}  \\
		{} & ser um professor da turma correspondente & 2.  Ao finalizar, o sistema salva todas essas alterações.\\ \hline
	\end{tabular}
\end{table}


\noindent  \textbf{Requisitos Relacionados:} \ref{rf-gerenciamento-turma}, \ref{rn-peso}       \\ \textbf{Classes Relacionadas:} User, Classroom.

\newpage
\clearpage
\begin{flushright}    \textbf{Descrição de Caso de Uso}   \end{flushright}         
\noindent \textbf{Projeto:} \imprimirtitulo  \\
\textbf{Identificador do Caso de Uso:} \UC\label{uc-enviar-mensagens} \\
\textbf{Caso de Uso:} Enviar mensagens \\
\noindent \textbf{Descrição Sucinta:} Este caso de uso permite que usuários de uma mesma turma troquem mensagens entre si.\\

\begin{table}[H]
	\centering \vspace{0.5cm} \footnotesize
	\caption{Fluxos de Eventos Normais}
	\begin{tabular}{|p{2.3cm}|p{2.5cm}|p{10cm}|} \hline  \rowcolor[rgb]{0.8,0.8,0.8}
		
		Nome do Fluxo & Precondição & Descrição  \\ \hline		
		
		Enviar mensagens & O usuário  deverá estar logado e& 1. O usuário deverá escolher para quem deseja enviar a mensagem e preencher o campo com o texto desejado. \\
		{} & estar inscrito na turma correspondente & 2. O sistema armazena a mensagem enviada e envia um e-mail para o(s) destinatário(s) informando-o(s) sobre a nova mensagem. \\ \hline
		
		
	\end{tabular}
\end{table}

\noindent  \textbf{Requisitos Relacionados:} \ref{rf-enviar-mensagem}, \ref{rn-mensagem}       \\ \textbf{Classes Relacionadas:} User, Message.

\newpage

\section{Subsistema Board}

A Figura~\ref{figura-caso-de-uso-board} apresenta o diagrama de casos de uso do subsistema Board.

\begin{figure}[h!]
	\centering
	\includegraphics[width=\textwidth]{figuras/casos-de-uso-board.png}
	\caption{Diagrama de Casos de Uso do Subsistema Board.}
	\label{figura-caso-de-uso-board}
\end{figure}

Todos os casos de uso mencionados geram um \textit{Log} indicando o usuário que realizou a ação, qual ação foi tomada e o momento em que isso aconteceu. Além disso, cada \textit{Log} fica associado a uma determinada categoria. Por exemplo, o \ref{uc-gerenciar-arquivos} pertenceria à categoria ``Gerenciamento de arquivos''.

A seguir, são apresentadas as descrições de cada um dos casos de uso identificados. Os casos de uso cadastrais de baixa complexidade, envolvendo inclusão, alteração, consulta e exclusão, são descritos na Tabela~\ref{tabela-board-cadastrais}.


\begin{longtable}{|c|c|c|p{6.3cm}|p{1.2cm}|p{2cm}|}
	\caption{Casos de Uso Cadastrais}\\
	 \hline  \rowcolor[rgb]{0.8,0.8,0.8}
		
		Id & Nome  &  Ações  &  Observações & Requisitos   & Classes  \\ 	\hline \hline
		\endhead
		\hline
		\endlastfoot
		
		{}  &  {}  &  I   & Informar: título e, opcionalmente, descrição. No momento da criação da seção, deve-se escolher a posição do \textit{board} em que esta será inserida. &   {}   & {}    \\\cline{3-4}
		{}  &  {}  &  A   &  {}   &   {}   &  {}  \\ \cline{3-4}
		{}  &  {}  &  C  &   {}   &   {}  &   {}    \\\cline{3-4}
		\multirow{-4}{*}{\UC\label{uc-gerenciar-secoes}}   &  \multirow{-4}{*}{\parbox{2cm}{Gerenciar seções do \textit{board}}}   &    E    &    A exclusão de uma seção deverá excluir todos os itens de \textit{board} associados a ela.   &  \multirow{-4}{1.5cm}{ \ref{rf-adicionar-secao}}  & \multirow{-4}{2cm}{User, Classroom e Section}  \\ \hline 
		
		{}  &  {}  &  I   &  A seção à qual o arquivo estará vinculado será escolhida antes de sua criação. Informar título, descrição e realizar o upload do arquivo desejado. No momento da criação do arquivo, deve-se escolher a posição da seção em que esta será inserido.&   {}   & {}    \\\cline{3-4}
		{}  &  {}  &  A   &  {}   &   {}   &  {}  \\ \cline{3-4}
		{}  &  {}  &  C  &   {}   &   {}  &   {}    \\\cline{3-4}
		\multirow{-7}{*}{\UC\label{uc-gerenciar-arquivos}}   &  \multirow{-7}{*}{\parbox{2cm}{Gerenciar arquivos do \textit{board}}}   &    E    &    {}   &  \multirow{-7}{1.5cm}{ \ref{rf-adicionar-arquivo} \ref{rn-item-secao}}  & \multirow{-7}{2cm}{User, Classroom, Section e File}  \\ \hline 
		
		{}  &  {}  &  I   &  A seção à qual o link estará vinculado será escolhida antes de sua criação. Informar título, descrição e o link desejado. No momento da criação do link, deve-se escolher a posição da seção em que esta será inserido. &   {}   & {}    \\\cline{3-4}
		{}  &  {}  &  A   &  {}   &   {}   &  {}  \\ \cline{3-4}
		{}  &  {}  &  C  &   {}   &   {}  &   {}    \\\cline{3-4}
		\multirow{-6}{*}{\UC\label{uc-gerenciar-links}}   &  \multirow{-6}{*}{\parbox{2cm}{Gerenciar links do \textit{board}}}   &    E    &    {}   &  \multirow{-6}{1.5cm}{ \ref{rf-adicionar-link}, \ref{rn-item-secao}}  & \multirow{-6}{2cm}{User, Classroom, Section e Link}  \\ \hline 
		
		{}  &  {}  &  I   &  A seção à qual o questionário estará vinculado será escolhida antes de sua criação. Informar título, descrição, data de início, data de encerramento, as perguntas do questionário e suas respectivas alternativas. No momento da criação do questionário, deve-se escolher a posição da seção em que esta será inserido. &   {}   & {}    \\\cline{3-4}
		{}  &  {}  &  A   &  {}   &   {}   &  {}  \\ \cline{3-4}
		{}  &  {}  &  C  &   {}   &   {}  &   {}    \\\cline{3-4}
		\multirow{-8}{*}{\UC\label{uc-gerenciar-questionarios}}   &  \multirow{-8}{*}{\parbox{2cm}{Gerenciar questionários do \textit{board}}}   &    E    &    A exclusão de um questionário deverá resultar na exclusão de todas as respostas que já foram enviadas.   &  \multirow{-8}{1.5cm}{ \ref{rf-adicionar-questionario}, \ref{rn-item-secao}}  & \multirow{-8}{2cm}{User, Classroom, Section, Survey, SurveyQuestion e SurveyAnswer}  \\ \hline 
		
		{}  &  {}  &  I   &  A seção à qual a prova estará vinculada será escolhida antes de sua criação. Informar título, descrição, data de início, data de encerramento, as questões da prova, seus valores (nota máxima) e seus respectivos tipos (discursiva ou objetiva). No caso de questões objetivas, o professor deverá informar suas alternativas e qual delas é a correta em cada questão. No momento da criação da prova, deve-se escolher a posição da seção em que esta será inserida.&   {}   & {}    \\\cline{3-4}
		{}  &  {}  &  A   &  {}   &   {}   &  {}  \\ \cline{3-4}
		{}  &  {}  &  C  &   {}   &   {}  &   {}    \\\cline{3-4}
		\multirow{-10}{*}{\UC\label{uc-gerenciar-provas}}   &  \multirow{-10}{*}{\parbox{2cm}{Gerenciar provas do \textit{board}}}   &    E    &    A exclusão de uma prova deverá resultar na exclusão de todas as resoluções que já foram enviadas.   &  \multirow{-10}{1.5cm}{ \ref{rf-adicionar-prova}, \ref{rn-item-secao}}  & \multirow{-10}{2cm}{User, Classroom, Section, Test, TestQuestion e Text Alternative}  \\ \hline 
		
		{}  &  {}  &  I   &  A seção à qual a tarefa estará vinculada será escolhida antes de sua criação. Informar título, descrição, data de início, data de encerramento e o tipo de tarefa. No caso de tarefas do tipo arquivo, deve-se informar o número máximo de arquivos que poderão ser submetidos. No momento da criação da tarefa, deve-se escolher a posição da seção em que esta será inserida.&   {}   & {}    \\\cline{3-4}
		{}  &  {}  &  A   &  {}   &   {}   &  {}  \\ \cline{3-4}
		{}  &  {}  &  C  &   {}   &   {}  &   {}    \\\cline{3-4}
		\multirow{-8}{*}{\UC\label{uc-gerenciar-tarefas}}   &  \multirow{-8}{*}{\parbox{2cm}{Gerenciar tarefas do \textit{board}}}   &    E    &    A exclusão de uma tarefa deverá resultar na exclusão de todas as submissões que já foram feitas e de seus respectivos feedbacks.   &  \multirow{-8}{1.5cm}{ \ref{rf-adicionar-tarefa}, \ref{rn-item-secao}}  & \multirow{-8}{2cm}{User, Classroom, Section e Assignment}  \\ \hline 
		
		
		{}  &  {}  &  I   &  A seção à qual a atividade externa estará vinculada será escolhida antes de sua criação. Informar título e descrição. No momento da criação da atividade externa, deve-se escolher a posição da seção em que esta será inserida. &   {}   & {}    \\\cline{3-4}
		{}  &  {}  &  A   &  {}   &   {}   &  {}  \\ \cline{3-4}
		{}  &  {}  &  C  &   {}   &   {}  &   {}    \\\cline{3-4}
		\multirow{-7}{*}{\UC\label{uc-gerenciar-atividadesexternas}}   &  \multirow{-7}{*}{\parbox{2cm}{Gerenciar atividades externas do \textit{board}}}   &    E    &    A exclusão de uma atividade externa deverá resultar na exclusão de todas as atividades associadas a ela.   &  \multirow{-7}{1.5cm}{ \ref{rf-adicionar-atividadeexterna}, \ref{rn-item-secao}}  & \multirow{-7}{2cm}{User, Classroom, Section, External Activity} 
		
	\label{tabela-board-cadastrais}
\end{longtable}

\section{Subsistema Board Interactions}

A Figura~\ref{figura-caso-de-uso-board-interactions} apresenta o diagrama de casos de uso do subsistema Board Interactions.

\begin{figure}[h!]
	\centering
	\includegraphics[width=\textwidth]{figuras/casos-de-uso-board-interactions.png}
	\caption{Diagrama de Casos de Uso do Subsistema Board Interactions.}
	\label{figura-caso-de-uso-board-interactions}
\end{figure}

Todos os casos de uso mencionados geram um \textit{Log} indicando o usuário que realizou a ação, qual ação foi tomada e o momento em que isso aconteceu. Além disso, cada \textit{Log} fica associado a uma determinada categoria. Por exemplo, o \ref{uc-submeter-tarefa} pertenceria à categoria ``Submissão de tarefa''.

Para os casos de uso \ref{uc-responder-questionario}, \ref{uc-resolver-prova} e \ref{uc-submeter-tarefa}, a classe correspondente deve ser vinculada ao usuário que interagiu com ela. Por exemplo, no caso de uso \ref{uc-submeter-tarefa}, a classe Submission deve estar associada ao usuário que submeter aquela tarefa.

A seguir, são apresentadas as descrições de cada um dos casos de uso identificados. 


\clearpage
\begin{flushright}    \textbf{Descrição de Caso de Uso}   \end{flushright}         
\noindent \textbf{Projeto:} \imprimirtitulo  \\
\textbf{Identificador do Caso de Uso:} \UC\label{uc-download-arquivo} \\
\textbf{Caso de Uso:} Fazer download de arquivo \\
\noindent \textbf{Descrição Sucinta:} Este caso de uso permite que o usuário realize o download de um arquivo adicionado no \textit{board} da turma.\\

\begin{table}[H]
	\centering \vspace{0.5cm} \footnotesize
	\caption{Fluxos de Eventos Normais}
	\begin{tabular}{|p{2.3cm}|p{2.5cm}|p{10cm}|} \hline  \rowcolor[rgb]{0.8,0.8,0.8}
		
		Nome do Fluxo & Precondição & Descrição  \\ \hline		
		
		Fazer download & O usuário deverá estar logado e & 1. O usuário escolhe o item do \textit{board} de sua turma correspondente ao arquivo.  \\
		{}    & inscrito na turma correspondente & 2.  O download é iniciado pelo sistema.\\ \hline 
	\end{tabular}
\end{table}


\noindent  \textbf{Requisitos Relacionados:} \ref{rf-acessar-arquivo}       \\ \textbf{Classes Relacionadas:} User, Classroom e File.

\newpage
\clearpage
\begin{flushright}    \textbf{Descrição de Caso de Uso}   \end{flushright}         
\noindent \textbf{Projeto:} \imprimirtitulo  \\
\textbf{Identificador do Caso de Uso:} \UC\label{uc-acessar-link} \\
\textbf{Caso de Uso:} Acessar um link \\
\noindent \textbf{Descrição Sucinta:} Este caso de uso permite que o usuário acesse um link adicionado no \textit{board} da turma.\\

\begin{table}[H]
	\centering \vspace{0.5cm} \footnotesize
	\caption{Fluxos de Eventos Normais}
	\begin{tabular}{|p{2.3cm}|p{2.5cm}|p{10cm}|} \hline  \rowcolor[rgb]{0.8,0.8,0.8}
		
		Nome do Fluxo & Precondição & Descrição  \\ \hline		
		
		Acessar link & O usuário deverá estar logado e & 1. O usuário escolhe o item do \textit{board} de sua turma correspondente ao link.  \\
		{}    & inscrito na turma correspondente& 2. O navegador acessará a URL quando o usuário escolher o item do \textit{board} correspondente ao link.\\ \hline 
		
		
	\end{tabular}
\end{table}

\noindent  \textbf{Requisitos Relacionados:} \ref{rf-acessar-link}       \\ \textbf{Classes Relacionadas:} User, Classroom e Link.

\newpage
\clearpage
\begin{flushright}    \textbf{Descrição de Caso de Uso}   \end{flushright}         
\noindent \textbf{Projeto:} \imprimirtitulo  \\
\textbf{Identificador do Caso de Uso:} \UC\label{uc-responder-questionario} \\
\textbf{Caso de Uso:} Responder a um questionário \\
\noindent \textbf{Descrição Sucinta:} Este caso de uso permite que um aluno responda a um questionário em sua turma.\\

\begin{table}[H]
	\centering \vspace{0.5cm} \footnotesize
	\caption{Fluxos de Eventos Normais}
	\begin{tabular}{|p{2.3cm}|p{2.5cm}|p{10cm}|} \hline  \rowcolor[rgb]{0.8,0.8,0.8}
		
		Nome do Fluxo & Precondição & Descrição  \\ \hline		
		
		Responder questionário & O usuário deverá estar logado & 1. O usuário escolhe o item do \textit{board} de sua turma correspondente ao questionário que deseja responder.  \\
		{}    &  e inscrito na turma  & 2. O usuário deverá responder a todas as perguntas obrigatórias e submeter o questionário. \\
		{}    & correspondente & 3. O sistema armazena todas as respostas para aquele questionário.\\ \hline 
		
		Visualizar resultado de questionário & O usuário deverá estar logado & 1. O usuário escolhe o item do \textit{board} de sua turma correspondente ao questionário.  \\
		{}    &  e inscrito na turma correspondente  & 2. O usuário poderá visualizar o resultado do questionário. \\ \hline 
		
		
	\end{tabular}
\end{table}

\begin{table}[H]
	\centering \vspace{0.5cm} \footnotesize
	\caption{Fluxos de Eventos Variantes}
	\begin{tabular}{|p{2.3cm}|p{2.5cm}|p{10.0cm}|} \hline  \rowcolor[rgb]{0.8,0.8,0.8}
		
		Nome do Fluxo & Variante & Descrição  \\ \hline		
		
		Data limite atingida & O usuário tenta acessar um questionário em uma data posterior à limite & 1. Passada a data limite, o sistema avisa que o período para responder ao questionário já terminou.  \\ \hline 
		
	\end{tabular}
\end{table}

\noindent  \textbf{Requisitos Relacionados:} \ref{rf-acessar-questionario}, \ref{rn-submeter}       \\ \textbf{Classes Relacionadas:} User, Classroom, Survey, SurveyQuestion, SurveyAnswer e SurveyResponse.

\newpage
\clearpage
\begin{flushright}    \textbf{Descrição de Caso de Uso}   \end{flushright}         
\noindent \textbf{Projeto:} \imprimirtitulo  \\
\textbf{Identificador do Caso de Uso:} \UC\label{uc-resolver-prova} \\
\textbf{Caso de Uso:} Resolver uma prova \\
\noindent \textbf{Descrição Sucinta:} Este caso de uso permite que o usuário resolva as questões de uma prova, sendo elas discursivas ou objetivas.\\

\begin{table}[H]
	\centering \vspace{0.5cm} \footnotesize
	\caption{Fluxos de Eventos Normais}
	\begin{tabular}{|p{2.3cm}|p{2.5cm}|p{10cm}|} \hline  \rowcolor[rgb]{0.8,0.8,0.8}
		
		Nome do Fluxo & Precondição & Descrição  \\ \hline		
		
		Resolver  prova & O usuário  deverá estar logado e& 1. O usuário escolhe o item do \textit{board} de sua turma correspondente à prova desejada.  \\
		{}    &  inscrito na turma  & 2. O usuário deverá responder às perguntas objetivas e discursivas e submeter a prova.\\
		{}    &  correspondente & 3. O sistema armazena as respostas do aluno.\\ \hline
		
		
	\end{tabular}
\end{table}


\begin{table}[H]
	\centering \vspace{0.5cm} \footnotesize
	\caption{Fluxos de Eventos Variantes}
	\begin{tabular}{|p{2.3cm}|p{2.5cm}|p{10.0cm}|} \hline  \rowcolor[rgb]{0.8,0.8,0.8}
		
		Nome do Fluxo & Variante & Descrição  \\ \hline		
		
		Data limite atingida & O usuário tenta resolver uma prova em uma data posterior à limite & 1. Passada a data limite, o sistema exibe apenas a nota de cada questão e seu feedback.  \\ \hline 
		
	\end{tabular}
\end{table}


\noindent  \textbf{Requisitos Relacionados:} \ref{rf-acessar-prova}, \ref{rn-submeter}       \\ \textbf{Classes Relacionadas:} User, Classroom, Test, TestQuestion, TestAlternative, Test TextResponse e Test AlternativeResponse.

\newpage	
\clearpage
\begin{flushright}    \textbf{Descrição de Caso de Uso}   \end{flushright}         
\noindent \textbf{Projeto:} \imprimirtitulo  \\
\textbf{Identificador do Caso de Uso:} \UC\label{uc-submeter-tarefa} \\
\textbf{Caso de Uso:} Submeter uma tarefa \\
\noindent \textbf{Descrição Sucinta:} Este caso de uso permite que o usuário submeta uma tarefa do tipo especificado pelo professor no momento de sua criação.\\

\begin{table}[H]
	\centering \vspace{0.5cm} \footnotesize
	\caption{Fluxos de Eventos Normais}
	\begin{tabular}{|p{2.3cm}|p{2.5cm}|p{10cm}|} \hline  \rowcolor[rgb]{0.8,0.8,0.8}
		
		Nome do Fluxo & Precondição & Descrição  \\ \hline		
		
		Submeter tarefa & O usuário deverá estar logado e & 1. O usuário escolhe o item do \textit{board} de sua turma correspondente à tarefa desejada.  \\
		{}    &  inscrito na turma correspondente & 2. O usuário deverá submeter a tarefa da seguinte forma: caso a tarefa seja do tipo Texto ou Código, ele deverá preencher o campo com o texto/código correspondente. Caso a tarefa seja do tipo Arquivo, o usuário deverá fazer upload do(s) arquivo(s) correspondente(s). Caso a tarefa seja do tipo Código, o usuário deverá selecionar qual a linguagem de programação foi utilizada.\\
		{}    & {} & 3. O sistema armazena as informações da tarefa enviada. Caso esta seja do tipo Código, cada linha é armazenada separadamente. \\ \hline
		
		
	\end{tabular}
\end{table}

\begin{table}[H]
	\centering \vspace{0.5cm} \footnotesize
	\caption{Fluxos de Eventos Variantes}
	\begin{tabular}{|p{2.3cm}|p{2.5cm}|p{10cm}|} \hline  \rowcolor[rgb]{0.8,0.8,0.8}
		
		Nome do Fluxo & Variante & Descrição  \\ \hline		
		
		Data limite atingida & O usuário tenta submeter uma tarefa em uma data posterior à limite & 1. Passada a data limite, o sistema exibe apenas a nota da tarefa e seu feedback.  \\ \hline 
		
		Número de arquivos superior ao permitido & O usuário tenta submeter uma tarefa do tipo Arquivo fazendo upload de mais arquivos que o permitido & 1. O sistema informa o número máximo de arquivos que o usuário pode submeter.  \\ \hline 
		
	\end{tabular}
\end{table}


\noindent  \textbf{Requisitos Relacionados:} \ref{rf-acessar-tarefa}, \ref{rn-limite-arquivos}, \ref{rn-submeter}       \\ \textbf{Classes Relacionadas:} User, Classroom, Assignment, TextSubmission, CodeSubmission e FileSubmission.

\newpage


\section{Subsistema Feedback}

A Figura~\ref{figura-caso-de-uso-feedback} apresenta o diagrama de casos de uso do subsistema Feedback.

\begin{figure}[h!]
	\centering
	\includegraphics[width=\textwidth]{figuras/casos-de-uso-feedback.png}
	\caption{Diagrama de Casos de Uso do Subsistema Feedback.}
	\label{figura-caso-de-uso-feedback}
\end{figure}

Todos os casos de uso mencionados geram um \textit{Log} indicando o usuário que realizou a ação, qual ação foi tomada e o momento em que isso aconteceu. Além disso, cada \textit{Log} fica associado a uma determinada categoria. Por exemplo, o \ref{uc-consultar-notas} pertenceria à categoria ``Consulta de notas''.

A seguir, são apresentadas as descrições de cada um dos casos de uso identificados. Os casos de uso de consulta mais abrangente que as consultas a um único objeto, mas ainda de baixa complexidade, tais como consultas que combinam informações de vários objetos envolvendo filtros, estão descritos na Tabela~\ref{tabela-feedback-consulta}.

\begin{table}[H]
	\centering  \vspace{0.5cm} 	\footnotesize 
	\caption{Casos de Uso de Consulta}
	\begin{tabular}{|c|p{2cm}|p{5.7cm}|c|p{3.2cm}|} \hline  \rowcolor[rgb]{0.8,0.8,0.8}
		
		Id & Nome   &  Observações & Requisitos   & Classes  \\ 	\hline	
		
	
		\UC\label{uc-visualizar-resultado-prova} & Visualizar resultado da prova & O resultado de uma prova fica disponível para ser visualizado na tela de detalhamento da prova correspondente. Para cada prova, espera-se ver a resposta dada para cada questão, a nota atribuída a cada questão e o feedback dado pelo professor. &   \ref{rf-consultar-nota-prova}        & User, Classroom, Test, TestQuestion, TestAlternative, Test TextResponse e Test AlternativeResponse\\ \hline
		\UC\label{uc-visualizar-feedback-tarefa} & Visualizar feedback de tarefa & O resultado de uma tarefa fica disponível para ser visualizado na tela de detalhamento da tarefa correspondente. Para cada tarefa, espera-se ver a submissão feita (seja ela texto, arquivo ou código), a nota atribuída à tarefa e o feedback dado pelo professor.   &    \ref{rf-consultar-nota-tarefa}        & User, Classroom, Assignment, Submission, Text Submission, File Submission, Code Submission, Code Line, Code Line Feedback e Text Feedback\\ \hline
		\UC\label{uc-visualizar-nota-externa} & Visualizar nota de atividade externa & A nota de uma atividade externa fica disponível para ser visualizada na tela de detalhamento da atividade correspondente. Para cada atividade externa, espera-se ver a nota atribuída a ela e o feedback dado pelo professor. &   \ref{rf-consultar-nota-atividadeexterna}        & User, Classroom, External Activity e Activity\\ \hline
		\UC\label{uc-consultar-notas} & Consultar suas notas & Os alunos de uma turma terão acesso ao seu livro de notas. Para cada prova, tarefa e atividade externa realizada ao longo do curso, espera-se ver a nota atribuída a ela. Além disso, deve ser possível também consultar a média parcial no curso. &    \ref{rf-consultar-notas}        & User, Classroom, Grade Category, Assignment, Submission, Test,  Test Question, Test Alternative, Test TextResponse, Test AlternativeResponse, ExternalActivity e Activity	\\ \hline
		
		
	\end{tabular}
	\label{tabela-feedback-consulta}
\end{table}


\clearpage
\begin{flushright}    \textbf{Descrição de Caso de Uso}   \end{flushright}         
\noindent \textbf{Projeto:} \imprimirtitulo  \\
\textbf{Identificador do Caso de Uso:} \UC\label{uc-download-lote} \\
\textbf{Caso de Uso:} Fazer download em lote das submissões de uma tarefa ou prova \\
\noindent \textbf{Descrição Sucinta:} Este caso de uso permite que o professor efetue o download de todas as submissões de uma tarefa ou das respostas de uma prova simultaneamente.\\

\begin{table}[H]
	\centering \vspace{0.5cm} \footnotesize
	\caption{Fluxos de Eventos Normais}
	\begin{tabular}{|p{2.3cm}|p{2.5cm}|p{10cm}|} \hline  \rowcolor[rgb]{0.8,0.8,0.8}
		
		Nome do Fluxo & Precondição & Descrição  \\ \hline		
		
		Download em lote & O usuário deverá estar logado e & 1. O professor escolhe o item do \textit{board} de sua turma correspondente à tarefa/prova desejada.  \\
		{} &  ser professor na turma correspondente & 2. Ao clicar sobre o botão correspondente, dar-se-á início ao download de um arquivo compactado contendo todas as provas/tarefas submetidas.\\ \hline
		
		
	\end{tabular}
\end{table}


\noindent  \textbf{Requisitos Relacionados:} \ref{rf-download-lote}       \\ \textbf{Classes Relacionadas:} User, Classroom, Test, TestQuestion, TestAnswer, Assignment, TextSubmission, CodeSubmission e FileSubmission.

\newpage
\clearpage
\begin{flushright}    \textbf{Descrição de Caso de Uso}   \end{flushright}         
\noindent \textbf{Projeto:} \imprimirtitulo  \\
\textbf{Identificador do Caso de Uso:} \UC\label{uc-avaliar-individualmente} \\
\textbf{Caso de Uso:} Avaliar tarefas, provas ou atividades externas \\
\noindent \textbf{Descrição Sucinta:} Este caso de uso permite que um professor avalie uma tarefa, prova ou atividade externa de um determinado aluno, sendo possível verificar o que foi submetido.\\

\begin{table}[H]
	\centering \vspace{0.5cm} \footnotesize
	\caption{Fluxos de Eventos Normais}
	\begin{tabular}{|p{2.3cm}|p{2.5cm}|p{10cm}|} \hline  \rowcolor[rgb]{0.8,0.8,0.8}
		
		Nome do Fluxo & Precondição & Descrição  \\ \hline		
		
		Avaliação individual & O usuário deverá estar logado& 1. O usuário escolhe o item do \textit{board} de sua turma correspondente à tarefa/prova/atividade externa desejada.  \\
		{}    &  e ser professor na turma correspondente & 2. O professor deverá escolher um dos alunos para que possa avaliar. A menos que seja uma atividade externa, ele terá acesso à submissão do aluno: no caso de uma prova, todas as questões e respostas são exibidas. No caso de uma tarefa, o usuário poderá fazer download dos arquivos submetidos (caso a tarefa seja do tipo Arquivo) ou visualizar diretamente na tela o conteúdo enviado (caso seja do tipo Texto ou Código). \\
		{}    &  & 3. O professor deverá informar a nota correspondente àquela tarefa/prova/atividade e, se desejar, um texto de feedback. No caso de uma tarefa do tipo Código, o professor poderá realizar comentários para cada linha de código. No caso de uma prova, cada questão terá uma nota individual e a soma será feita automaticamente. No caso de uma atividade externa, a avaliação da atividade deverá estar associada ao usuário que realizou a tarefa.\\ \hline
		
		Avaliação em lote& O usuário deverá estar logado e & 1. O usuário escolhe o item do \textit{board} de sua turma correspondente à tarefa/prova/atividade externa.  \\
		{}     &  ser professor na turma correspondente & 2. O usuário deverá realizar o download de uma planilha no formato .xls já preenchida com o nome dos alunos e que deverá ser completada com a nota de cada um dos alunos e, opcionalmente, um texto de feedback. No caso de uma prova, a planilha conterá uma coluna para cada questão, sendo possível avaliá-las individualmente. No caso de tarefas de código, o feedback fica restrito a um feedback do tipo texto, não sendo possivel avaliar linhas de código separadamente. No caso de uma atividade externa, a avaliação da atividade deverá estar associada ao usuário que realizou a tarefa.\\
		{}    &  {} & 3. O professor deverá efetuar o upload da planilha devidamente preenchida. \\ 
		{}    & {} & 4. O sistema irá armazenar as notas de todos os alunos para aquela atividade. \\ \hline
		
	\end{tabular}
\end{table}

\noindent  \textbf{Requisitos Relacionados:} \ref{rf-avaliar-tarefa}, \ref{rn-avaliar}, \ref{rn-nota}       \\ \textbf{Classes Relacionadas:} User, Classroom, Test, TestQuestion, Test TextResponse, Assignment, Submission, TextFeedback, CodeLineFeedback.


\newpage
\section{Subsistema Discussion}

A Figura~\ref{figura-caso-de-uso-discussion} apresenta o diagrama de casos de uso do subsistema Discussions.

\begin{figure}[h!]
	\centering
	\includegraphics[width=\textwidth]{figuras/casos-de-uso-discussion.png}
	\caption{Diagrama de Casos de Uso do Subsistema Discussion.}
	\label{figura-caso-de-uso-discussion}
\end{figure}

Com exceção do caso de uso \ref{uc-deletar-livequestion}, todos os outros casos de uso mencionados geram um \textit{Log} indicando o usuário que realizou a ação, qual ação foi tomada e o momento em que isso aconteceu. Além disso, cada \textit{Log} fica associado a uma determinada categoria. Por exemplo, o \ref{uc-criar-topico} pertenceria à categoria ``Criação de tópico''.

No caso das \textit{Live Questions}, o fluxo consiste no fato do aluno \textbf{Adicionar nova Live Question} enquanto o professor poderá \textbf{Visualizar Live Questions}. Como o intuito é de que essas perguntas sejam feitas em momento de aula, elas serão excluídas passadas 24 horas de sua criação, conforme Figura~\ref{fig-requisitos-discussion-diagrama-estado}.

\begin{figure}[h]
	\centering
	\includegraphics[width=0.2\textwidth]{figuras/diagrama-estado-livequestion}
	\caption{Diagrama de Estados para uma LiveQuestion.}
	\label{fig-requisitos-discussion-diagrama-estado}
\end{figure}

Para os casos de uso \ref{uc-criar-topico} e \ref{uc-gerenciar-noticias}, a classe correspondente deve ser vinculada ao usuário que interagiu com ela. Por exemplo, no caso de uso \ref{uc-criar-topico}, a classe Forum Topic deve estar associada ao usuário que o criou. No caso do caso de uso \ref{uc-adicionar-livequestion}, o usuário só será vinculado à classe caso o usuário não tenha escolhido permanecer no anonimato.

A seguir, são apresentadas as descrições de cada um dos casos de uso identificados.  Os casos de uso cadastrais de baixa complexidade, envolvendo inclusão, alteração, consulta e exclusão, são descritos na Tabela~\ref{tabela-discussion-cadastrais}.

\newpage

\begin{longtable}{|c|c|c|p{6.3cm}|p{1.2cm}|p{2cm}|}
	\caption{Casos de Uso Cadastrais}\\
	\hline  \rowcolor[rgb]{0.8,0.8,0.8}
	
	Id & Nome  &  Ações  &  Observações & Requisitos   & Classes  \\ 	\hline \hline
	\endhead
	\hline
	\endlastfoot
	
	{}  &  {}  &  I   &  Informar: título e conteúdo. A inclusão de uma nova notícia deve enviar um email para todos os alunos daquela turma contendo seu conteúdo. A notícia criada deve estar associada ao usuário que a criou. &   {}   & {}    \\\cline{3-4}
	{}  &  {}  &  A   &  {}   &   {}   &  {}  \\ \cline{3-4}
	{}  &  {}  &  C  &   {}   &   {}  &   {}    \\\cline{3-4}
	\multirow{-7}{*}{\UC\label{uc-gerenciar-noticias}}   &  \multirow{-7}{*}{\parbox{2cm}{Gerenciar notícias}}   &    E    &   {}   &  \multirow{-7}{1.5cm}{\ref{rf-adicionar-noticia}}  & \multirow{-7}{2cm}{User, Classroom, News} \\ \hline
	
	{}  &  {}  &  I   & Descrito em  \ref{uc-criar-topico} &   {}   & {}    \\\cline{3-4}
	{}  &  {}  &  A   &  {}   &   {}   &  {}  \\ \cline{3-4}
	{}  &  {}  &  C  &   {}   &   {}  &   {}    \\\cline{3-4}
	\multirow{-3}{*}{\UC\label{uc-gerenciar-topicos}}   &  \multirow{-3}{*}{\parbox{2cm}{Gerenciar tópicos do fórum}}   &    E    &   A exclusão de um tópico deve acarretar na exclusão de todas as réplicas vinculadas a ele.   &  \multirow{-3}{1.5cm}{\ref{rf-gerenciamento-forum}}  & \multirow{-3}{2cm}{User, Classroom, ForumTopic e ForumReply} 
	
	\label{tabela-discussion-cadastrais}
\end{longtable}


Os casos de uso de consulta mais abrangente que as consultas a um único objeto, mas ainda de baixa complexidade, tais como consultas que combinam informações de vários objetos envolvendo filtros, estão descritos na Tabela~\ref{tabela-board-consulta}.

\begin{table}[H]
	\centering  \vspace{0.5cm} 	\footnotesize 
	\caption{Casos de Uso de Consulta}
	\begin{tabular}{|c|p{2cm}|p{5.7cm}|c|p{3.2cm}|} \hline  \rowcolor[rgb]{0.8,0.8,0.8}
		
		Id & Nome   &  Observações & Requisitos   & Classes  \\ 	\hline	
		
		
		\UC\label{uc-visualizar-noticia} & Visualizar as notícias da turma & As notícias de uma turma ficam disponíveis para serem visualizadas na lista de notícias no topo do \textit{board}. Para cada notícia, espera-se ver o título, conteúdo, data de publicação, nome do autor e foto do autor. &    \ref{rf-visualizar-noticia}        & User, Classroom e News	\\ \hline
		\UC\label{uc-visualizar-livequestions} & Visualizar \textit{Live Questions} & O professor de uma turma terá acesso a uma lista com todas as \textit{Live Questions} ativas no momento para que possa visualizá-las e responder as dúvidas em sala de aula. &    \ref{rf-visualizar-livequestions}       & User, Classroom e LiveQuestion	\\ \hline
		
		
	\end{tabular}
	\label{tabela-board-consulta}
\end{table}

\clearpage
\begin{flushright}    \textbf{Descrição de Caso de Uso}   \end{flushright}         
\noindent \textbf{Projeto:} \imprimirtitulo  \\
\textbf{Identificador do Caso de Uso:} \UC\label{uc-criar-topico} \\
\textbf{Caso de Uso:} Criar tópico no fórum de discussão \\
\noindent \textbf{Descrição Sucinta:} Este caso de uso permite que o usuário crie um novo tópico no fórum de discussões de uma turma.\\

\begin{table}[H]
	\centering \vspace{0.5cm} \footnotesize
	\caption{Fluxos de Eventos Normais}
	\begin{tabular}{|p{2.3cm}|p{2.5cm}|p{10cm}|} \hline  \rowcolor[rgb]{0.8,0.8,0.8}
		
		Nome do Fluxo & Precondição & Descrição  \\ \hline		
		
		Criar tópico & O usuário & 1. O usuário irá acessar o fórum de discussões da turma.  \\
		no fórum de     & deverá estar  & 2. O usuário deve inserir o título e o conteúdo do tópico.\\
		discussões    & logado e inscrito na turma correspondente & 3. O sistema salva o novo tópico.\\ \hline 
		
		Responder tópico no fórum & O usuário deverá estar logado e & 1. O usuário irá acessar o fórum de discussões de sua turma e selecionar o tópico que deseja responder.  \\
		de discussões    &   e inscrito & 2. O usuário deverá inserir o conteúdo de sua resposta ao tópico.\\
		{}    & na turma correspondente & 3. O sistema salva as informações da nova réplica. \\  \hline 
		
		Visualizar os tópicos & O usuário deverá estar logado e & 1. O usuário irá acessar o fórum de discussões de sua turma e selecionar o tópico que deseja visualizar.  \\
	 	do fórum de discussão    &   e inscrito & 2. O usuário poderá visualizar o tópico e suas respostas. \\ \hline 
		
		
	\end{tabular}
\end{table}


\noindent  \textbf{Requisitos Relacionados:} \ref{rf-gerenciamento-forum}       \\ \textbf{Classes Relacionadas:} User, Classroom e ForumTopic.

\newpage
\clearpage
\begin{flushright}    \textbf{Descrição de Caso de Uso}   \end{flushright}         
\noindent \textbf{Projeto:} \imprimirtitulo  \\
\textbf{Identificador do Caso de Uso:} \UC\label{uc-adicionar-livequestion} \\
\textbf{Caso de Uso:} Adicionar nova Live Question \\
\noindent \textbf{Descrição Sucinta:} Este caso de uso permite que o aluno de uma turma crie uma nova \textit{Live Question}.\\

\begin{table}[H]
	\centering \vspace{0.5cm} \footnotesize
	\caption{Fluxos de Eventos Normais}
	\begin{tabular}{|p{2.3cm}|p{2.5cm}|p{10cm}|} \hline  \rowcolor[rgb]{0.8,0.8,0.8}
		
		Nome do Fluxo & Precondição & Descrição  \\ \hline		
		
		Adiciona nova \textit{LiveQuestion}& O usuário deverá estar logado e & 1. O usuário escolhe o item do \textit{board} de sua turma correspondente à lista de \textit{LiveQuestions}  \\
		{}    &  inscrito na turma  & 2. O usuário deverá informar qual é a sua dúvida e se deseja permanecer no anonimato ou não.\\
		{}    & correspondente & 3. O sistema armazena as informações da dúvida e o momento de sua criação. Após 24h que a dúvida foi submetida, o sistema deve deletá-la automaticamente. \\ \hline
		
		
	\end{tabular}
\end{table}



\noindent  \textbf{Requisitos Relacionados:} \ref{rf-adicionar-livequestion}      \\ \textbf{Classes Relacionadas:} User, Classroom, LiveQuestion.

\newpage

\section{Subsistema Log}

A Figura~\ref{figura-caso-de-uso-log} apresenta o diagrama de casos de uso do subsistema Log.

\begin{figure}[h!]
	\centering
	\includegraphics[scale=0.7]{figuras/casos-de-uso-log.png}
	\caption{Diagrama de Casos de Uso do Subsistema Log.}
	\label{figura-caso-de-uso-log}
\end{figure}

Os casos de uso de consulta mais abrangente que as consultas a um único objeto, mas ainda de baixa complexidade, tais como consultas que combinam informações de vários objetos envolvendo filtros, estão descritos na Tabela~\ref{tabela-log-consulta}.

\begin{table}[H]
	\centering  \vspace{0.5cm} 	\footnotesize 
	\caption{Casos de Uso de Consulta}
	\begin{tabular}{|c|p{2.3cm}|p{6.8cm}|c|p{1.8cm}|} \hline  \rowcolor[rgb]{0.8,0.8,0.8}
		
		Id & Nome   &  Observações & Requisitos   & Classes  \\ 	\hline	
		
		
		\UC\label{uc-logs-alunos} & Visualizar logs dos alunos & Dentro de uma turma em que o usuário é professor, ele poderá visualizar os logs de todas as ações realizadas pelos alunos daquela turma, podendo filtrá-los pela data do log, categoria ou pelo usuário que está vinculado a ele. Para cada log, espera-se ver sua data de criação e qual ação foi realizada.&    \ref{rf-gerar-log-prof}        & User e Log	\\ \hline
		\UC\label{uc-logs-sistema} & Visualizar logs do sistema & Administradores do sistema podem visualizar os logs de todas as ações de todos os usuários dentro do sistema, podendo filtrá-los pela data do log, categoria ou pelo usuário que está vinculado a ele. Para cada log, espera-se ver sua data de criação e qual ação foi realizada. &    \ref{rf-gerar-log-admin}        & User e Log	\\ \hline
		
	\end{tabular}
	\label{tabela-log-consulta}
\end{table}

\newpage

\chapter{Modelo Estrutural}
\label{sec-modelo-estrutural}

O modelo conceitual estrutural visa capturar e descrever as informações (classes, associações e atributos) que o sistema deve representar para prover as funcionalidades descritas na seção anterior. A seguir, são apresentados os diagramas de classes de cada um dos subsistemas identificados no contexto deste projeto. Na Seção~\ref{sec-dicionario} – Dicionário de Projeto – são apresentadas as descrições das classes, atributos e operações presentes nos diagramas apresentados nesta seção.

 Vale ressaltar que algumas associações são consideradas obrigatórias nos dois sentidos pois considera-se que o tempo decorrido entre a criação de ambas é extremamente pequeno. Além disso, por estarmos considerando \textbf{entidades}, uma só faria sentido quando estivesse associado à outra.

\section{Subsistema Classroom}


A Figura~\ref{figura-classroom-classe} apresenta o diagrama de classes do subsistema Classroom.

\begin{figure}[h]
	\centering
	\includegraphics[width=\textwidth]{figuras/diagrama-classe-classroom.png}
	\caption{Diagrama de Classes do subsistema Classroom.}
	\label{figura-classroom-classe}
\end{figure} 

\newpage

\section{Subsistema Board}


A Figura~\ref{figura-board-classe} apresenta o diagrama de classes do subsistema Board.

\begin{figure}[h]
	\centering
	\includegraphics[width=\textwidth]{figuras/diagrama-classe-board.png}
	\caption{Diagrama de Classes do Subsistema Board.}
	\label{figura-board-classe}
\end{figure} 
\newpage

\section{Subsistema Board Interactions}


A Figura~\ref{figura-board-interactions-classe} apresenta o diagrama de classes do subsistema Board Interactions.

\begin{figure}[h]
	\centering
	\includegraphics[width=\textwidth]{figuras/diagrama-classe-board-interactions.png}
	\caption{Diagrama de Classes do Subsistema Board Interactions.}
	\label{figura-board-interactions-classe}
\end{figure} 
\newpage

\section{Subsistema Feedback}


A Figura~\ref{figura-feedback-classe} apresenta o diagrama de classes do subsistema Feedback.

\begin{figure}[h]
	\centering
	\includegraphics[width=\textwidth]{figuras/diagrama-classe-feedback.png}
	\caption{Diagrama de Classes do Subsistema Feedback.}
	\label{figura-feedback-classe}
\end{figure} 
\newpage

\section{Subsistema Discussion}


A Figura~\ref{figura-discussion-classe} apresenta o diagrama de classes do subsistema Discussion.

\begin{figure}[h]
	\centering
	\includegraphics[width=0.6\textwidth]{figuras/diagrama-classe-discussion.png}
	\caption{Diagrama de Classes do Subsistema Discussion.}
	\label{figura-discussion-classe}
\end{figure} 
\newpage

\section{Subsistema Log}


A Figura~\ref{figura-log-classe} apresenta o diagrama de classes do subsistema Log.

\begin{figure}[h]
	\centering
	\includegraphics[scale=0.7]{figuras/diagrama-classe-log.png}
	\caption{Diagrama de Classes do Subsistema Log.}
	\label{figura-log-classe}
\end{figure} 

\newpage




\newpage

\chapter{Dicionário de Projeto}
\label{sec-dicionario}
\setcounter{table}{0}

Esta seção apresenta as definições das classes (e seus atributos), servindo como um glossário do projeto. As definições são organizadas por subsistema. Vale destacar que eventuais operações que estas classes vierem a ter não são listadas e descritas nesta fase do projeto. Além disse, algumas classes estão envolvidas em mais de um subsistema. Nesses casos, a classe será descrita apenas uma vez.


\section{Subsistema Classroom}

\subsection{User} \label{User}
\begin{table}[H]
	\footnotesize
	\begin{tabularx}{\textwidth}{|p{2.6cm}|X|c|p{7.8cm}|}   \hline \rowcolor[rgb]{0.8,0.8,0.8}
		
		\textbf{Propriedade} & \textbf{Tipo} & \textbf{Obrigatório?} & \centerline{\textbf{Descrição}} \\\hline  	
		
		email & Texto & x & Email do usuário utilizado para realizar login. \\\hline	
		photo & Arquivo & {} & Arquivo referente à foto do usuário. \\\hline	
		fullname & Texto & x & Nome completo do usuário. \\\hline	
		cep & Texto & x & CEP da residência do usuário. \\\hline	
		address & Texto & x & Logradouro da residência do usuário. \\\hline	
		number & Número & x & Número da residência do usuário. \\\hline	
		complement & Texto & {} & Complemento da residência do usuário. \\\hline	
		neighborhood & Texto & x & Bairro do usuário. \\\hline	
		city & Texto & x & Cidade do usuário. \\\hline	
		state & Texto & x & Estado do usuário. \\\hline	
		country & Texto & x & País do usuário. \\\hline	
		
	\end{tabularx}	
\end{table}


%%%%%       CLASSE CLASSROOM        %%%%%%%%%%%%%%%%%%%%%%%%%%%%%%%%%%%%%%%%%%
\subsection{Classroom} \label{Classroom}
\begin{table}[H]
	\footnotesize
	\begin{tabularx}{\textwidth}{|p{2.6cm}|X|c|p{7.8cm}|}   \hline \rowcolor[rgb]{0.8,0.8,0.8}
		
		\textbf{Propriedade} & \textbf{Tipo} & \textbf{Obrigatório?} & \centerline{\textbf{Descrição}} \\\hline  	
		
		name & Texto & x & Nome da turma. \\\hline		
		password & Texto & {} & Senha de acesso para a turma. \\\hline	
		has\_grades & Booleano & x & Indica se a turma tem livro de notas. \\\hline	
		has\_attendance & Booleano & x & Indica se a turma tem lista de presença. \\ \hline	
		minimum\_grade & Número & x & Nota mínima exigida para a aprovação dos alunos. \\\hline
		users & User & {} & Usuários inscritos na turma. \\\hline
		grade\_categories & Grade Category & {} & Categorias de nota associadas à turma para cálculo da média final dos alunos. \\\hline
		events & Event & {} & Eventos adicionados ao calendário da turma \\\hline
		classes & Class & {} & Aulas adicionadas ao calendário da turma \\\hline	
		
	\end{tabularx}	
\end{table}

\subsection{Subscription} \label{Subscription}
\begin{table}[H]
	\footnotesize
	\begin{tabularx}{\textwidth}{|X|X|c|p{7.8cm}|}   \hline \rowcolor[rgb]{0.8,0.8,0.8}
		
		\textbf{Propriedade} & \textbf{Tipo} & \textbf{Obrigatório?} & \centerline{\textbf{Descrição}} \\\hline  	
		
		role & Texto & x & Papel atribuído ao usuário naquela turma (Professor ou Aluno). \\\hline
		user & User & x & Usuário que está se inscrevendo na turma. \\\hline
		classroom & Classroom & x & Turma em que o usuário está se inscrevendo. \\\hline		
		
	\end{tabularx}	
\end{table}


%%%%%       CLASSE GRADECATEGORY        %%%%%%%%%%%%%%%%%%%%%%%%%%%%%%%%%%%%%%%%%%
\subsection{Grade Category} \label{Grade Category}
\begin{table}[H]
	\footnotesize
	\begin{tabularx}{\textwidth}{|X|X|c|p{7.8cm}|}   \hline \rowcolor[rgb]{0.8,0.8,0.8}
		
		\textbf{Propriedade} & \textbf{Tipo} & \textbf{Obrigatório?} & \centerline{\textbf{Descrição}} \\\hline  	
		
		title & Texto & x & Nome da categoria. \\\hline		
		weight & Número & x & Peso da categoria no cálculo da média. \\\hline	
		classroom & Classroom & x & Turma que utiliza a categoria de nota para cálculo da média. \\\hline
		
	\end{tabularx}	
\end{table}


%%%%%       CLASSE MESSAGE        %%%%%%%%%%%%%%%%%%%%%%%%%%%%%%%%%%%%%%%%%%
\subsection{Message} \label{Message}
\begin{table}[H]
	\footnotesize
	\begin{tabularx}{\textwidth}{|X|X|c|p{7.8cm}|}   \hline \rowcolor[rgb]{0.8,0.8,0.8}
		
		\textbf{Propriedade} & \textbf{Tipo} & \textbf{Obrigatório?} & \centerline{\textbf{Descrição}} \\\hline  	
		
		content & Texto & x & Conteúdo da mensagem. \\\hline		
		created\_at & Data & x & Timestamp do momento em que a mensagem foi criado. \\\hline
		read & Booleano & x & Indica se a mensagem já foi lida. \\\hline
		sent\_by & User & x & Usuário que enviou a mensagem. \\\hline
		received\_by & User & x & Usuários que receberam a mensagem. \\\hline
		
	\end{tabularx}	
\end{table}

%%%%%       CLASSE EVENT        %%%%%%%%%%%%%%%%%%%%%%%%%%%%%%%%%%%%%%%%%%
\subsection{Event} \label{Event}
\begin{table}[H]
	\footnotesize
	\begin{tabularx}{\textwidth}{|X|X|c|p{7.8cm}|}   \hline \rowcolor[rgb]{0.8,0.8,0.8}
		
		\textbf{Propriedade} & \textbf{Tipo} & \textbf{Obrigatório?} & \centerline{\textbf{Descrição}} \\\hline  	
		
		title & Texto & x & Nome do evento. \\\hline		
		description & Texto & {} & Descrição do evento \\\hline		
		date & Data & x & Timestamp da data em que o evento ocorrerá. \\\hline	
		classroom & Classroom & x & Turma na qual o evento foi adicionado ao calendário. \\\hline
		
	\end{tabularx}	
\end{table}

%%%%%       CLASSE CLASS        %%%%%%%%%%%%%%%%%%%%%%%%%%%%%%%%%%%%%%%%%%
\subsection{Class} \label{Class}
\begin{table}[H]
	\footnotesize
	\begin{tabularx}{\textwidth}{|X|X|c|p{7.8cm}|}   \hline \rowcolor[rgb]{0.8,0.8,0.8}
		
		\textbf{Propriedade} & \textbf{Tipo} & \textbf{Obrigatório?} & \centerline{\textbf{Descrição}} \\\hline  	
		
		title & Texto & {} & Nome associado à aula. \\\hline		
		description & Texto & {} & Descrição da aula \\\hline		
		date & Data & x & Timestamp da data em que a aula ocorrerá. \\\hline
		classroom & Classroom & x & Turma na qual a aula foi adicionada ao calendário. \\\hline
		students & User & {} & Alunos que compareceram (ou não) à aula. \\\hline	
		
	\end{tabularx}	
\end{table}


%%%%%       CLASSE PRESENCE        %%%%%%%%%%%%%%%%%%%%%%%%%%%%%%%%%%%%%%%%%%
\subsection{Presence} \label{Presence}
\begin{table}[H]
	\footnotesize
	\begin{tabularx}{\textwidth}{|X|X|c|p{7.8cm}|}   \hline \rowcolor[rgb]{0.8,0.8,0.8}
		
		\textbf{Propriedade} & \textbf{Tipo} & \textbf{Obrigatório?} & \centerline{\textbf{Descrição}} \\\hline  	
		
		present & Booleano & x & Indica se o aluno esteve presente ou não à aula correspondente. \\\hline	
		class & Class & x & Aula à qual o aluno esteve presente (ou não). \\\hline
		user & User & x & Aluno do qual a presença se trata. \\\hline
		
	\end{tabularx}	
\end{table}




%%%%%       CLASSE CONFIGURATION        %%%%%%%%%%%%%%%%%%%%%%%%%%%%%%%%%%%%%%%%%%
\subsection{Configuration} \label{Configuration}
\begin{table}[H]
	\footnotesize
	\begin{tabularx}{\textwidth}{|X|X|c|p{7.8cm}|}   \hline \rowcolor[rgb]{0.8,0.8,0.8}
		
		\textbf{Propriedade} & \textbf{Tipo} & \textbf{Obrigatório?} & \centerline{\textbf{Descrição}} \\\hline  	
		
		name & Texto & x & Nome da configuração \\\hline		
		value & Data & x & Valor associado à configuração \\\hline	
		
	\end{tabularx}	
\end{table}

\newpage

\section{Subsistema Board}

%%%%%       CLASSE SECTION        %%%%%%%%%%%%%%%%%%%%%%%%%%%%%%%%%%%%%%%%%%
\subsection{Section} \label{Section}
\begin{table}[H]
	\footnotesize
	\begin{tabularx}{\textwidth}{|X|X|c|p{7.8cm}|}   \hline \rowcolor[rgb]{0.8,0.8,0.8}
		
		\textbf{Propriedade} & \textbf{Tipo} & \textbf{Obrigatório?} & \centerline{\textbf{Descrição}} \\\hline  	
		
		title & Texto & {} & Título da seção. \\\hline		
		description & Texto & {} & Descrição da seção. \\\hline		
		position & Número & x & Posição da seção no board. \\\hline		
		created\_at & Data & x & Timestamp do momento em que a seção foi criada. \\\hline	
		classroom & Classroom & x & Turma à qual a seção está vinculada. \\\hline
		items & Board Item & {} & Items do board vinculados à seção. \\\hline
		
	\end{tabularx}	
\end{table}


%%%%%       CLASSE BOARDITEM        %%%%%%%%%%%%%%%%%%%%%%%%%%%%%%%%%%%%%%%%%%
\subsection{BoardItem} \label{BoardItem}
\begin{table}[H]
	\footnotesize
	\begin{tabularx}{\textwidth}{|X|X|c|p{7.8cm}|}   \hline \rowcolor[rgb]{0.8,0.8,0.8}
		
		\textbf{Propriedade} & \textbf{Tipo} & \textbf{Obrigatório?} & \centerline{\textbf{Descrição}} \\\hline  	
		
		title & Texto & x & Título do item. \\\hline		
		description & Texto & {} & Descrição do item. \\\hline		
		position & Número & x & Posição do item na seção. \\\hline		
		created\_at & Data & x & Timestamp do momento em que o item foi criada. \\\hline	
		section & Section & x & Seção à qual o item está vinculado. \\\hline
		
	\end{tabularx}	
\end{table}


%%%%%       CLASSE FILE        %%%%%%%%%%%%%%%%%%%%%%%%%%%%%%%%%%%%%%%%%%
\subsection{File} \label{File}
\begin{table}[H]
	\footnotesize
	\begin{tabularx}{\textwidth}{|X|X|c|p{7.8cm}|}   \hline \rowcolor[rgb]{0.8,0.8,0.8}
		
		\textbf{Propriedade} & \textbf{Tipo} & \textbf{Obrigatório?} & \centerline{\textbf{Descrição}} \\\hline  	
		
		file & Arquivo & x & Arquivo a ser baixado. \\\hline			
		
	\end{tabularx}	
\end{table}



%%%%%       CLASSE LINK        %%%%%%%%%%%%%%%%%%%%%%%%%%%%%%%%%%%%%%%%%%
\subsection{Link} \label{Link}
\begin{table}[H]
	\footnotesize
	\begin{tabularx}{\textwidth}{|X|X|c|p{7.8cm}|}   \hline \rowcolor[rgb]{0.8,0.8,0.8}
		
		\textbf{Propriedade} & \textbf{Tipo} & \textbf{Obrigatório?} & \centerline{\textbf{Descrição}} \\\hline  	
		
		url & Texto & x & URL da página a ser acessada. \\\hline			
		
	\end{tabularx}	
\end{table}



%%%%%       CLASSE SURVEY        %%%%%%%%%%%%%%%%%%%%%%%%%%%%%%%%%%%%%%%%%%
\subsection{Survey} \label{Survey}
\begin{table}[H]
	\footnotesize
	\begin{tabularx}{\textwidth}{|X|X|c|p{7.8cm}|}   \hline \rowcolor[rgb]{0.8,0.8,0.8}
		
		\textbf{Propriedade} & \textbf{Tipo} & \textbf{Obrigatório?} & \centerline{\textbf{Descrição}} \\\hline  	
		
		start & Data & x & Timestamp do momento em que o questionário deve ser iniciado. \\\hline		
		end & Data & x & Timestamp do momento em que o questionário deve ser encerrado. \\\hline
		questions & Survey Question & x & Questões do questionário \\\hline			
		
	\end{tabularx}	
\end{table}


%%%%%       CLASSE SURVEY QUESTION       %%%%%%%%%%%%%%%%%%%%%%%%%%%%%%%%%%%%%%%%%%
\subsection{Survey Question} \label{Survey Question}
\begin{table}[H]
	\footnotesize
	\begin{tabularx}{\textwidth}{|X|X|c|p{7.8cm}|}   \hline \rowcolor[rgb]{0.8,0.8,0.8}
		
		\textbf{Propriedade} & \textbf{Tipo} & \textbf{Obrigatório?} & \centerline{\textbf{Descrição}} \\\hline  	
		
		question & Texto & x & Pergunta do questionário que deve ser respondida. \\\hline			
		required & Booleano & x & Indica se a questão é obrigatória ou não. \\\hline	
		survey & Survey & x & Questionário ao qual a pergunta está vinculada. \\\hline
		answers & Survey Answer & x & Alternativas possíveis para a questão. \\\hline	
		
	\end{tabularx}	
\end{table}


%%%%%       CLASSE SURVEY ANSWER       %%%%%%%%%%%%%%%%%%%%%%%%%%%%%%%%%%%%%%%%%%
\subsection{Survey Answer} \label{Survey Answer}
\begin{table}[H]
	\footnotesize
	\begin{tabularx}{\textwidth}{|X|X|c|p{7.8cm}|}   \hline \rowcolor[rgb]{0.8,0.8,0.8}
		
		\textbf{Propriedade} & \textbf{Tipo} & \textbf{Obrigatório?} & \centerline{\textbf{Descrição}} \\\hline  	
		
		answer & Texto & x & Resposta para uma pergunta do questionário. \\\hline
		question & Survey Question & x & Pergunta do questionário à qual a resposta está vinculada. \\\hline
		responses & Survey Response & x	& Associação com o usuário que selecionou a opção. \\\hline 		
		
	\end{tabularx}	
\end{table}


%%%%%       CLASSE TEST       %%%%%%%%%%%%%%%%%%%%%%%%%%%%%%%%%%%%%%%%%%
\subsection{Test} \label{Test}
\begin{table}[H]
	\footnotesize
	\begin{tabularx}{\textwidth}{|X|X|c|p{7.8cm}|}   \hline \rowcolor[rgb]{0.8,0.8,0.8}
		
		\textbf{Propriedade} & \textbf{Tipo} & \textbf{Obrigatório?} & \centerline{\textbf{Descrição}} \\\hline  	
		
		start & Data & x & Timestamp do momento em que a prova deve ser iniciada. \\\hline			
		end & Data & x & Timestamp do momento em que a prova deve ser encerrada. \\\hline
		questions & Test Question & x & Questões da prova.			\\\hline
		category & Grade Category & x & Categoria de nota à qual a prova está associada. \\\hline
		
		
	\end{tabularx}	
\end{table}



%%%%%       CLASSE TEST QUESTION      %%%%%%%%%%%%%%%%%%%%%%%%%%%%%%%%%%%%%%%%%%
\subsection{Test Question} \label{Test Question}
\begin{table}[H]
	\footnotesize
	\begin{tabularx}{\textwidth}{|X|X|c|p{7.8cm}|}   \hline \rowcolor[rgb]{0.8,0.8,0.8}
		
		\textbf{Propriedade} & \textbf{Tipo} & \textbf{Obrigatório?} & \centerline{\textbf{Descrição}} \\\hline  	
		
		question & Texto & x & Pergunta da enquete que deve ser respondida. \\\hline		
		type & Texto & x & Representa se a questão é objetiva ou discursiva. \\\hline		
		value & Número & x & Nota máxima para a questão. \\\hline		
		alternatives & Test Alternative & {} & No caso de questões objetivas, representa as alternativas possíveis para a questão. \\\hline
		responses & Test TextResponse & {} & No caso de questões discursivas, representa a resposta que o aluno deu para a questão. \\\hline
		
	\end{tabularx}	
\end{table}

%%%%%       CLASSE TEST ALTERNATIVE      %%%%%%%%%%%%%%%%%%%%%%%%%%%%%%%%%%%%%%%%%%
\subsection{Test Alternative} \label{Test Alternative}
\begin{table}[H]
	\footnotesize
	\begin{tabularx}{\textwidth}{|X|X|c|p{7.8cm}|}   \hline \rowcolor[rgb]{0.8,0.8,0.8}
		
		\textbf{Propriedade} & \textbf{Tipo} & \textbf{Obrigatório?} & \centerline{\textbf{Descrição}} \\\hline  	
		
		content & Texto & x & Resposta da alternativa para a questão. \\\hline		
		correct & Booleano & x & Representa se esta é a alternativa correta para a questão.  \\\hline				
		question & Test Question & x & Representa a questão à qual a alternativa está vinculada. \\\hline
		responses & Test AlternativeResponse & {} & Associação com o usuário que selecionou a alternativa. \\\hline
		
	\end{tabularx}	
\end{table}


%%%%%       CLASSE ASSIGNMENT        %%%%%%%%%%%%%%%%%%%%%%%%%%%%%%%%%%%%%%%%%%
\subsection{Assignment} \label{Assignment}
\begin{table}[H]
	\footnotesize
	\begin{tabularx}{\textwidth}{|X|X|c|p{7.8cm}|}   \hline \rowcolor[rgb]{0.8,0.8,0.8}
		
		\textbf{Propriedade} & \textbf{Tipo} & \textbf{Obrigatório?} & \centerline{\textbf{Descrição}} \\\hline  	
		
		type & Texto & x & Representa se a tarefa é do tipo Texto, Arquivo ou Código \\\hline			
		start & Data & x & Timestamp do momento em que a tarefa deve ser iniciada. \\\hline			
		end & Data & x & Timestamp do momento em que a tarefa deve ser encerrada. \\\hline			
		file\_limit & Número & {} & Número máximo de arquivos que poderão ser enviados para uma tarefa do tipo Arquivo. \\\hline	
		category & Grade Category & x & Categoria de nota à qual a tarefa está associada. \\\hline	
		submissions & Submission & {} & Submissões realizadas naquela tarefa. \\\hline	
		
	\end{tabularx}	
\end{table}







%%%%%       CLASSE EXTERNAL ACTIVITY        %%%%%%%%%%%%%%%%%%%%%%%%%%%%%%%%%%%%%%%%%%
\subsection{External Activity} \label{External Activity}
\begin{table}[H]
	\footnotesize
	\begin{tabularx}{\textwidth}{|X|X|c|p{7.8cm}|}   \hline \rowcolor[rgb]{0.8,0.8,0.8}
		
		\textbf{Propriedade} & \textbf{Tipo} & \textbf{Obrigatório?} & \centerline{\textbf{Descrição}} \\\hline  	
		
		activities & Activity & {} & Atividades que os alunos realizaram. \\\hline
		category & Grade Category & x & Categoria de nota à qual a atividade externa está associada.\\\hline			
		
	\end{tabularx}	
\end{table}


\newpage

\section{Subsistema Board Interactions}

%%%%%       CLASSE SURVEY RESPONSE     %%%%%%%%%%%%%%%%%%%%%%%%%%%%%%%%%%%%%%%%%%
\subsection{Survey Response} \label{Survey Response}
\begin{table}[H]
	\footnotesize
	\begin{tabularx}{\textwidth}{|X|X|c|p{7.8cm}|}   \hline \rowcolor[rgb]{0.8,0.8,0.8}
		
		\textbf{Propriedade} & \textbf{Tipo} & \textbf{Obrigatório?} & \centerline{\textbf{Descrição}} \\\hline  	
		
		user & User & x & Usuário que respondeu ao questionário. \\\hline
		answer & Survey Answer & x & Alternativa selecionada pelo usuário. \\\hline		
		
	\end{tabularx}	
\end{table}

%%%%%       CLASSE TEST ALTERNATIVE RESPONSE      %%%%%%%%%%%%%%%%%%%%%%%%%%%%%%%%%%%%%%%%%%
\subsection{Test AlternativeResponse} \label{Test AlternativeResponse}
\begin{table}[H]
	\footnotesize
	\begin{tabularx}{\textwidth}{|X|X|c|p{7.8cm}|}   \hline \rowcolor[rgb]{0.8,0.8,0.8}
		
		\textbf{Propriedade} & \textbf{Tipo} & \textbf{Obrigatório?} & \centerline{\textbf{Descrição}} \\\hline  	
		
		alternative & Test Alternative & x & Alternativa selecionada pelo usuário. \\\hline
		user & User & x & Usuário que respondeu à questão. \\\hline
		feedback & Text Feedback & {} & Feedback concedido à questão no momento da avaliação. \\\hline
		
	\end{tabularx}	
\end{table}

%%%%%       CLASSE TEST TEXTRESPONSE      %%%%%%%%%%%%%%%%%%%%%%%%%%%%%%%%%%%%%%%%%%
\subsection{Test TextResponse} \label{Test TextResponse}
\begin{table}[H]
	\footnotesize
	\begin{tabularx}{\textwidth}{|X|X|c|p{7.8cm}|}   \hline \rowcolor[rgb]{0.8,0.8,0.8}
		
		\textbf{Propriedade} & \textbf{Tipo} & \textbf{Obrigatório?} & \centerline{\textbf{Descrição}} \\\hline  	
		
		response & Texto & x & Resposta para a questão. \\\hline		
		grade & Número & {} & Nota atribuída à resposta. \\\hline	
		question & Test Question & x & Questão para a qual a resposta foi dada.\\\hline
		user & User & x & Usuário que respondeu à questão. \\\hline
		feedback & Text Feedback & {} & Feedback concedido à questão no momento da avaliação. \\\hline			
		
	\end{tabularx}	
\end{table}


%%%%%       CLASSE SUBMISSION        %%%%%%%%%%%%%%%%%%%%%%%%%%%%%%%%%%%%%%%%%%
\subsection{Submission} \label{Submission}
\begin{table}[H]
	\footnotesize
	\begin{tabularx}{\textwidth}{|X|X|c|p{7.8cm}|}   \hline \rowcolor[rgb]{0.8,0.8,0.8}
		
		\textbf{Propriedade} & \textbf{Tipo} & \textbf{Obrigatório?} & \centerline{\textbf{Descrição}} \\\hline  	
		
		grade & Número & {} & Nota atribuída à submissão. \\\hline	
		assignment & Assignment & x & Tarefa na qual a submissão foi realizada. \\\hline
		user & User & x & Usuário que realizou a submissão da tarefa \\\hline
		feedback & Text Feedback & {} & Feedback dado à tarefa no momento da avaliação. \\\hline			
		
	\end{tabularx}	
\end{table}


%%%%%       CLASSE TEXT SUBMISSION        %%%%%%%%%%%%%%%%%%%%%%%%%%%%%%%%%%%%%%%%%%
\subsection{Text Submission} \label{Text Submission}
\begin{table}[H]
	\footnotesize
	\begin{tabularx}{\textwidth}{|X|X|c|p{7.8cm}|}   \hline \rowcolor[rgb]{0.8,0.8,0.8}
		
		\textbf{Propriedade} & \textbf{Tipo} & \textbf{Obrigatório?} & \centerline{\textbf{Descrição}} \\\hline  	
		
		content & Texto & x & Conteúdo (texto) da submissão. \\\hline				
		
	\end{tabularx}	
\end{table}


%%%%%       CLASSE FILE SUBMISSION        %%%%%%%%%%%%%%%%%%%%%%%%%%%%%%%%%%%%%%%%%%
\subsection{File Submission} \label{File Submission}
\begin{table}[H]
	\footnotesize
	\begin{tabularx}{\textwidth}{|X|X|c|p{7.8cm}|}   \hline \rowcolor[rgb]{0.8,0.8,0.8}
		
		\textbf{Propriedade} & \textbf{Tipo} & \textbf{Obrigatório?} & \centerline{\textbf{Descrição}} \\\hline  	
		
		file & Arquivo & x & Arquivo submetido para a tarefa. \\\hline				
		
	\end{tabularx}	
\end{table}


%%%%%       CLASSE CODE SUBMISSION        %%%%%%%%%%%%%%%%%%%%%%%%%%%%%%%%%%%%%%%%%%
\subsection{Code Submission} \label{Code Submission}
\begin{table}[H]
	\footnotesize
	\begin{tabularx}{\textwidth}{|X|X|c|p{7.8cm}|}   \hline \rowcolor[rgb]{0.8,0.8,0.8}
		
		\textbf{Propriedade} & \textbf{Tipo} & \textbf{Obrigatório?} & \centerline{\textbf{Descrição}} \\\hline  	
		
		language & Texto & x & Linguagem de programação do código submetido \\\hline	
		lines & Code Line & x & Linhas do código submetido na tarefa. \\\hline			
		
	\end{tabularx}	
\end{table}


%%%%%       CLASSE CODE LINE        %%%%%%%%%%%%%%%%%%%%%%%%%%%%%%%%%%%%%%%%%%
\subsection{Code Line} \label{Code Line}
\begin{table}[H]
	\footnotesize
	\begin{tabularx}{\textwidth}{|X|X|c|p{7.8cm}|}   \hline \rowcolor[rgb]{0.8,0.8,0.8}
		
		\textbf{Propriedade} & \textbf{Tipo} & \textbf{Obrigatório?} & \centerline{\textbf{Descrição}} \\\hline  	
		
		line\_number & Número & x & Posição da linha no código \\\hline				
		content & Texto & x & Conteúdo da linha de código \\\hline	
		submission & Code Submission & x & Código ao qual a linha pertence. \\\hline
		feedback & Code Line Feedback & {} & Feedback dado àquela linha de código no momento da avaliação. \\\hline			
		
	\end{tabularx}	
\end{table}


%%%%%       CLASSE ACTIVITY        %%%%%%%%%%%%%%%%%%%%%%%%%%%%%%%%%%%%%%%%%%
\subsection{Activity} \label{Activity}
\begin{table}[H]
	\footnotesize
	\begin{tabularx}{\textwidth}{|X|X|c|p{7.8cm}|}   \hline \rowcolor[rgb]{0.8,0.8,0.8}
		
		\textbf{Propriedade} & \textbf{Tipo} & \textbf{Obrigatório?} & \centerline{\textbf{Descrição}} \\\hline  	
		
		grade & Número & x & Nota atribuída àquela atividade. \\\hline	
		activity & External Activity & x & Atividade externa que foi realizada. \\\hline			
		user & User & x & Aluno que realizou a atividade externa. \\\hline			
		feedback & Text Feedback & {} & Feedback atribuído à atividade no momento da avaliação. \\\hline			
		
	\end{tabularx}	
\end{table}
\newpage

\section{Subsistema Feedback}

%%%%%       CLASSE CODE LINE FEEDBACK        %%%%%%%%%%%%%%%%%%%%%%%%%%%%%%%%%%%%%%%%%%
\subsection{Code Line Feedback} \label{Code Line Feedback}
\begin{table}[H]
	\footnotesize
	\begin{tabularx}{\textwidth}{|X|X|c|p{7.8cm}|}   \hline \rowcolor[rgb]{0.8,0.8,0.8}
		
		\textbf{Propriedade} & \textbf{Tipo} & \textbf{Obrigatório?} & \centerline{\textbf{Descrição}} \\\hline  	
		
		feedback & Texto & x & Comentário/Feedback dado a uma linha de código específica. \\\hline	
		line & Code Line & x & Linha de código à qual o feedback foi atribuído. \\\hline			
		
	\end{tabularx}	
\end{table}


%%%%%       CLASSE TEXT FEEDBACK        %%%%%%%%%%%%%%%%%%%%%%%%%%%%%%%%%%%%%%%%%%
\subsection{Text Feedback} \label{Text Feedback}
\begin{table}[H]
	\footnotesize
	\noindent
	\begin{tabularx}{\textwidth}{|X|X|c|p{7.8cm}|}   \hline \rowcolor[rgb]{0.8,0.8,0.8}
		
		\textbf{Propriedade} & \textbf{Tipo} & \textbf{Obrigatório?} & \centerline{\textbf{Descrição}} \\\hline  	
		
		feedback & Texto & x & Comentário/Feedback dado a uma submissão em formato de texto. \\\hline	
		alternative & Test Alternative Response & {} & Questão objetiva de prova à qual o feedback foi atribuído. \\\hline	
		response & Test TextResponse & {} & Questão discursiva de prova à qual o feedback foi atribuído. \\\hline				
		submission & Submission & {} & Submissão de tarefa à qual o feedback foi atribuído. \\\hline	
		activity & Activity & {} & Atividade externa à qual o feedback foi atribuído. \\\hline	
	\end{tabularx}	
\end{table}

\newpage

\section{Subsistema Discussion}
%%%%%       CLASSE NEWS        %%%%%%%%%%%%%%%%%%%%%%%%%%%%%%%%%%%%%%%%%%
\subsection{News} \label{News}
\begin{table}[H]
	\footnotesize
	\begin{tabularx}{\textwidth}{|X|X|c|p{7.8cm}|}   \hline \rowcolor[rgb]{0.8,0.8,0.8}
		
		\textbf{Propriedade} & \textbf{Tipo} & \textbf{Obrigatório?} & \centerline{\textbf{Descrição}} \\\hline  	
		
		title & Texto & x & Título da notícia. \\\hline		
		content & Texto & x & Conteúdo da notícia \\\hline		
		created\_at & Data & x & Timestamp da data em que a notícia foi criada. \\\hline
		author & User & x & Usuário que publicou a notícia. \\\hline
		classroom & Classroom & x & Turma na qual a notícia foi publicada. \\\hline	
		
	\end{tabularx}	
\end{table}

%%%%%       CLASSE FORUM TOPIC        %%%%%%%%%%%%%%%%%%%%%%%%%%%%%%%%%%%%%%%%%%
\subsection{Forum Topic} \label{Forum Topic}
\begin{table}[H]
	\footnotesize
	\begin{tabularx}{\textwidth}{|X|X|c|p{7.8cm}|}   \hline \rowcolor[rgb]{0.8,0.8,0.8}
		
		\textbf{Propriedade} & \textbf{Tipo} & \textbf{Obrigatório?} & \centerline{\textbf{Descrição}} \\\hline  	
		
		title & Texto & x & Título do tópico. \\\hline		
		content & Texto & x & Conteúdo do tópico \\\hline		
		created\_at & Data & x & Timestamp da data em que o tópico foi criado. \\\hline	
		author & User & x & Usuário que publicou o tópico. \\\hline
		classroom & Classroom & x & Turma na qual o tópico foi publicado. \\\hline
		replies & Forum Reply & {} & Réplicas criadas para o tópico. \\\hline
		
	\end{tabularx}	
\end{table}

%%%%%       CLASSE FORUM REPLY        %%%%%%%%%%%%%%%%%%%%%%%%%%%%%%%%%%%%%%%%%%
\subsection{Forum Reply} \label{Forum Reply}
\begin{table}[H]
	\footnotesize
	\begin{tabularx}{\textwidth}{|X|X|c|p{7.8cm}|}   \hline \rowcolor[rgb]{0.8,0.8,0.8}
		
		\textbf{Propriedade} & \textbf{Tipo} & \textbf{Obrigatório?} & \centerline{\textbf{Descrição}} \\\hline  	
		
		content & Texto & x & Conteúdo da resposta \\\hline		
		created\_at & Data & x & Timestamp da data em que a resposta foi criada. \\\hline	
		author & User & x & Usuário que publicou a réplica. \\\hline
		classroom & Classroom & x & Turma na qual a réplica foi publicada. \\\hline
		topic & Forum Topic & x & Tópico no qual a réplica foi publicada. \\\hline
		
	\end{tabularx}	
\end{table}

%%%%%       CLASSE LIVE QUESTION       %%%%%%%%%%%%%%%%%%%%%%%%%%%%%%%%%%%%%%%%%%
\subsection{Live Question} \label{Live Question}
\begin{table}[H]
	\footnotesize
	\begin{tabularx}{\textwidth}{|X|X|c|p{7.8cm}|}   \hline \rowcolor[rgb]{0.8,0.8,0.8}
		
		\textbf{Propriedade} & \textbf{Tipo} & \textbf{Obrigatório?} & \centerline{\textbf{Descrição}} \\\hline  	
		
		question & Texto & x & Dúvida do usuário. \\\hline		
		created\_at & Data & x &  Timestamp do momento em que a dúvida foi criada. \\\hline		
		classroom & Classroom & x &  Turma na qual a dúvida foi feita. \\\hline		
		user & User & {} &  Usuário que criou a dúvida. \\\hline		
		
	\end{tabularx}	
\end{table}

\newpage

\section{Subsistema Log}

%%%%%       CLASSE LOG        %%%%%%%%%%%%%%%%%%%%%%%%%%%%%%%%%%%%%%%%%%
\subsection{Log} \label{Log}
\begin{table}[H]
	\footnotesize
	\begin{tabularx}{\textwidth}{|X|X|c|p{7.8cm}|}   \hline \rowcolor[rgb]{0.8,0.8,0.8}
		
		\textbf{Propriedade} & \textbf{Tipo} & \textbf{Obrigatório?} & \centerline{\textbf{Descrição}} \\\hline  	
		
		content & Texto & x & Descrição do log. \\\hline		
		category & Texto & x & Indica o tipo de ação a que o log se refere. \\\hline
		created\_at & Data & x & Timestamp do momento em que o log foi criado. \\\hline
		user & User & x & Usuário que realizou a ação que deu origem ao log \\\hline	
		
	\end{tabularx}	
\end{table}


% Finaliza a parte no bookmark do PDF para que se inicie o bookmark na raiz e adiciona espaço de parte no sumário.
\phantompart

% Marca o início dos elementos pós-textuais.
\postextual

\phantompart
\printindex

% Fim do documento.
\end{document}
