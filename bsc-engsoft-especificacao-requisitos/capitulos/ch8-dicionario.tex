\chapter{Dicionário de Projeto}
\label{sec-dicionario}
\setcounter{table}{0}

Esta seção apresenta as definições das classes (e seus atributos), servindo como um glossário do projeto. As definições são organizadas por subsistema. Vale destacar que eventuais operações que estas classes vierem a ter não são listadas e descritas nesta fase do projeto. Além disse, algumas classes estão envolvidas em mais de um subsistema. Nesses casos, a classe será descrita apenas uma vez.


\section{Subsistema Classroom}

\subsection{User} \label{User}
\begin{table}[H]
	\footnotesize
	\begin{tabularx}{\textwidth}{|p{2.6cm}|X|c|p{7.8cm}|}   \hline \rowcolor[rgb]{0.8,0.8,0.8}
		
		\textbf{Propriedade} & \textbf{Tipo} & \textbf{Obrigatório?} & \centerline{\textbf{Descrição}} \\\hline  	
		
		email & Texto & x & Email do usuário utilizado para realizar login. \\\hline	
		photo & Arquivo & {} & Arquivo referente à foto do usuário. \\\hline	
		fullname & Texto & x & Nome completo do usuário. \\\hline	
		cep & Texto & x & CEP da residência do usuário. \\\hline	
		address & Texto & x & Logradouro da residência do usuário. \\\hline	
		number & Número & x & Número da residência do usuário. \\\hline	
		complement & Texto & {} & Complemento da residência do usuário. \\\hline	
		neighborhood & Texto & x & Bairro do usuário. \\\hline	
		city & Texto & x & Cidade do usuário. \\\hline	
		state & Texto & x & Estado do usuário. \\\hline	
		country & Texto & x & País do usuário. \\\hline	
		
	\end{tabularx}	
\end{table}


%%%%%       CLASSE CLASSROOM        %%%%%%%%%%%%%%%%%%%%%%%%%%%%%%%%%%%%%%%%%%
\subsection{Classroom} \label{Classroom}
\begin{table}[H]
	\footnotesize
	\begin{tabularx}{\textwidth}{|p{2.6cm}|X|c|p{7.8cm}|}   \hline \rowcolor[rgb]{0.8,0.8,0.8}
		
		\textbf{Propriedade} & \textbf{Tipo} & \textbf{Obrigatório?} & \centerline{\textbf{Descrição}} \\\hline  	
		
		name & Texto & x & Nome da turma. \\\hline		
		password & Texto & {} & Senha de acesso para a turma. \\\hline	
		has\_grades & Booleano & x & Indica se a turma tem livro de notas. \\\hline	
		has\_attendance & Booleano & x & Indica se a turma tem lista de presença. \\ \hline	
		minimum\_grade & Número & x & Nota mínima exigida para a aprovação dos alunos. \\\hline
		users & User & {} & Usuários inscritos na turma. \\\hline
		grade\_categories & Grade Category & {} & Categorias de nota associadas à turma para cálculo da média final dos alunos. \\\hline
		events & Event & {} & Eventos adicionados ao calendário da turma \\\hline
		classes & Class & {} & Aulas adicionadas ao calendário da turma \\\hline	
		
	\end{tabularx}	
\end{table}

\subsection{Subscription} \label{Subscription}
\begin{table}[H]
	\footnotesize
	\begin{tabularx}{\textwidth}{|X|X|c|p{7.8cm}|}   \hline \rowcolor[rgb]{0.8,0.8,0.8}
		
		\textbf{Propriedade} & \textbf{Tipo} & \textbf{Obrigatório?} & \centerline{\textbf{Descrição}} \\\hline  	
		
		role & Texto & x & Papel atribuído ao usuário naquela turma (Professor ou Aluno). \\\hline
		user & User & x & Usuário que está se inscrevendo na turma. \\\hline
		classroom & Classroom & x & Turma em que o usuário está se inscrevendo. \\\hline		
		
	\end{tabularx}	
\end{table}


%%%%%       CLASSE GRADECATEGORY        %%%%%%%%%%%%%%%%%%%%%%%%%%%%%%%%%%%%%%%%%%
\subsection{Grade Category} \label{Grade Category}
\begin{table}[H]
	\footnotesize
	\begin{tabularx}{\textwidth}{|X|X|c|p{7.8cm}|}   \hline \rowcolor[rgb]{0.8,0.8,0.8}
		
		\textbf{Propriedade} & \textbf{Tipo} & \textbf{Obrigatório?} & \centerline{\textbf{Descrição}} \\\hline  	
		
		title & Texto & x & Nome da categoria. \\\hline		
		weight & Número & x & Peso da categoria no cálculo da média. \\\hline	
		classroom & Classroom & x & Turma que utiliza a categoria de nota para cálculo da média. \\\hline
		
	\end{tabularx}	
\end{table}


%%%%%       CLASSE MESSAGE        %%%%%%%%%%%%%%%%%%%%%%%%%%%%%%%%%%%%%%%%%%
\subsection{Message} \label{Message}
\begin{table}[H]
	\footnotesize
	\begin{tabularx}{\textwidth}{|X|X|c|p{7.8cm}|}   \hline \rowcolor[rgb]{0.8,0.8,0.8}
		
		\textbf{Propriedade} & \textbf{Tipo} & \textbf{Obrigatório?} & \centerline{\textbf{Descrição}} \\\hline  	
		
		content & Texto & x & Conteúdo da mensagem. \\\hline		
		created\_at & Data & x & Timestamp do momento em que a mensagem foi criado. \\\hline
		read & Booleano & x & Indica se a mensagem já foi lida. \\\hline
		sent\_by & User & x & Usuário que enviou a mensagem. \\\hline
		received\_by & User & x & Usuários que receberam a mensagem. \\\hline
		
	\end{tabularx}	
\end{table}

%%%%%       CLASSE EVENT        %%%%%%%%%%%%%%%%%%%%%%%%%%%%%%%%%%%%%%%%%%
\subsection{Event} \label{Event}
\begin{table}[H]
	\footnotesize
	\begin{tabularx}{\textwidth}{|X|X|c|p{7.8cm}|}   \hline \rowcolor[rgb]{0.8,0.8,0.8}
		
		\textbf{Propriedade} & \textbf{Tipo} & \textbf{Obrigatório?} & \centerline{\textbf{Descrição}} \\\hline  	
		
		title & Texto & x & Nome do evento. \\\hline		
		description & Texto & {} & Descrição do evento \\\hline		
		date & Data & x & Timestamp da data em que o evento ocorrerá. \\\hline	
		classroom & Classroom & x & Turma na qual o evento foi adicionado ao calendário. \\\hline
		
	\end{tabularx}	
\end{table}

%%%%%       CLASSE CLASS        %%%%%%%%%%%%%%%%%%%%%%%%%%%%%%%%%%%%%%%%%%
\subsection{Class} \label{Class}
\begin{table}[H]
	\footnotesize
	\begin{tabularx}{\textwidth}{|X|X|c|p{7.8cm}|}   \hline \rowcolor[rgb]{0.8,0.8,0.8}
		
		\textbf{Propriedade} & \textbf{Tipo} & \textbf{Obrigatório?} & \centerline{\textbf{Descrição}} \\\hline  	
		
		title & Texto & {} & Nome associado à aula. \\\hline		
		description & Texto & {} & Descrição da aula \\\hline		
		date & Data & x & Timestamp da data em que a aula ocorrerá. \\\hline
		classroom & Classroom & x & Turma na qual a aula foi adicionada ao calendário. \\\hline
		students & User & {} & Alunos que compareceram (ou não) à aula. \\\hline	
		
	\end{tabularx}	
\end{table}


%%%%%       CLASSE PRESENCE        %%%%%%%%%%%%%%%%%%%%%%%%%%%%%%%%%%%%%%%%%%
\subsection{Presence} \label{Presence}
\begin{table}[H]
	\footnotesize
	\begin{tabularx}{\textwidth}{|X|X|c|p{7.8cm}|}   \hline \rowcolor[rgb]{0.8,0.8,0.8}
		
		\textbf{Propriedade} & \textbf{Tipo} & \textbf{Obrigatório?} & \centerline{\textbf{Descrição}} \\\hline  	
		
		present & Booleano & x & Indica se o aluno esteve presente ou não à aula correspondente. \\\hline	
		class & Class & x & Aula à qual o aluno esteve presente (ou não). \\\hline
		user & User & x & Aluno do qual a presença se trata. \\\hline
		
	\end{tabularx}	
\end{table}




%%%%%       CLASSE CONFIGURATION        %%%%%%%%%%%%%%%%%%%%%%%%%%%%%%%%%%%%%%%%%%
\subsection{Configuration} \label{Configuration}
\begin{table}[H]
	\footnotesize
	\begin{tabularx}{\textwidth}{|X|X|c|p{7.8cm}|}   \hline \rowcolor[rgb]{0.8,0.8,0.8}
		
		\textbf{Propriedade} & \textbf{Tipo} & \textbf{Obrigatório?} & \centerline{\textbf{Descrição}} \\\hline  	
		
		name & Texto & x & Nome da configuração \\\hline		
		value & Data & x & Valor associado à configuração \\\hline	
		
	\end{tabularx}	
\end{table}

\newpage

\section{Subsistema Board}

%%%%%       CLASSE SECTION        %%%%%%%%%%%%%%%%%%%%%%%%%%%%%%%%%%%%%%%%%%
\subsection{Section} \label{Section}
\begin{table}[H]
	\footnotesize
	\begin{tabularx}{\textwidth}{|X|X|c|p{7.8cm}|}   \hline \rowcolor[rgb]{0.8,0.8,0.8}
		
		\textbf{Propriedade} & \textbf{Tipo} & \textbf{Obrigatório?} & \centerline{\textbf{Descrição}} \\\hline  	
		
		title & Texto & {} & Título da seção. \\\hline		
		description & Texto & {} & Descrição da seção. \\\hline		
		position & Número & x & Posição da seção no board. \\\hline		
		created\_at & Data & x & Timestamp do momento em que a seção foi criada. \\\hline	
		classroom & Classroom & x & Turma à qual a seção está vinculada. \\\hline
		items & Board Item & {} & Items do board vinculados à seção. \\\hline
		
	\end{tabularx}	
\end{table}


%%%%%       CLASSE BOARDITEM        %%%%%%%%%%%%%%%%%%%%%%%%%%%%%%%%%%%%%%%%%%
\subsection{BoardItem} \label{BoardItem}
\begin{table}[H]
	\footnotesize
	\begin{tabularx}{\textwidth}{|X|X|c|p{7.8cm}|}   \hline \rowcolor[rgb]{0.8,0.8,0.8}
		
		\textbf{Propriedade} & \textbf{Tipo} & \textbf{Obrigatório?} & \centerline{\textbf{Descrição}} \\\hline  	
		
		title & Texto & x & Título do item. \\\hline		
		description & Texto & {} & Descrição do item. \\\hline		
		position & Número & x & Posição do item na seção. \\\hline		
		created\_at & Data & x & Timestamp do momento em que o item foi criada. \\\hline	
		section & Section & x & Seção à qual o item está vinculado. \\\hline
		
	\end{tabularx}	
\end{table}


%%%%%       CLASSE FILE        %%%%%%%%%%%%%%%%%%%%%%%%%%%%%%%%%%%%%%%%%%
\subsection{File} \label{File}
\begin{table}[H]
	\footnotesize
	\begin{tabularx}{\textwidth}{|X|X|c|p{7.8cm}|}   \hline \rowcolor[rgb]{0.8,0.8,0.8}
		
		\textbf{Propriedade} & \textbf{Tipo} & \textbf{Obrigatório?} & \centerline{\textbf{Descrição}} \\\hline  	
		
		file & Arquivo & x & Arquivo a ser baixado. \\\hline			
		
	\end{tabularx}	
\end{table}



%%%%%       CLASSE LINK        %%%%%%%%%%%%%%%%%%%%%%%%%%%%%%%%%%%%%%%%%%
\subsection{Link} \label{Link}
\begin{table}[H]
	\footnotesize
	\begin{tabularx}{\textwidth}{|X|X|c|p{7.8cm}|}   \hline \rowcolor[rgb]{0.8,0.8,0.8}
		
		\textbf{Propriedade} & \textbf{Tipo} & \textbf{Obrigatório?} & \centerline{\textbf{Descrição}} \\\hline  	
		
		url & Texto & x & URL da página a ser acessada. \\\hline			
		
	\end{tabularx}	
\end{table}



%%%%%       CLASSE SURVEY        %%%%%%%%%%%%%%%%%%%%%%%%%%%%%%%%%%%%%%%%%%
\subsection{Survey} \label{Survey}
\begin{table}[H]
	\footnotesize
	\begin{tabularx}{\textwidth}{|X|X|c|p{7.8cm}|}   \hline \rowcolor[rgb]{0.8,0.8,0.8}
		
		\textbf{Propriedade} & \textbf{Tipo} & \textbf{Obrigatório?} & \centerline{\textbf{Descrição}} \\\hline  	
		
		start & Data & x & Timestamp do momento em que o questionário deve ser iniciado. \\\hline		
		end & Data & x & Timestamp do momento em que o questionário deve ser encerrado. \\\hline
		questions & Survey Question & x & Questões do questionário \\\hline			
		
	\end{tabularx}	
\end{table}


%%%%%       CLASSE SURVEY QUESTION       %%%%%%%%%%%%%%%%%%%%%%%%%%%%%%%%%%%%%%%%%%
\subsection{Survey Question} \label{Survey Question}
\begin{table}[H]
	\footnotesize
	\begin{tabularx}{\textwidth}{|X|X|c|p{7.8cm}|}   \hline \rowcolor[rgb]{0.8,0.8,0.8}
		
		\textbf{Propriedade} & \textbf{Tipo} & \textbf{Obrigatório?} & \centerline{\textbf{Descrição}} \\\hline  	
		
		question & Texto & x & Pergunta do questionário que deve ser respondida. \\\hline			
		required & Booleano & x & Indica se a questão é obrigatória ou não. \\\hline	
		survey & Survey & x & Questionário ao qual a pergunta está vinculada. \\\hline
		answers & Survey Answer & x & Alternativas possíveis para a questão. \\\hline	
		
	\end{tabularx}	
\end{table}


%%%%%       CLASSE SURVEY ANSWER       %%%%%%%%%%%%%%%%%%%%%%%%%%%%%%%%%%%%%%%%%%
\subsection{Survey Answer} \label{Survey Answer}
\begin{table}[H]
	\footnotesize
	\begin{tabularx}{\textwidth}{|X|X|c|p{7.8cm}|}   \hline \rowcolor[rgb]{0.8,0.8,0.8}
		
		\textbf{Propriedade} & \textbf{Tipo} & \textbf{Obrigatório?} & \centerline{\textbf{Descrição}} \\\hline  	
		
		answer & Texto & x & Resposta para uma pergunta do questionário. \\\hline
		question & Survey Question & x & Pergunta do questionário à qual a resposta está vinculada. \\\hline
		responses & Survey Response & x	& Associação com o usuário que selecionou a opção. \\\hline 		
		
	\end{tabularx}	
\end{table}


%%%%%       CLASSE TEST       %%%%%%%%%%%%%%%%%%%%%%%%%%%%%%%%%%%%%%%%%%
\subsection{Test} \label{Test}
\begin{table}[H]
	\footnotesize
	\begin{tabularx}{\textwidth}{|X|X|c|p{7.8cm}|}   \hline \rowcolor[rgb]{0.8,0.8,0.8}
		
		\textbf{Propriedade} & \textbf{Tipo} & \textbf{Obrigatório?} & \centerline{\textbf{Descrição}} \\\hline  	
		
		start & Data & x & Timestamp do momento em que a prova deve ser iniciada. \\\hline			
		end & Data & x & Timestamp do momento em que a prova deve ser encerrada. \\\hline
		questions & Test Question & x & Questões da prova.			\\\hline
		category & Grade Category & x & Categoria de nota à qual a prova está associada. \\\hline
		
		
	\end{tabularx}	
\end{table}



%%%%%       CLASSE TEST QUESTION      %%%%%%%%%%%%%%%%%%%%%%%%%%%%%%%%%%%%%%%%%%
\subsection{Test Question} \label{Test Question}
\begin{table}[H]
	\footnotesize
	\begin{tabularx}{\textwidth}{|X|X|c|p{7.8cm}|}   \hline \rowcolor[rgb]{0.8,0.8,0.8}
		
		\textbf{Propriedade} & \textbf{Tipo} & \textbf{Obrigatório?} & \centerline{\textbf{Descrição}} \\\hline  	
		
		question & Texto & x & Pergunta da enquete que deve ser respondida. \\\hline		
		type & Texto & x & Representa se a questão é objetiva ou discursiva. \\\hline		
		value & Número & x & Nota máxima para a questão. \\\hline		
		alternatives & Test Alternative & {} & No caso de questões objetivas, representa as alternativas possíveis para a questão. \\\hline
		responses & Test TextResponse & {} & No caso de questões discursivas, representa a resposta que o aluno deu para a questão. \\\hline
		
	\end{tabularx}	
\end{table}

%%%%%       CLASSE TEST ALTERNATIVE      %%%%%%%%%%%%%%%%%%%%%%%%%%%%%%%%%%%%%%%%%%
\subsection{Test Alternative} \label{Test Alternative}
\begin{table}[H]
	\footnotesize
	\begin{tabularx}{\textwidth}{|X|X|c|p{7.8cm}|}   \hline \rowcolor[rgb]{0.8,0.8,0.8}
		
		\textbf{Propriedade} & \textbf{Tipo} & \textbf{Obrigatório?} & \centerline{\textbf{Descrição}} \\\hline  	
		
		content & Texto & x & Resposta da alternativa para a questão. \\\hline		
		correct & Booleano & x & Representa se esta é a alternativa correta para a questão.  \\\hline				
		question & Test Question & x & Representa a questão à qual a alternativa está vinculada. \\\hline
		responses & Test AlternativeResponse & {} & Associação com o usuário que selecionou a alternativa. \\\hline
		
	\end{tabularx}	
\end{table}


%%%%%       CLASSE ASSIGNMENT        %%%%%%%%%%%%%%%%%%%%%%%%%%%%%%%%%%%%%%%%%%
\subsection{Assignment} \label{Assignment}
\begin{table}[H]
	\footnotesize
	\begin{tabularx}{\textwidth}{|X|X|c|p{7.8cm}|}   \hline \rowcolor[rgb]{0.8,0.8,0.8}
		
		\textbf{Propriedade} & \textbf{Tipo} & \textbf{Obrigatório?} & \centerline{\textbf{Descrição}} \\\hline  	
		
		type & Texto & x & Representa se a tarefa é do tipo Texto, Arquivo ou Código \\\hline			
		start & Data & x & Timestamp do momento em que a tarefa deve ser iniciada. \\\hline			
		end & Data & x & Timestamp do momento em que a tarefa deve ser encerrada. \\\hline			
		file\_limit & Número & {} & Número máximo de arquivos que poderão ser enviados para uma tarefa do tipo Arquivo. \\\hline	
		category & Grade Category & x & Categoria de nota à qual a tarefa está associada. \\\hline	
		submissions & Submission & {} & Submissões realizadas naquela tarefa. \\\hline	
		
	\end{tabularx}	
\end{table}







%%%%%       CLASSE EXTERNAL ACTIVITY        %%%%%%%%%%%%%%%%%%%%%%%%%%%%%%%%%%%%%%%%%%
\subsection{External Activity} \label{External Activity}
\begin{table}[H]
	\footnotesize
	\begin{tabularx}{\textwidth}{|X|X|c|p{7.8cm}|}   \hline \rowcolor[rgb]{0.8,0.8,0.8}
		
		\textbf{Propriedade} & \textbf{Tipo} & \textbf{Obrigatório?} & \centerline{\textbf{Descrição}} \\\hline  	
		
		activities & Activity & {} & Atividades que os alunos realizaram. \\\hline
		category & Grade Category & x & Categoria de nota à qual a atividade externa está associada.\\\hline			
		
	\end{tabularx}	
\end{table}


\newpage

\section{Subsistema Board Interactions}

%%%%%       CLASSE SURVEY RESPONSE     %%%%%%%%%%%%%%%%%%%%%%%%%%%%%%%%%%%%%%%%%%
\subsection{Survey Response} \label{Survey Response}
\begin{table}[H]
	\footnotesize
	\begin{tabularx}{\textwidth}{|X|X|c|p{7.8cm}|}   \hline \rowcolor[rgb]{0.8,0.8,0.8}
		
		\textbf{Propriedade} & \textbf{Tipo} & \textbf{Obrigatório?} & \centerline{\textbf{Descrição}} \\\hline  	
		
		user & User & x & Usuário que respondeu ao questionário. \\\hline
		answer & Survey Answer & x & Alternativa selecionada pelo usuário. \\\hline		
		
	\end{tabularx}	
\end{table}

%%%%%       CLASSE TEST ALTERNATIVE RESPONSE      %%%%%%%%%%%%%%%%%%%%%%%%%%%%%%%%%%%%%%%%%%
\subsection{Test AlternativeResponse} \label{Test AlternativeResponse}
\begin{table}[H]
	\footnotesize
	\begin{tabularx}{\textwidth}{|X|X|c|p{7.8cm}|}   \hline \rowcolor[rgb]{0.8,0.8,0.8}
		
		\textbf{Propriedade} & \textbf{Tipo} & \textbf{Obrigatório?} & \centerline{\textbf{Descrição}} \\\hline  	
		
		alternative & Test Alternative & x & Alternativa selecionada pelo usuário. \\\hline
		user & User & x & Usuário que respondeu à questão. \\\hline
		feedback & Text Feedback & {} & Feedback concedido à questão no momento da avaliação. \\\hline
		
	\end{tabularx}	
\end{table}

%%%%%       CLASSE TEST TEXTRESPONSE      %%%%%%%%%%%%%%%%%%%%%%%%%%%%%%%%%%%%%%%%%%
\subsection{Test TextResponse} \label{Test TextResponse}
\begin{table}[H]
	\footnotesize
	\begin{tabularx}{\textwidth}{|X|X|c|p{7.8cm}|}   \hline \rowcolor[rgb]{0.8,0.8,0.8}
		
		\textbf{Propriedade} & \textbf{Tipo} & \textbf{Obrigatório?} & \centerline{\textbf{Descrição}} \\\hline  	
		
		response & Texto & x & Resposta para a questão. \\\hline		
		grade & Número & {} & Nota atribuída à resposta. \\\hline	
		question & Test Question & x & Questão para a qual a resposta foi dada.\\\hline
		user & User & x & Usuário que respondeu à questão. \\\hline
		feedback & Text Feedback & {} & Feedback concedido à questão no momento da avaliação. \\\hline			
		
	\end{tabularx}	
\end{table}


%%%%%       CLASSE SUBMISSION        %%%%%%%%%%%%%%%%%%%%%%%%%%%%%%%%%%%%%%%%%%
\subsection{Submission} \label{Submission}
\begin{table}[H]
	\footnotesize
	\begin{tabularx}{\textwidth}{|X|X|c|p{7.8cm}|}   \hline \rowcolor[rgb]{0.8,0.8,0.8}
		
		\textbf{Propriedade} & \textbf{Tipo} & \textbf{Obrigatório?} & \centerline{\textbf{Descrição}} \\\hline  	
		
		grade & Número & {} & Nota atribuída à submissão. \\\hline	
		assignment & Assignment & x & Tarefa na qual a submissão foi realizada. \\\hline
		user & User & x & Usuário que realizou a submissão da tarefa \\\hline
		feedback & Text Feedback & {} & Feedback dado à tarefa no momento da avaliação. \\\hline			
		
	\end{tabularx}	
\end{table}


%%%%%       CLASSE TEXT SUBMISSION        %%%%%%%%%%%%%%%%%%%%%%%%%%%%%%%%%%%%%%%%%%
\subsection{Text Submission} \label{Text Submission}
\begin{table}[H]
	\footnotesize
	\begin{tabularx}{\textwidth}{|X|X|c|p{7.8cm}|}   \hline \rowcolor[rgb]{0.8,0.8,0.8}
		
		\textbf{Propriedade} & \textbf{Tipo} & \textbf{Obrigatório?} & \centerline{\textbf{Descrição}} \\\hline  	
		
		content & Texto & x & Conteúdo (texto) da submissão. \\\hline				
		
	\end{tabularx}	
\end{table}


%%%%%       CLASSE FILE SUBMISSION        %%%%%%%%%%%%%%%%%%%%%%%%%%%%%%%%%%%%%%%%%%
\subsection{File Submission} \label{File Submission}
\begin{table}[H]
	\footnotesize
	\begin{tabularx}{\textwidth}{|X|X|c|p{7.8cm}|}   \hline \rowcolor[rgb]{0.8,0.8,0.8}
		
		\textbf{Propriedade} & \textbf{Tipo} & \textbf{Obrigatório?} & \centerline{\textbf{Descrição}} \\\hline  	
		
		file & Arquivo & x & Arquivo submetido para a tarefa. \\\hline				
		
	\end{tabularx}	
\end{table}


%%%%%       CLASSE CODE SUBMISSION        %%%%%%%%%%%%%%%%%%%%%%%%%%%%%%%%%%%%%%%%%%
\subsection{Code Submission} \label{Code Submission}
\begin{table}[H]
	\footnotesize
	\begin{tabularx}{\textwidth}{|X|X|c|p{7.8cm}|}   \hline \rowcolor[rgb]{0.8,0.8,0.8}
		
		\textbf{Propriedade} & \textbf{Tipo} & \textbf{Obrigatório?} & \centerline{\textbf{Descrição}} \\\hline  	
		
		language & Texto & x & Linguagem de programação do código submetido \\\hline	
		lines & Code Line & x & Linhas do código submetido na tarefa. \\\hline			
		
	\end{tabularx}	
\end{table}


%%%%%       CLASSE CODE LINE        %%%%%%%%%%%%%%%%%%%%%%%%%%%%%%%%%%%%%%%%%%
\subsection{Code Line} \label{Code Line}
\begin{table}[H]
	\footnotesize
	\begin{tabularx}{\textwidth}{|X|X|c|p{7.8cm}|}   \hline \rowcolor[rgb]{0.8,0.8,0.8}
		
		\textbf{Propriedade} & \textbf{Tipo} & \textbf{Obrigatório?} & \centerline{\textbf{Descrição}} \\\hline  	
		
		line\_number & Número & x & Posição da linha no código \\\hline				
		content & Texto & x & Conteúdo da linha de código \\\hline	
		submission & Code Submission & x & Código ao qual a linha pertence. \\\hline
		feedback & Code Line Feedback & {} & Feedback dado àquela linha de código no momento da avaliação. \\\hline			
		
	\end{tabularx}	
\end{table}


%%%%%       CLASSE ACTIVITY        %%%%%%%%%%%%%%%%%%%%%%%%%%%%%%%%%%%%%%%%%%
\subsection{Activity} \label{Activity}
\begin{table}[H]
	\footnotesize
	\begin{tabularx}{\textwidth}{|X|X|c|p{7.8cm}|}   \hline \rowcolor[rgb]{0.8,0.8,0.8}
		
		\textbf{Propriedade} & \textbf{Tipo} & \textbf{Obrigatório?} & \centerline{\textbf{Descrição}} \\\hline  	
		
		grade & Número & x & Nota atribuída àquela atividade. \\\hline	
		activity & External Activity & x & Atividade externa que foi realizada. \\\hline			
		user & User & x & Aluno que realizou a atividade externa. \\\hline			
		feedback & Text Feedback & {} & Feedback atribuído à atividade no momento da avaliação. \\\hline			
		
	\end{tabularx}	
\end{table}
\newpage

\section{Subsistema Feedback}

%%%%%       CLASSE CODE LINE FEEDBACK        %%%%%%%%%%%%%%%%%%%%%%%%%%%%%%%%%%%%%%%%%%
\subsection{Code Line Feedback} \label{Code Line Feedback}
\begin{table}[H]
	\footnotesize
	\begin{tabularx}{\textwidth}{|X|X|c|p{7.8cm}|}   \hline \rowcolor[rgb]{0.8,0.8,0.8}
		
		\textbf{Propriedade} & \textbf{Tipo} & \textbf{Obrigatório?} & \centerline{\textbf{Descrição}} \\\hline  	
		
		feedback & Texto & x & Comentário/Feedback dado a uma linha de código específica. \\\hline	
		line & Code Line & x & Linha de código à qual o feedback foi atribuído. \\\hline			
		
	\end{tabularx}	
\end{table}


%%%%%       CLASSE TEXT FEEDBACK        %%%%%%%%%%%%%%%%%%%%%%%%%%%%%%%%%%%%%%%%%%
\subsection{Text Feedback} \label{Text Feedback}
\begin{table}[H]
	\footnotesize
	\noindent
	\begin{tabularx}{\textwidth}{|X|X|c|p{7.8cm}|}   \hline \rowcolor[rgb]{0.8,0.8,0.8}
		
		\textbf{Propriedade} & \textbf{Tipo} & \textbf{Obrigatório?} & \centerline{\textbf{Descrição}} \\\hline  	
		
		feedback & Texto & x & Comentário/Feedback dado a uma submissão em formato de texto. \\\hline	
		alternative & Test Alternative Response & {} & Questão objetiva de prova à qual o feedback foi atribuído. \\\hline	
		response & Test TextResponse & {} & Questão discursiva de prova à qual o feedback foi atribuído. \\\hline				
		submission & Submission & {} & Submissão de tarefa à qual o feedback foi atribuído. \\\hline	
		activity & Activity & {} & Atividade externa à qual o feedback foi atribuído. \\\hline	
	\end{tabularx}	
\end{table}

\newpage

\section{Subsistema Discussion}
%%%%%       CLASSE NEWS        %%%%%%%%%%%%%%%%%%%%%%%%%%%%%%%%%%%%%%%%%%
\subsection{News} \label{News}
\begin{table}[H]
	\footnotesize
	\begin{tabularx}{\textwidth}{|X|X|c|p{7.8cm}|}   \hline \rowcolor[rgb]{0.8,0.8,0.8}
		
		\textbf{Propriedade} & \textbf{Tipo} & \textbf{Obrigatório?} & \centerline{\textbf{Descrição}} \\\hline  	
		
		title & Texto & x & Título da notícia. \\\hline		
		content & Texto & x & Conteúdo da notícia \\\hline		
		created\_at & Data & x & Timestamp da data em que a notícia foi criada. \\\hline
		author & User & x & Usuário que publicou a notícia. \\\hline
		classroom & Classroom & x & Turma na qual a notícia foi publicada. \\\hline	
		
	\end{tabularx}	
\end{table}

%%%%%       CLASSE FORUM TOPIC        %%%%%%%%%%%%%%%%%%%%%%%%%%%%%%%%%%%%%%%%%%
\subsection{Forum Topic} \label{Forum Topic}
\begin{table}[H]
	\footnotesize
	\begin{tabularx}{\textwidth}{|X|X|c|p{7.8cm}|}   \hline \rowcolor[rgb]{0.8,0.8,0.8}
		
		\textbf{Propriedade} & \textbf{Tipo} & \textbf{Obrigatório?} & \centerline{\textbf{Descrição}} \\\hline  	
		
		title & Texto & x & Título do tópico. \\\hline		
		content & Texto & x & Conteúdo do tópico \\\hline		
		created\_at & Data & x & Timestamp da data em que o tópico foi criado. \\\hline	
		author & User & x & Usuário que publicou o tópico. \\\hline
		classroom & Classroom & x & Turma na qual o tópico foi publicado. \\\hline
		replies & Forum Reply & {} & Réplicas criadas para o tópico. \\\hline
		
	\end{tabularx}	
\end{table}

%%%%%       CLASSE FORUM REPLY        %%%%%%%%%%%%%%%%%%%%%%%%%%%%%%%%%%%%%%%%%%
\subsection{Forum Reply} \label{Forum Reply}
\begin{table}[H]
	\footnotesize
	\begin{tabularx}{\textwidth}{|X|X|c|p{7.8cm}|}   \hline \rowcolor[rgb]{0.8,0.8,0.8}
		
		\textbf{Propriedade} & \textbf{Tipo} & \textbf{Obrigatório?} & \centerline{\textbf{Descrição}} \\\hline  	
		
		content & Texto & x & Conteúdo da resposta \\\hline		
		created\_at & Data & x & Timestamp da data em que a resposta foi criada. \\\hline	
		author & User & x & Usuário que publicou a réplica. \\\hline
		classroom & Classroom & x & Turma na qual a réplica foi publicada. \\\hline
		topic & Forum Topic & x & Tópico no qual a réplica foi publicada. \\\hline
		
	\end{tabularx}	
\end{table}

%%%%%       CLASSE LIVE QUESTION       %%%%%%%%%%%%%%%%%%%%%%%%%%%%%%%%%%%%%%%%%%
\subsection{Live Question} \label{Live Question}
\begin{table}[H]
	\footnotesize
	\begin{tabularx}{\textwidth}{|X|X|c|p{7.8cm}|}   \hline \rowcolor[rgb]{0.8,0.8,0.8}
		
		\textbf{Propriedade} & \textbf{Tipo} & \textbf{Obrigatório?} & \centerline{\textbf{Descrição}} \\\hline  	
		
		question & Texto & x & Dúvida do usuário. \\\hline		
		created\_at & Data & x &  Timestamp do momento em que a dúvida foi criada. \\\hline		
		classroom & Classroom & x &  Turma na qual a dúvida foi feita. \\\hline		
		user & User & {} &  Usuário que criou a dúvida. \\\hline		
		
	\end{tabularx}	
\end{table}

\newpage

\section{Subsistema Log}

%%%%%       CLASSE LOG        %%%%%%%%%%%%%%%%%%%%%%%%%%%%%%%%%%%%%%%%%%
\subsection{Log} \label{Log}
\begin{table}[H]
	\footnotesize
	\begin{tabularx}{\textwidth}{|X|X|c|p{7.8cm}|}   \hline \rowcolor[rgb]{0.8,0.8,0.8}
		
		\textbf{Propriedade} & \textbf{Tipo} & \textbf{Obrigatório?} & \centerline{\textbf{Descrição}} \\\hline  	
		
		content & Texto & x & Descrição do log. \\\hline		
		category & Texto & x & Indica o tipo de ação a que o log se refere. \\\hline
		created\_at & Data & x & Timestamp do momento em que o log foi criado. \\\hline
		user & User & x & Usuário que realizou a ação que deu origem ao log \\\hline	
		
	\end{tabularx}	
\end{table}
