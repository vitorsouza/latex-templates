\chapter{Requisitos de Usuário}
\label{sec-requisitos}

\vitor{Listar os principais \textit{stakeholders} do projeto no parágrafo abaixo e preencher as tabelas de requisitos funcionais, não funcionais e regras de negócio, com atenção aos identificadores e \textit{labels}.}

Tomando por base o contexto do sistema descrito na Seção~\ref{sec-minimundo} e considerando como principais \textit{stakeholders} \hl{(listar aqui os principais stakeholders do projeto)}, foram identificados os seguintes requisitos de usuário e regras de negócio.



% Define contador e identificador para requisitos funcionais.
% Usar \RF\label{rf-nome-do-label} para cada requisito definido.
\newcounter{rfcount}
\renewcommand*\therfcount{RF-\arabic{rfcount}}
\newcommand*\RF{\refstepcounter{rfcount}\therfcount}
\setcounter{rfcount}{0}

% Tabela de requisitos funcionais.
\begin{longtable}{|c|p{9cm}|c|p{2.2cm}|}
	\caption{Requisitos Funcionais.}
	\label{tbl-requisitos-rfs} \\\hline 
	
	% Cabeçalho e repetição do mesmo em cada nova página. Manter como está.
	\rowcolor{lightgray}
	\textbf{ID} & \textbf{Descrição} & \textbf{Prioridade} & \textbf{Depende} \\\hline	
	\endfirsthead
	\hline
	\rowcolor{lightgray}
	\textbf{ID} & \textbf{Descrição} & \textbf{Prioridade} & \textbf{Depende} \\\hline	
	\endhead
	
	% Especificar os requisitos abaixo, substituindo os exemplos.
	\RF\label{rf-exemplo-01} &  O sistema deve ...  & Alta & \\\hline
	
	\RF\label{rf-exemplo-02} &  O sistema deve ...  & Alta & \ref{rf-exemplo-01} \\\hline

	\RF\label{rf-exemplo-03} &  O sistema deve ...  & Alta & \ref{rf-exemplo-01}, \ref{rf-exemplo-02} \\\hline
\end{longtable}



% Define contador e identificador para requisitos não funcionais.
% Usar \RNF\label{rnf-nome-do-label} para cada requisito definido.
\newcounter{rnfcount}
\renewcommand*\thernfcount{RNF-\arabic{rnfcount}}
\newcommand*\RNF{\refstepcounter{rnfcount}\thernfcount}
\setcounter{rnfcount}{0}

% Tabela de requisitos não funcionais.
\begin{longtable}{|c|p{5.3cm}|c|c|c|}
	\caption{Requisitos Não Funcionais.}
	\label{tbl-requisitos-rnfs} \\\hline 
	
	% Cabeçalho e repetição do mesmo em cada nova página. Manter como está.
	\rowcolor{lightgray}
	\textbf{ID} & \textbf{Descrição} & \textbf{Categoria} & \textbf{Escopo} & \textbf{Prioridade} \\\hline		
	\endfirsthead
	\hline
	\rowcolor{lightgray}
	\textbf{ID} & \textbf{Descrição} & \textbf{Categoria} & \textbf{Escopo} & \textbf{Prioridade} \\\hline		
	\endhead
	
	% Especificar os requisitos abaixo, substituindo os exemplos.
	\RNF\label{rnf-exemplo-01} & Descrição sucinta. & Característica & Funcionalidade & Baixa \\\hline 	

	\RNF\label{rnf-exemplo-02} & Descrição sucinta. & de & ou & Média \\\hline 	

	\RNF\label{rnf-exemplo-03} & Descrição sucinta. & qualidade & Sistema & ou Alta \\\hline
\end{longtable}



% Define contador e identificador para regras de negócio.
% Usar \RN\label{rn-nome-do-label} para cada regra definida.
\newcounter{rncount}
\renewcommand*\therncount{RN-\arabic{rncount}}
\newcommand*\RN{\refstepcounter{rncount}\therncount}
\setcounter{rncount}{0}

% Tabela de regras de negócio.
\begin{longtable}{|c|p{9cm}|c|p{2.2cm}|}
	\caption{Regras de Negócio.}
	\label{tbl-requisitos-rns} \\\hline 
	
	% Cabeçalho e repetição do mesmo em cada nova página. Manter como está.
	\rowcolor{lightgray}
	\textbf{ID} & \textbf{Descrição} & \textbf{Prioridade} & \textbf{Depende} \\\hline	
	\endfirsthead
	\hline
	\rowcolor{lightgray}
	\textbf{ID} & \textbf{Descrição} & \textbf{Prioridade} & \textbf{Depende} \\\hline	
	\endhead
	
	% Especificar as regras abaixo, substituindo os exemplos.
	\RN\label{rn-exemplo-01} & Descrição sucinta. & Baixa & \ref{rf-exemplo-01} \\\hline

	\RN\label{rn-exemplo-02} & Descrição sucinta. & Média & \ref{rf-exemplo-01}, \ref{rf-exemplo-02} \\\hline

	\RN\label{rn-exemplo-03} & Descrição sucinta. & ou Alta & \ref{rf-exemplo-03} \\\hline
\end{longtable}
