\chapter{ Identificação de Subsistemas}
\label{sec-subsistemas}

Tal projeto foi dividido em 4 subsistemas a fim de facilitar o desenvolvimento de funcionalidades que são, de certa forma, isoladas, originando assim os subsistemas Auth, Log, Classroom e Board.
A Figura~\ref{figura-subsistema} mostra os subsistemas identificados no contexto do presente projeto e suas interdependências, enquanto a Tabela~\ref{tabela-subsistema} apresenta breve descrição de cada um deles.


\begin{figure}[h]
	\centering
	\includegraphics[width=\textwidth]{figuras/subsistemas.png}
	\caption{Diagrama de Pacotes e os Subsistemas Identificados.}
	\label{figura-subsistema}
\end{figure} 

\begin{table}[h]
	\centering	
	\vspace{0.5cm}
	\caption{ Subsistemas}
	\label{tabela-subsistema}
	\begin{tabular}{|p{3cm}|p{12cm}|}  \hline \rowcolor[rgb]{0.8,0.8,0.8}
		
		Subsistema & Descrição \\\hline 
		
		Classroom & Subsistema contendo as funcionalidades relacionadas diretamente às turmas.  \\\hline
		
		Board & Envolve todas as funcionalidades relativas ao gerenciamento dos itens do board. \\\hline
		
		Board Interactions & Corresponde a todas as interações que os usuários podem realizar com os itens do board, como submeter tarefas ou realizar provas. \\\hline
		
		Feedback & Permite a avaliação de atividades e a visualização das notas do aluno em uma turma. \\\hline
		
		Discussions & Subsistema contendo as funcionalidades relativas ao fórum de discussão, às notícias e às \textit{Live Questions}. \\\hline
		
		Logs & Este sistema consiste na visualização de logs gerados a partir das atividades dos usuários do sistema. \\\hline  
		
	\end{tabular}	
\end{table}
