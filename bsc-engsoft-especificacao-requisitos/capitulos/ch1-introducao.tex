\chapter{Introdução}
\label{sec-intro}


Este documento apresenta os requisitos de usuário e a análise dos requisitos do sistema Khoeus. A atividade de análise de requisitos foi conduzida aplicando-se técnicas de modelagem de casos de uso, modelagem de classes e levando em consideração as boas práticas de programação relacionadas ao desenvolvimento de aplicações web. Os modelos apresentados foram elaborados usando a UML.

Na Seção~\ref{sec-proposito} deste documento descreve-se de forma geral o sistema, apresentando superficialmente suas principais características. Já na Seção~\ref{sec-minimundo} estão apresentadas as funcionalidades do Khoeus, bem como suas dependências. A Seção~\ref{sec-requisitos} lista os requisitos de usuário do sistema (funcionais, não funcionais e suas regras de negócio). Além disso, a Seção~\ref{sec-subsistemas} exibe os subsistemas considerados para este projeto enquanto a Seção~\ref{sec-caso-de-uso} apresenta o modelo de casos de uso, incluindo descrições de atores, os diagramas de casos de uso e suas respectivas descrições. Na Seção~\ref{sec-modelo-estrutural} estão apresentados os modelos conceituais estruturais do sistema na forma de diagramas de classes. Por fim, a Seção~\ref{sec-dicionario} apresenta o  dicionário do projeto, contendo as definições das classes identificadas.
