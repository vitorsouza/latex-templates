\chapter{Introdução}
\label{sec-intro}

Este documento apresenta a especificação dos requisitos do sistema \emph{\imprimirtitulo}.

\vitor{Adicione ao parágrafo acima o que foi utilizado na especificação dos requisitos. Ex.: ``Esta especificação foi construída aplicando-se técnicas de levantamento de requisitos, bem como modelagem de casos de uso e de classes utilizando a linguagem UML.'' Adaptar esta frase caso outras técnicas (ex.: modelagem de objetivos, estórias de usuário, BDD, etc.) tenham sido utilizadas.}

A Seção~\ref{sec-requisitos} descreve os requisitos levantados junto aos \textit{stakeholders}.
A Seção~\ref{sec-subsistemas} explica a divisão em subsistemas, descrevendo brevemente cada um deles. 
A Seção~\ref{sec-casos-de-uso} apresenta o modelo de casos de uso, incluindo descrições de atores, os diagramas de casos de uso e suas respectivas descrições. 
A Seção~\ref{sec-modelo-estrutural} traz os modelos conceituais estruturais do sistema na forma de diagramas de classes.  Por fim, a Seção~\ref{sec-dicionario} detalha o dicionário do projeto, contendo as definições das classes identificadas.
