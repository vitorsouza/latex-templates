\chapter{Requisitos de Usuário}
\label{sec-requisitos}

\vitor{Está sendo utilizado o \textit{template} que lista as estórias de usuário. Existe também um \textit{template} para os requisitos funcionais, basta modificar o arquivo \texttt{ch4-} que está sendo incluído no documento principal.}

\vitor{Listar os principais \textit{stakeholders} do projeto no parágrafo abaixo e preencher as tabelas de estórias de usuário, requisitos não-funcionais e regras de negócio, com atenção aos identificadores e \textit{labels}.}

Tomando por base o contexto do sistema descrito na Seção~\ref{sec-minimundo} e considerando como principais \textit{stakeholders} \hl{(listar aqui os principais stakeholders do projeto)}, foram identificadas estórias de usuário, requisitos não funcionais e regras de negócio.

As estórias de usuário são apresentadas na Tabela~\ref{tbl-requisitos-uss}, os requisitos não-funcionais na Tabela~\ref{tbl-requisitos-rnfs} e as regras de negócio globais (aquelas que não são caracterizadas como critérios de aceitação de estórias de usuário específica) na Tabela~\ref{tbl-requisitos-rns}. As tabelas são apresentadas nas páginas a seguir, em formato paisagem, para melhor visualização.

% Mostra as tabelas em formato paisagem.
\begin{landscape}

% Define contador e identificador para estórias de usuário.
% Usar \US\label{rf-nome-do-label} para cada requisito definido.
\newcounter{uscount}
\renewcommand*\theuscount{US-\arabic{uscount}}
\newcommand*\US{\refstepcounter{uscount}\theuscount}
\setcounter{uscount}{0}

% Tabela de requisitos funcionais.
\begin{longtable}{|c|p{8cm}|p{9.7cm}|c|p{2.2cm}|}
	\caption{Estórias de Usuário.}
	\label{tbl-requisitos-uss} \\\hline 
	
	% Cabeçalho e repetição do mesmo em cada nova página. Manter como está.
	\rowcolor{lightgray}
	\textbf{ID} & \textbf{Descrição} & \textbf{Critérios de Aceitação} & \textbf{Prioridade} & \textbf{Depende} \\\hline	
	\endfirsthead
	\hline
	\rowcolor{lightgray}
	\textbf{ID} & \textbf{Descrição} & \textbf{Critérios de Aceitação} & \textbf{Prioridade} & \textbf{Depende} \\\hline	
	\endhead
	
	% Especificar as estórias de usuário abaixo, substituindo os exemplos.
	\US\label{us-exemplo-01} & 
		\hl{Genérico: } Como \hl{ator}, quero \hl{função}, para \hl{finalidade}. &
		\parbox{9.7cm}{
		\begin{enumerate}[leftmargin=15mm,label=-- CA\arabic*:]\itemsep-2mm
			\item Critério de aceitação 1;
			\item Critério de aceitação 2;
			\item Critério de aceitação N.
		\end{enumerate}
		} & Baixa & \\\hline
	
	\US\label{us-exemplo-02} & 
		\hl{Cadastro/CRUD: } Como \hl{ator}, quero cadastrar \hl{objeto}, para \hl{finalidade}. &
		\parbox{9.7cm}{
		\begin{enumerate}[leftmargin=15mm,label=-- CA\arabic*:]\itemsep-2mm
			\item Para cadastrar um \hl{objeto} devem ser informados \hl{campos};
			\item Um \hl{objeto} não pode ter as seguintes informações alteradas: \hl{campos};
			\item A consulta pode ser feita por \hl{campos};
			\item \hl{Objetos} relacionados a \hl{outros objetos} não podem ser excluídos;
			\item Alterações ou exclusões devem ser comunicadas e/ou deve ser requerida confirmação.
		\end{enumerate}
		} & Média & \ref{us-exemplo-01} \\\hline

	\US\label{us-exemplo-03} & 
		\hl{Consulta: } Como \hl{ator}, quero consultar \hl{informações}, para \hl{finalidade}. &
		\parbox{9.7cm}{
		\begin{enumerate}[leftmargin=15mm,label=-- CA\arabic*:]\itemsep-2mm
			\item Para realizar a consulta, o usuário deve informar como parâmetros: \hl{parâmetros};
			\item Devem ser apresentadas as seguintes informações para o usuário: \hl{campos}.
		\end{enumerate}
		} & ou Alta & \ref{us-exemplo-01}, \ref{us-exemplo-02} \\\hline
\end{longtable}



% Define contador e identificador para requisitos não funcionais.
% Usar \RNF\label{rnf-nome-do-label} para cada requisito definido.
\newcounter{rnfcount}
\renewcommand*\thernfcount{RNF-\arabic{rnfcount}}
\newcommand*\RNF{\refstepcounter{rnfcount}\thernfcount}
\setcounter{rnfcount}{0}

% Tabela de requisitos não funcionais.
\begin{longtable}{|c|p{14.3cm}|c|c|c|}
	\caption{Requisitos Não Funcionais.}
	\label{tbl-requisitos-rnfs} \\\hline 
	
	% Cabeçalho e repetição do mesmo em cada nova página. Manter como está.
	\rowcolor{lightgray}
	\textbf{ID} & \textbf{Descrição} & \textbf{Categoria} & \textbf{Escopo} & \textbf{Prioridade} \\\hline		
	\endfirsthead
	\hline
	\rowcolor{lightgray}
	\textbf{ID} & \textbf{Descrição} & \textbf{Categoria} & \textbf{Escopo} & \textbf{Prioridade} \\\hline		
	\endhead
	
	% Especificar os requisitos abaixo, substituindo os exemplos.
	\RNF\label{rnf-exemplo-01} & Descrição sucinta. & Característica & Funcionalidade & Baixa \\\hline 	

	\RNF\label{rnf-exemplo-02} & Descrição sucinta. & de & ou & Média \\\hline 	

	\RNF\label{rnf-exemplo-03} & Descrição sucinta. & qualidade & Sistema & ou Alta \\\hline
\end{longtable}



% Define contador e identificador para regras de negócio.
% Usar \RN\label{rn-nome-do-label} para cada regra definida.
\newcounter{rncount}
\renewcommand*\therncount{RN-\arabic{rncount}}
\newcommand*\RN{\refstepcounter{rncount}\therncount}
\setcounter{rncount}{0}

% Tabela de regras de negócio.
\begin{longtable}{|c|p{18cm}|c|p{2.2cm}|}
	\caption{Regras de Negócio.}
	\label{tbl-requisitos-rns} \\\hline 
	
	% Cabeçalho e repetição do mesmo em cada nova página. Manter como está.
	\rowcolor{lightgray}
	\textbf{ID} & \textbf{Descrição} & \textbf{Prioridade} & \textbf{Depende} \\\hline	
	\endfirsthead
	\hline
	\rowcolor{lightgray}
	\textbf{ID} & \textbf{Descrição} & \textbf{Prioridade} & \textbf{Depende} \\\hline	
	\endhead
	
	% Especificar as regras abaixo, substituindo os exemplos.
	\RN\label{rn-exemplo-01} & Descrição sucinta. & Baixa & \ref{us-exemplo-01} \\\hline

	\RN\label{rn-exemplo-02} & \hl{Devem ser incluídas apenas regras de negócio globais.} & Média & \ref{us-exemplo-01}, \ref{us-exemplo-02} \\\hline

	\RN\label{rn-exemplo-03} & \hl{Regras de negócio associadas a uma estória de usuário específica devem constar como critério de aceitação da referida estória.} & ou Alta & \ref{us-exemplo-03} \\\hline
\end{longtable}

% Fim do formato paisagem.
\end{landscape}
