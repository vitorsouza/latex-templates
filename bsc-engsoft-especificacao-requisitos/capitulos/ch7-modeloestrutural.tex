\chapter{Modelo Estrutural}
\label{sec-modelo-estrutural}

O modelo conceitual estrutural visa capturar e descrever as informações (classes, associações e atributos) que o sistema deve representar para prover as funcionalidades descritas na seção anterior. A seguir, são apresentados os diagramas de classes de cada um dos subsistemas identificados no contexto deste projeto. Na Seção~\ref{sec-dicionario} – Dicionário de Projeto – são apresentadas as descrições das classes, atributos e operações presentes nos diagramas apresentados nesta seção.

 Vale ressaltar que algumas associações são consideradas obrigatórias nos dois sentidos pois considera-se que o tempo decorrido entre a criação de ambas é extremamente pequeno. Além disso, por estarmos considerando \textbf{entidades}, uma só faria sentido quando estivesse associado à outra.

\section{Subsistema Classroom}


A Figura~\ref{figura-classroom-classe} apresenta o diagrama de classes do subsistema Classroom.

\begin{figure}[h]
	\centering
	\includegraphics[width=\textwidth]{figuras/diagrama-classe-classroom.png}
	\caption{Diagrama de Classes do subsistema Classroom.}
	\label{figura-classroom-classe}
\end{figure} 

\newpage

\section{Subsistema Board}


A Figura~\ref{figura-board-classe} apresenta o diagrama de classes do subsistema Board.

\begin{figure}[h]
	\centering
	\includegraphics[width=\textwidth]{figuras/diagrama-classe-board.png}
	\caption{Diagrama de Classes do Subsistema Board.}
	\label{figura-board-classe}
\end{figure} 
\newpage

\section{Subsistema Board Interactions}


A Figura~\ref{figura-board-interactions-classe} apresenta o diagrama de classes do subsistema Board Interactions.

\begin{figure}[h]
	\centering
	\includegraphics[width=\textwidth]{figuras/diagrama-classe-board-interactions.png}
	\caption{Diagrama de Classes do Subsistema Board Interactions.}
	\label{figura-board-interactions-classe}
\end{figure} 
\newpage

\section{Subsistema Feedback}


A Figura~\ref{figura-feedback-classe} apresenta o diagrama de classes do subsistema Feedback.

\begin{figure}[h]
	\centering
	\includegraphics[width=\textwidth]{figuras/diagrama-classe-feedback.png}
	\caption{Diagrama de Classes do Subsistema Feedback.}
	\label{figura-feedback-classe}
\end{figure} 
\newpage

\section{Subsistema Discussion}


A Figura~\ref{figura-discussion-classe} apresenta o diagrama de classes do subsistema Discussion.

\begin{figure}[h]
	\centering
	\includegraphics[width=0.6\textwidth]{figuras/diagrama-classe-discussion.png}
	\caption{Diagrama de Classes do Subsistema Discussion.}
	\label{figura-discussion-classe}
\end{figure} 
\newpage

\section{Subsistema Log}


A Figura~\ref{figura-log-classe} apresenta o diagrama de classes do subsistema Log.

\begin{figure}[h]
	\centering
	\includegraphics[scale=0.7]{figuras/diagrama-classe-log.png}
	\caption{Diagrama de Classes do Subsistema Log.}
	\label{figura-log-classe}
\end{figure} 

\newpage




\newpage