% ==============================================================================
% PG - Nome do Aluno
% Capítulo 3 - Contribuição
% ==============================================================================
\chapter{Contribuição do Trabalho}
\label{sec-contribuicao}

Este capítulo deve apresentar a principal contribuição do trabalho. Caso o aluno e orientador desejem, o título do capítulo pode ser alterado para referenciar diretamente a contribuição (por exemplo, PIS: Plataforma para Integração de Serviços; Um Sistema para Controle de Processos da UFES, Solução de Otimização para Carregamento de Contêineres; etc.)

O capítulo deve ser estruturado em seções de forma a apresentar de forma clara e com todas as informações necessárias, a contribuição do trabalho. Por exemplo, caso a contribuição produzida seja um sistema de informação, espera-se que sejam apresentados seus requisitos, funcionalidades, modelos (p.ex., modelo estrutural, modelo da arquitetura, etc.) e telas do sistema. Caso seja uma plataforma, espera-se que a plataforma como um todo seja apresentada e que seus componentes sejam descritos sejam apropriadamente.

