% ==============================================================================
% PG - Nome do Aluno
% Capítulo 5 - Considerações Finais
% ==============================================================================
\chapter{Conclusão}
\label{sec-conclusoes}

Neste capítulo devem ser realizadas as considerações finais do trabalho, sendo apresentadas suas principais contribuições, limitações, lições aprendidas durante o desenvolvimento do trabalho, dificuldades enfrentadas e perspectivas de trabalhos futuros. O capítulo deve ter entre 3 e 5 páginas.


%%% Início de seção. %%%
\section{Considerações Finais}
\label{sec-conclusoes-consideracoes}

Esta seção deve apresentar um texto de fechamento do trabalho, devendo incluir considerações sobre o trabalho desenvolvido, suas limitações, contribuições, experiência adquirida pelo aluno e lições aprendidas ao longo do desenvolvimento, bem como dificuldades enfrentadas durante o desenvolvimento do trabalho. Nesta seção é preciso mostrar claramente a relação entre os resultados produzidos no trabalho e os objetivos estabelecidos no Capítulo~\ref{sec-intro} 


%%% Início de seção. %%%
\section{Trabalhos Futuros}
\label{sec-conclusoes-trabalhosfuturos}

Nesta seção devem ser identificados trabalhos futuros que poderão ser realizados a partir dos resultados obtidos até o momento no trabalho. Idealmente, trabalhos futuros não devem apenas ser citados. Recomenda-se discutir aspectos sobre como podem ser realizados e por que é importante que sejam realizados (que benefícios podem ser obtidos com sua realização).
