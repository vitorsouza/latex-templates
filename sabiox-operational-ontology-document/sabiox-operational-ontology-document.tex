\documentclass[table,usenames,dvipsnames]{article}

%
% LaTeX Packages
%

% Determines the paper size and the size of the contents.
\usepackage[a4paper, total={170mm, 257mm}]{geometry}

% To insert blank spaces in macros.
\usepackage{xspace}

% To insert colored comments so authors can collaborate on the content.
\usepackage[colorinlistoftodos, textwidth=20mm, textsize=footnotesize]{todonotes}
\newcommand{\camila}[1]{\todo[author=\textbf{Camila},color=green!30,caption={},inline]{#1}}
\newcommand{\vitor}[1]{\todo[author=\textbf{Vítor},color=red!30,caption={},inline]{#1}}

% To use the \hl{} command to highlight pieces of text.
\usepackage{soulutf8}

% To change the margins in paragraphs.
\usepackage{changepage}

% To define a box with colored background.
\usepackage{tcolorbox}

% To insert hyperlinks and use \nameref{}. Hides the default red boxes for links.
\usepackage[hidelinks]{hyperref}

% To include tables that span across pages.
\usepackage{longtable}

% To have cells span multiple rows in tables.
\usepackage{multirow}

% To include listings with the command \lstinputlisting{}.
\usepackage{listings}
\usepackage{caption}
\definecolor{mygray}{rgb}{0.5,0.5,0.5}
\lstset{
	xrightmargin=0.05\linewidth,
	basicstyle=\scriptsize\linespread{0.8},
	breaklines=true,
	breakatwhitespace=false,
	showstringspaces=false,
	keepspaces=true,
	showspaces=false,
	showtabs=false, 
	numbers=left,
	numbersep=5pt,
	numberstyle=\tiny\color{mygray}, 
	rulecolor=\color{black},	
	tabsize=6,
	inputencoding=utf8,
	extendedchars=true,
	literate=%
	{é}{{\'{e}}}1
	{è}{{\`{e}}}1
	{ê}{{\^{e}}}1
	{ë}{{\¨{e}}}1
	{É}{{\'{E}}}1
	{Ê}{{\^{E}}}1
	{û}{{\^{u}}}1
	{ù}{{\`{u}}}1
	{â}{{\^{a}}}1
	{à}{{\`{a}}}1
	{á}{{\'{a}}}1
	{ã}{{\~{a}}}1
	{Á}{{\'{A}}}1
	{Â}{{\^{A}}}1
	{Ã}{{\~{A}}}1
	{ç}{{\c{c}}}1
	{Ç}{{\c{C}}}1
	{õ}{{\~{o}}}1
	{ó}{{\'{o}}}1
	{ô}{{\^{o}}}1
	{Õ}{{\~{O}}}1
	{Ó}{{\'{O}}}1
	{Ô}{{\^{O}}}1
	{î}{{\^{i}}}1
	{Î}{{\^{I}}}1
	{í}{{\'{i}}}1
	{Í}{{\~{Í}}}1
}



%
% Macros.
%

% The name of the method.
\newcommand{\sabiox}{SABiOx\xspace}
\newcommand{\sabioxfull}{Extended Systematic Approach for Building Ontologies\xspace}

% Document meta-data.
\newcommand{\ontologyacronym}{YA-O\xspace}
\newcommand{\ontologyname}{Yet Another Ontology\xspace}
\newcommand{\authorname}{\sabiox User\xspace}
\newcommand{\documentversion}{0.1\xspace}


% Title Page
\title{\ontologyacronym: \ontologyname
	\\{\large Operational Ontology Document}
	\\{\normalsize Version: \documentversion}}
\author{\authorname}



% Document contents.
\begin{document}
\maketitle


\section{Introduction}

This document presents the design of the operational ontology \ontologyname (\ontologyacronym) as result of the Design phase of \sabiox: the \sabioxfull~\cite{aguiar-souza:report24}.

This document is organized as follows:
	Section~\ref{sec-premises} presents the ontology premises, namely codification language and vocabularies to reuse;
	Section~\ref{sec-coderules} specifies the codification rules to be followed;
	Section~\ref{sec-architecture} describes the operational ontology architecture.


\section{Operational Ontology Premises:}
\label{sec-premises}

These are the premises for the operational ontology \ontologyacronym:


\subsection{Codification Language:}
\label{sec-premises-language}

OWL.\footnote{\url{https://www.w3.org/TR/owl-features/}}



\subsection{Vocabularies:}
\label{sec-premises-vocabularies}

\begin{center}
	\begin{small}
		\begin{longtable}{ p{20mm} p{140mm} }
			\hline
			\textbf{Prefix} & \textbf{Namespace URL} \\\hline
			
			\texttt{owl}  & \url{http://www.w3.org/2002/07/owl#} \\\hline
			\texttt{rdf}  & \url{http://www.w3.org/1999/02/22-rdf-syntax-ns#} \\\hline
			\texttt{rdfs} & \url{http://www.w3.org/2000/01/rdf-schema#} \\\hline
			\texttt{xml}  & \url{http://www.w3.org/XML/1998/namespace} \\\hline
			\texttt{xsd}  & \url{http://www.w3.org/2001/XMLSchema#} \\\hline
		\end{longtable}
	\end{small}
\end{center}



\section{Codification Rules:}
\label{sec-coderules}

\begin{center}
	\begin{small}
		\begin{longtable}{ p{20mm} p{140mm} }
			\hline
			\textbf{Rule} & \textbf{Definition} \\\hline
			
			(Rule name) 
				& (Rule definition) 
				\\\hline			
		\end{longtable}
	\end{small}
\end{center}



\section{Architecture:}
\label{sec-architecture}

\subsection{Ontologies:}
\label{sec-architecture-ontologies}

\begin{center}
	\begin{small}
		\begin{longtable}{ p{20mm} p{140mm} }
			\hline
			\textbf{Prefix} & \textbf{Description} \\\hline
			
			\texttt{(prefix)}
				& (Description)
				\\\hline
		\end{longtable}
	\end{small}
\end{center}



\subsection{Taxonomy:}
\label{sec-architecture-taxonomy}

\begin{itemize}
	\item Concepts of the \emph{kind} and \emph{category} types are anchored in the \textsf{Class} class of OWL;

	\item Concepts of the \emph{subkind} type are anchored to the respective kinds of the domain according to the conceptual model;

	\item Concepts of type \emph{quality} are anchored to the \textsf{ObjectProperty} property of OWL, except for the concept name which is anchored in the \textsf{DatypeProperty} property of OWL.
\end{itemize}


% Bibliography
\bibliographystyle{alpha}
\bibliography{bibliography}

\end{document}          
