
\chapter{Modelagem FrameWeb}
\label{sec-frameweb}

\emph{\imprimirtitulo} é um sistema Web cuja arquitetura utiliza \textit{frameworks} comuns no desenvolvimento para esta plataforma. Desta forma, como alternativa/complemento aos modelos apresentados na Seção~\ref{sec-componentes}, o sistema pode ser modelado utilizando a abordagem FrameWeb~\cite{souza-celebratingfalbo20}.

\vitor{Caso a afirmação acima não seja verdadeira para o seu sistema, você pode modificar o parágrafo para dizer ``Suponha que \emph{\imprimirtitulo} seja um sistema Web cuja ...''}

A Tabela~\ref{tabela-frameworks} indica os \textit{frameworks} presentes na arquitetura do sistema que se encaixam em cada uma das categorias de \textit{frameworks} que FrameWeb dá suporte. Em seguida, os modelos FrameWeb são apresentados para cada subsistema da arquitetura.

\begin{footnotesize}
	\begin{longtable}{|c|c|}
		\caption{\textit{Frameworks} da arquitetura do sistema separados por categoria.}
		\label{tabela-frameworks}\\\hline
		
		\rowcolor{lightgray}
		\textbf{Categoria de \textit{Framework}} & \textbf{\textit{Framework} Utilizado} \\\hline 
		\endfirsthead
		\hline
		\rowcolor{lightgray}
		\textbf{Categoria de \textit{Framework}} & \textbf{\textit{Framework} Utilizado} \\\hline 
		\endhead

		Controlador Frontal & \hl{JSF} \\\hline

		Injeção de Dependências & \hl{CDI} \\\hline

		Mapeamento Objeto/Relacional & \hl{JPA} \\\hline

		Segurança & \hl{JAAS} \\\hline
	\end{longtable}
\end{footnotesize}


\section{Subsistema SS01}
\label{sec-frameweb-ss01}

\vitor{Incluir diagramas e explicações em texto para os modelos de Entidade, Persistência (se utilizado o padrão DAO), Aplicação e Navegação de FrameWeb. Deste último, bastam 2 exemplos.}


\section{Subsistema SS02}
\label{sec-frameweb-ss02}

\vitor{Incluir diagramas e explicações em texto para os modelos de Entidade, Persistência (se utilizado o padrão DAO), Aplicação e Navegação de FrameWeb. Deste último, bastam 2 exemplos.}
