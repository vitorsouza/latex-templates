% ==============================================================================
% PG - Nome do Aluno
% Capítulo 1 - Introdução
% ==============================================================================
\chapter{Introdução}
\label{sec-intro}

O Capítulo de Introdução deve apresentar o contexto, motivação e justificativa do trabalho, seus objetivos, método de desenvolvimento e organização da monografia. Deve conter de 3 a 5 páginas.


%%% Início de seção. %%%
\section{Motivação e Justificativa}
\label{sec-intro-motjus}

A \textbf{Motivação} apresenta as circunstâncias que levaram à escolha do tema abordado e ao desenvolvimento do que é proposto no trabalho. A \textbf{Justificativa} apresenta o porquê da escolha do tema e do problema tratado e destaca a relevância do trabalho, referindo-se a estudos anteriores sobre o tema, ressaltando suas eventuais limitações e destacando a necessidade de se continuar explorando o assunto.


%%% Início de seção. %%%
\section{Objetivos}
\label{sec-intro-obj}

Nesta subseção, deve ser descrito o objetivo geral do trabalho, detalhando em seguida, seus objetivos específicos. O \textbf{Objetivo Geral} expressa a finalidade principal do trabalho: para quê? Deve ter coerência direta com o tema do trabalho e ser apresentado em uma frase que inicie com um verbo no infinitivo. O objetivo geral do trabalho está relacionado ao resultado principal do trabalho. Os \textbf{Objetivos Específicos} apresentam os detalhes ou desdobramentos do objetivo geral que levam a resultados intermediários e relevantes para alcançar o objetivo geral. Sempre será mais de um objetivo específico, todos iniciando com verbo no infinitivo.


%%% Início de seção. %%%
\section{Método de Desenvolvimento do Trabalho}
\label{sec-intro-met}

Nesta subseção, deve ser apresentado o \textbf{Método de Desenvolvimento} (ou o \textbf{Método de Pesquisa}, quando for o caso) utilizado no trabalho. Aqui são apresentadas as atividades realizadas e os procedimentos/técnicas que foram usados durante o desenvolvimento do trabalho.


%%% Início de seção. %%%
\section{Organização da Monografia}
\label{sec-intro-organizacao}

Por fim, a última subseção da monografia apresenta a estrutura do texto. Por exemplo, para este documento esta seção poderia conter o seguinte texto:

Além desta introdução, este modelo de monografia é composto por outros cinco capítulos:

\begin{itemize}
	\item O Capítulo~\ref{sec-referencial} apresenta os aspectos relativos ao conteúdo teórico relevante para o trabalho;
	
	\item O Capítulo~\ref{sec-contribuicao} apresenta a principal contribuição do trabalho;
	
	\item O Capítulo~\ref{sec-avaliacao} apresenta a avaliação da proposta, quando a mesma tiver sido realizada e requeira uma descrição detalhada;
	
	\item O Capítulo~\ref{sec-conclusoes} apresenta as considerações finais do trabalho;
	
	\item O Capítulo~\ref{sec-dicaslatex} traz dicas básicas para escrita de textos científicos em \latex.
\end{itemize}


