% ==============================================================================
% Relatório de Estudo Dirigido - Nome do Aluno
% Seção 1 - Introdução
% ==============================================================================
\section{Introdução}
\label{sec-intro}

\instrucoes{\begin{itemize}
		\item No arquivo \texttt{.tex} principal, procure por \texttt{(*)} e insira seus dados para a capa do documento;
		\item Leia com atenção todas as instruções;
		\item Caso não tenha muita experiência com \LaTeX, leia o Apêndice~\ref{sec-dicaslatex};
		\item Antes de entregar a versão final do documento, exclua todas as instruções e o Apêndice~\ref{sec-dicaslatex};
		\item Use BibTeX para referências bibliográficas e não altere a formatação do modelo básico abnTeX2.\footnote{\url{http://www.abntex.net.br}.}
\end{itemize}

\textbf{Especificamente com relação a esta seção:} na introdução, descreva brevemente o propósito do Estudo Dirigido. Alguns propósitos comuns:
\begin{itemize}
	\item Fazer uma revisão bibliográfica de modo a definir um problema de pesquisa;
	\item Leitura de artigos selecionados para apropriação do referencial teórico a ser utilizado;
	\item Fazer uma revisão bibliográfica (sistemática ou não) em busca de trabalhos relacionados para comparação;
	\item Condução de outras etapas da pesquisa, caso as etapas acima já tenham sido concluídas.
\end{itemize}

Mencione, também, qual o assunto que foi tema do Estudo Dirigido e faça um breve resumo do que foi efetivamente realizado. Conclua descrevendo como o restante do texto foi organizado.}
