% ==============================================================================
% Relatório de Estudo Dirigido - Nome do Aluno
% Seção 2 - Visão Geral da Pesquisa
% ==============================================================================
\section{Visão Geral da Pesquisa}
\label{sec-visao}

Esta seção traz uma visão geral do problema de pesquisa a ser tratado ao longo do curso de pós-graduação.

\instrucoes{\begin{itemize}
		\item Preencher as subseções abaixo de acordo com as instruções;
		\item A definição de assunto, tema, problema, objetivos, hipótese e justificativa costumam ser trabalhadas no início do curso, portanto no primeiro Estudo Dirigido. No entanto, mantenha esta seção nos estudos dirigidos seguintes, de modo a evoluir e refinar seu conteúdo, ou, mesmo que não haja alterações, apenas para o relatório ficar completo;
		\item Caso deseje, pode complementar esta seção com o método de pesquisa, à medida que avança no curso.
	\end{itemize}}


\subsection{Assunto} 
\label{sec-visao-assunto}

\instrucoes{dentro da área da Computação, indicar em qual subárea o trabalho será conduzido. Ex.: Engenharia Web, Redes de Sensores Sem Fio, Evolução de Software, etc.}


\subsection{Tema} 
\label{sec-visao-tema}

\instrucoes{dentro do assunto especificado anteriormente, qual o tema específico que será abordado? Ex.: ``Representação ontológica de \textit{frameworks} Web'', ``Disseminação de códigos em redes de sensores sem fio orientada a máquinas virtuais'', etc.}



\subsection{Problema de Pesquisa}
\label{sec-visao-problema}

\instrucoes{segundo \citeonline{moresi:report03}:
	\begin{itemize}
		\item Na acepção científica, ``problema é qualquer questão não resolvida e que é objeto de discussão, em qualquer domínio do conhecimento'';
		\item Problema é uma questão que a pesquisa pretende responder e todo o processo de pesquisa irá girar em torno de sua solução.
		\item Na escolha de um problema de pesquisa, o pesquisador deve fazer as seguintes perguntas:
		\begin{itemize}
			\item o problema é original?
			\item o problema é relevante?
			\item ainda que seja ``interessante'', é adequado?
			\item existem possibilidades reais para executar tal pesquisa?
			\item existem recursos financeiros que viabilizarão a execução do projeto?
			\item existe tempo suficiente para investigar tal questão?
		\end{itemize}
		\item O problema sinaliza o foco que o pesquisador dará à pesquisa;
		\item Uma pergunta errada dificilmente vai levar a uma resposta certa;
		\item O problema deve ser formulado como pergunta, para facilitar a identificação do que se deseja pesquisar;
		\item O problema tem que ter dimensão viável: deve ser restrito para permitir a sua viabilidade. O problema formulado de forma ampla poderá tornar inviável a realização da pesquisa;
		\item O problema deve ter clareza: os termos adotados devem ser definidos para esclarecer os significados com que estão sendo usados na pesquisa;
		\item O problema deve ser preciso: além de definir os termos é necessário que sua aplicação esteja delimitada.
	\end{itemize}
}


\subsection{Objetivos}
\label{sec-visao-objetivos}

Este trabalho tem como objetivo geral <<objetivo geral>>. Esse objetivo geral pode ser detalhado nos seguintes objetivos específicos:

\begin{itemize}
	\item <<objetivo específico 1>>;
	\item <<objetivo específico 2>>;
	\item <<etc...>>.
\end{itemize}

\instrucoes{segundo \citeonline{moresi:report03}:
	\begin{itemize}
		\item Sintetiza o que se pretende alcançar com a pesquisa;
		\item Deve estar coerente com o problema proposto;
		\item O objetivo normalmente comporta uma hipótese do trabalho.
		\item A especificação do objetivo responde às questões \emph{para que?} e \emph{para quem?};
		\item Os objetivos de um trabalho são desmembrados em duas partes principais:
		\begin{itemize}
			\item O objetivo geral é a síntese do que se pretende alcançar;
			\item Os objetivos específicos explicitam os detalhes e são um desdobramento do objetivo geral.
		\end{itemize}
		\item Os objetivos informam os resultados que se pretende alcançar ou a contribuição que a pesquisa vai efetivamente proporcionar;
	\end{itemize}

	Segundo \citeonline{wazlawick:book14}:
	\begin{itemize}
		\item O objetivo geral e os objetivos específicos devem ser expressos na forma de uma condição não trivial cujo sucesso possa ser verificado ao final do trabalho;
		\item Deve-se tomar cuidado com certos verbos que determinam objetivos cuja verificação é trivial e, portanto, inadequada. Ex.: propor, estudar, apresentar, etc.
	\end{itemize}

	Além disso:
	\begin{itemize}
		\item Objetivos Específicos:
		\begin{itemize}
			\item Não são etapas do trabalho, mas subprodutos;
			\item Não se deve confundir os objetivos específicos com os passos do método de pesquisa.
		\end{itemize}
		\item Objetivos devem ser verificáveis ao final do trabalho;
		\item O enunciado de um objetivo deve começar com um verbo no infinitivo e esse verbo deve indicar uma ação passível de mensuração.
	\end{itemize}
}



\subsection{Hipótese}
\label{sec-visao-hipotese}

\instrucoes{segundo \citeonline{moresi:report03}:
	\begin{itemize}
		\item São suposições colocadas como respostas plausíveis e provisórias para o problema de pesquisa;
		\item São provisórias porque poderão ser confirmadas ou refutadas para o desenvolvimento da pesquisa;
		\item O processo de pesquisa estará voltado para a procura de evidências que comprovem, sustentem ou refutem a afirmativa feita na hipótese;
		\item A hipótese define até aonde você quer chegar e, por isso, será a diretriz de todo o processo de investigação;
		\item A hipótese é sempre uma afirmação, uma resposta possível ao problema proposto.
	\end{itemize}
}



\subsection{Justificativa} 
\label{sec-visao-justificativa}

\instrucoes{segundo \citeonline{moresi:report03}:
	\begin{itemize}
		\item Esta etapa inclui a reflexão sobre o porquê da realização da pesquisa procurando identificar as razões da preferência pelo tema escolhido e sua importância em relação a outros temas:
		\begin{itemize}
			\item O tema é relevante e, se é, por quê?
			\item Quais os pontos positivos que você percebe na abordagem proposta?
			\item Que vantagens e benefícios vocêe pressupõe que sua pesquisa irá proporcionar?
		\end{itemize}
		\item A justificativa deverá convencer quem for ler a monografia, com relação à importância e à relevância da pesquisa em discussão.
	\end{itemize}

	Segundo \citeonline{wazlawick:book14}:
	\begin{itemize}
		\item Em geral, a justificativa do tema aparece na contextualização do trabalho, em que se tenta justificar ao leitor que o problema escolhido é realmente relevante;
		\item Mais importante: justificar a escolha dos objetivos e das hipóteses;
		\item Uma hipótese de trabalho deve estar apoiada em uma boa justificativa que apresente evidências de que vale a pena investir tempo e recursos na tentativa de comprovar a hipótese;
		\item Deve-se apresentar alguma evidência de que a linha de pesquisa seguida pode levar a bons resultados;
		\item Essas evidências podem ser referências a outros trabalhos que eventualmente mostraram algum tipo de resultado e que apontem para a viabilidade da hipótese escolhida.
	\end{itemize}
}
