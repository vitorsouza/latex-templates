% Exemplo de tabela 04:
\begin{table}[h]
	\caption{Exemplo que utiliza o pacote \texttt{tabularx}, extraído de um artigo ainda não publicado.}
	\label{tbl-intro-exemplo04}
	\centering\tiny\def\tabularxcolumn#1{m{#1}}
	\begin{tabularx}{\columnwidth}{ >{\centering}X | >{\centering}X | >{\hsize=1.2\hsize\centering}X | >{\hsize=0.9\hsize\centering}X | >{\hsize=0.9\hsize\centering\arraybackslash}X }
		\hline
		\textbf{Applied Criteria} & \textbf{Analyzed Content} & \textbf{Initial\\Occurrences} & \textbf{Final Results} & \textbf{Reduction (\%)} \\
		\hline
		Duplicate Removal & Title, authors and year & 903 & 420 & 54,84\% \\ 
		\hline 
		IC and ECs & Title, abstract and keywords & 420 & 130 & 69,05\% \\ 
		\hline 
		IC and ECs & Full text & 130 & 117 & 10\% \\ 
		\hline 
		Final Results & -- & 903 & 117 & 87,04\% \\ 
		\hline 
	\end{tabularx}
\end{table}