\documentclass[table,usenames,dvipsnames]{article}

%
% LaTeX Packages
%

% Determines the paper size and the size of the contents.
\usepackage[a4paper, total={170mm, 257mm}]{geometry}

% To insert blank spaces in macros.
\usepackage{xspace}

% To insert colored comments so authors can collaborate on the content.
\usepackage[colorinlistoftodos, textwidth=20mm, textsize=footnotesize]{todonotes}
\newcommand{\camila}[1]{\todo[author=\textbf{Camila},color=green!30,caption={},inline]{#1}}
\newcommand{\vitor}[1]{\todo[author=\textbf{Vítor},color=red!30,caption={},inline]{#1}}

% To use the \hl{} command to highlight pieces of text.
\usepackage{soulutf8}

% To change the margins in paragraphs.
\usepackage{changepage}

% To define a box with colored background.
\usepackage{tcolorbox}

% To insert hyperlinks and use \nameref{}. Hides the default red boxes for links.
\usepackage[hidelinks]{hyperref}

% To include tables that span across pages.
\usepackage{longtable}

% To have cells span multiple rows in tables.
\usepackage{multirow}

% To include listings with the command \lstinputlisting{}.
\usepackage{listings}
\usepackage{caption}
\definecolor{mygray}{rgb}{0.5,0.5,0.5}
\lstset{
	xrightmargin=0.05\linewidth,
	basicstyle=\scriptsize\linespread{0.8},
	breaklines=true,
	breakatwhitespace=false,
	showstringspaces=false,
	keepspaces=true,
	showspaces=false,
	showtabs=false, 
	numbers=left,
	numbersep=5pt,
	numberstyle=\tiny\color{mygray}, 
	rulecolor=\color{black},	
	tabsize=6,
	inputencoding=utf8,
	extendedchars=true,
	literate=%
	{é}{{\'{e}}}1
	{è}{{\`{e}}}1
	{ê}{{\^{e}}}1
	{ë}{{\¨{e}}}1
	{É}{{\'{E}}}1
	{Ê}{{\^{E}}}1
	{û}{{\^{u}}}1
	{ù}{{\`{u}}}1
	{â}{{\^{a}}}1
	{à}{{\`{a}}}1
	{á}{{\'{a}}}1
	{ã}{{\~{a}}}1
	{Á}{{\'{A}}}1
	{Â}{{\^{A}}}1
	{Ã}{{\~{A}}}1
	{ç}{{\c{c}}}1
	{Ç}{{\c{C}}}1
	{õ}{{\~{o}}}1
	{ó}{{\'{o}}}1
	{ô}{{\^{o}}}1
	{Õ}{{\~{O}}}1
	{Ó}{{\'{O}}}1
	{Ô}{{\^{O}}}1
	{î}{{\^{i}}}1
	{Î}{{\^{I}}}1
	{í}{{\'{i}}}1
	{Í}{{\~{Í}}}1
}



%
% Macros.
%

% The name of the method.
\newcommand{\sabiox}{SABiOx\xspace}
\newcommand{\sabioxfull}{Extended Systematic Approach for Building Ontologies\xspace}

% Document meta-data.
\newcommand{\ontologyacronym}{YA-O\xspace}
\newcommand{\ontologyname}{Yet Another Ontology\xspace}
\newcommand{\authorname}{\sabiox User\xspace}
\newcommand{\documentversion}{0.1\xspace}


% Title Page
\title{\ontologyacronym: \ontologyname
	\\{\large Reference Ontology Document}
	\\{\normalsize Version: \documentversion}}
\author{\authorname}



% Document contents.
\begin{document}
\maketitle


\section{Introduction}

This document presents the reference ontology \ontologyname (\ontologyacronym) as result of the Setup and Capture phases of \sabiox: the \sabioxfull~\cite{aguiar-souza:report24}.

This document is organized as follows:
	Section~\ref{sec-premises} presents the ontology premises, namely modeling language, foundational ontology, concept criteria and ontologies to reuse;
	Section~\ref{sec-modularization} specifies how the ontology is modularized;
	Sections~\ref{sec-sub1}--\ref{sec-sub3} present the reference ontology diagram, axioms and concepts for each module of the ontology.


\section{Reference Ontology Premises:}
\label{sec-premises}

These are the premises for the reference ontology \ontologyacronym:


\subsection{Modeling Language:}
\label{sec-premises-language}

OntoUML~\cite{guizzardi-et-al:er18}.



\subsection{Foundational Ontology:}
\label{sec-premises-foundational}

UFO~\cite{guizzardi-et-al:aoj22}.



\subsection{Concept Criteria:}
\label{sec-premises-criteria}

% Define counter and ID for functional requirements.
\newcounter{crcount}
\renewcommand*\thecrcount{CR-\arabic{crcount}}
\newcommand*\CR{\refstepcounter{crcount}\thecrcount}
\setcounter{crcount}{0}

% Define a macro for a functional requirement.
\newcommand{\crit}[2]{\CR\label{#1} & #2 \\\hline}


\begin{center}
	\begin{small}
		\begin{longtable}{ p{20mm} p{140mm} }
			\hline
			\textbf{ID} & \textbf{Description} \\\hline
			
			\crit{cr-first-example}{
				Lorem ipsum dolor sit amet, consectetur adipiscing elit. Nunc purus sem, rutrum eget enim eu, vestibulum maximus nisl. Phasellus tristique purus a magna aliquet dapibus.
			}
			
			\crit{cr-second-example}{
				Maecenas nunc diam, accumsan id commodo a, maximus ultrices metus. Sed molestie imperdiet massa, quis mollis felis facilisis ac. In vel turpis et diam sodales commodo eget sed enim.
			}
		\end{longtable}
	\end{small}
\end{center}



\subsection{Ontologies to Reuse:}
\label{sec-premises-reuse}

\begin{center}
	\begin{small}
		\begin{longtable}{ p{15mm} p{70mm} p{20mm} p{15mm} p{30mm} }
			\hline
			\textbf{Prefix} & \textbf{Definition} & \textbf{Foundation} & \textbf{Type} & \textbf{Analysis} \\\hline
			
			XPTO
				& Lorem ipsum dolor sit amet, consectetur adipiscing elit. Nunc purus sem, rutrum eget enim eu, vestibulum maximus nisl.
				& UFO
				& Core
				& Phasellus tristique purus a magna aliquet dapibus.
				\\\hline
			
			YPTO
				& Maecenas nunc diam, accumsan id commodo a, maximus ultrices metus. Sed molestie imperdiet massa, quis mollis felis facilisis ac.
				& BFO
				& Domain
				& In vel turpis et diam sodales commodo eget sed enim.
				\\\hline
		\end{longtable}
	\end{small}
\end{center}



\section{Modularization:}
\label{sec-modularization}

\begin{center}
	\begin{small}
		\begin{longtable}{ p{30mm} p{130mm} }
			\hline
			\textbf{Sub-ontology} & \textbf{Definition} \\\hline
			
			Sub-ontology 1
				& Lorem ipsum dolor sit amet, consectetur adipiscing elit. Nunc purus sem, rutrum eget enim eu, vestibulum maximus nisl.
				\\\hline
			
			Sub-ontology 2
				& Maecenas nunc diam, accumsan id commodo a, maximus ultrices metus. Sed molestie imperdiet massa, quis mollis felis facilisis ac.
				\\\hline
		\end{longtable}
	\end{small}
\end{center}

\begin{center}
	\includegraphics[width=.85\textwidth]{figures/fig-modularization}	
\end{center}



\section{Sub-ontology 1:}
\label{sec-sub1}

\begin{center}
	\includegraphics[width=.85\textwidth]{figures/fig-ontology}	
\end{center}


\subsection{Axioms:}
\label{sec-sub1-axioms}

\begin{center}
	\begin{small}
		\begin{longtable}{ p{80mm} p{80mm} }
			\hline
			\textbf{Axiom} & \textbf{Description} \\\hline
			
			$\forall x,y,z:Concept\ 1,\ prop(x, y) \wedge prop(y, z) \rightarrow prop(x, z)$
				& Lorem ipsum dolor sit amet, consectetur adipiscing elit. Nunc purus sem, rutrum eget enim eu, vestibulum maximus nisl.
				\\\hline
		\end{longtable}
	\end{small}
\end{center}


\subsection{Dictionary of Concepts:}
\label{sec-sub1-axioms}

\begin{center}
	\begin{small}
		\begin{longtable}{ p{30mm} p{130mm} }
			\hline
			\textbf{Concept} & \textbf{Definition} \\\hline
			
			Concept 1
				& Lorem ipsum dolor sit amet, consectetur adipiscing elit. Nunc purus sem, rutrum eget enim eu, vestibulum maximus nisl.
				\\\hline
			
			Concept 2
				& Maecenas nunc diam, accumsan id commodo a, maximus ultrices metus. Sed molestie imperdiet massa, quis mollis felis facilisis ac.
				\\\hline
		\end{longtable}
	\end{small}
\end{center}



\section{Sub-ontology 2:}
\label{sec-sub2}

...



\section{Sub-ontology 3:}
\label{sec-sub3}

...


% Bibliography
\bibliographystyle{alpha}
\bibliography{bibliography}

\end{document}          
